\chapter{Appendix}

\begin{lemma}
	\label{identities_for_measure_theoretic_boundaries}
	The following statments hold for Borel-measurable sets.
	\begin{enumerate}
		\item 
		\label{item_disjoint_sets}
		If $ A, B, C \subset \flattorus $ are mutually disjoint, then
		\begin{equation*}	
			\hm^{ d - 1 } \left(
				\partial_{ \ast } ( A \cup B ) 
				\triangle  
				\left( \partial_{ \ast } A \triangle \partial_{ \ast } B \right)
			\right)
			=
			0
			\quad 
			\text{ and }
			\quad
			\hm^{ d - 1 } \left(
				\partial_{ \ast } A
				\cap
				\partial_{ \ast } B
				\cap
				\partial_{ \ast } C
			\right)
			=
			0.
		\end{equation*}
		
		\item 
		\label{rewriting_boundary_via_interfaces}
		If $ ( \Omega_{ i } )_{ i = 1 , \dotsc , P } $ is a partition of $ \flattorus $, then
		\begin{equation*}
			\hm^{ d - 1 } \left(
				\partial_{ \ast } \Omega_{ i }
				\triangle
				\bigcup_{ j \neq i }
				\partial_{ \ast } \Omega_{ i }
				\cap
				\partial_{ \ast } \Omega_{ j }
			\right) = 0.
		\end{equation*}
	\end{enumerate}
\end{lemma}

\begin{remark}
	In a maybe less confusing way, the first and third result yield that up to 
	sets 
	of $ \hm^{ d - 1 } $-measure zero, we have
	$  \partial_{ \ast } ( A \cup B ) 
	=  
	 \partial_{ \ast } A \triangle \partial_{ \ast } B
	$
	and
	$
	\partial_{ \ast } \Omega_{ i }
	=
	\bigcup_{ j \neq i }
	\partial_{ \ast } \Omega_{ i }
	\cap
	\partial_{ \ast } \Omega_{ j }
	$
	under the given assumptions.
\end{remark}

\begin{proof}
	Let us first prove \cref{item_disjoint_sets}. Initially assume that $ x \in \partial_{ \ast } ( A \cup B ) $. By definition this implies that
	\begin{equation*}
		\limsup_{ r \to 0 }
		\frac{ \lm^{ d } ( B_{ r } ( x ) \cap ( A \cup B ) ) }{ r^{ d } }
		>0.
	\end{equation*}
	Then we must either have
	\begin{equation*}
		\limsup_{ r \to 0 }
		\frac{ \lm^{ d } ( B_{ r } ( x ) \cap A ) }{ r^{ d } }
		>0
		\quad 
		\text{ or }
		\quad
		\limsup_{ r \to 0 }
		\frac{ \lm^{ d } ( B_{ r } ( x ) \cap  B ) }{ r^{ d } }
		>0,
	\end{equation*}
	so let us without loss of generality assume the former. Since
	\begin{equation*}
		\limsup_{ r \to 0 }
		\frac{ \lm^{ d } ( B_{ r } ( x ) \setminus A ) }{ r^{ d } }
		\geq
		\limsup_{ r \to 0 }
		\frac{ \lm^{ d } ( B_{ r } ( x ) \setminus ( A \cup B ) ) }{ r^{ d } }
		>0,
	\end{equation*}
	we thus have $ x \in \partial_{ \ast } A $.
	We now want to show that $ x \notin \partial_{ \ast } B $.
	Combining Thm.~5.14 and Lemma~5.5 in \cite{evans_gariepy_measure_theory_and_fine_props}, we can rewrite for some measurable set $ N $ with $ \hm^{ d- 1 } ( N ) = 0 $ the measure theoretic boundary of $ A $ as
	\begin{equation}
		\label{measure_theoretic_boundary_almost_half_boundary}
		\partial_{ \ast } A 
		=
		\partial^{ 1/2 } A \cup N 
		\coloneqq
		\left\{
		x 
		\, \colon \,
		\lim_{ r \to 0 }
		\frac{ \lm^{ d } ( B_{ r } ( x ) \cap A ) }{ \alpha( d ) r^{ d } }
		=
		\frac{ 1 }{ 2 }
		=
		\lim_{ r \to 0 }
		\frac{ \lm^{ d } ( B_{ r } ( x ) \setminus A ) }{ \alpha( d ) r^{ d } }
		\right\}
		\cup N,
	\end{equation}
	where $ \alpha ( d ) \coloneqq \lm^{ d } ( B_{ 1 } ( 0 ) ) $ is the volume of the unit ball in $ d $ dimensions.
	Thus assume $ x \in \partial^{ 1/2 } A \cap \partial^{ 1/2 } B $. Then since $ A $ and $ B $ are disjoint, we would have
	\begin{equation*}
		\limsup_{ r \to 0 }
		\frac{ \lm^{ d } ( B_{ r } ( x ) \setminus  ( A \cup B ) ) }{ \alpha( d 
		) r^{ d } }
		=
		\limsup_{ r \to 0 }
		\frac{ \lm^{ d } ( B_{ r } ( x ) \setminus A ) - \lm^{ d } ( B_{ r } ( x ) \cap B ) }{ \alpha( d ) r^{ d } }
		=
		0,
	\end{equation*}
	which contradicts $ x \in \partial_{ \ast } ( A \cup B ) $. Thus up to a set of $(d-1)$-dimensional Hausdorff-measure zero, we have 
	$ \partial_{ \ast } ( A \cup B ) \subseteq \partial_{ \ast } A \triangle \partial_{ \ast } B $.
	
	For the other inclusion, suppose that $ x \in \partial_{\ast } A \triangle \partial_{ \ast } B $ and without loss of generality that $ x \in \partial_{ \ast } A \setminus \partial_{ \ast } B $. Then it must either hold that 
	\begin{equation}
		\label{one_equation_has_to_hold}
		\limsup_{ r \to 0 } \frac{ \lm^{ d } ( B_{ r } ( x ) \cap B ) }{ r^{ d } } = 0
		\quad \text{or} \quad
		\limsup_{ r \to 0 } \frac{ \lm^{ d } ( B_{ r } ( x ) \setminus B ) }{ r^{ d } } = 0.
	\end{equation}
	But since $ x \in \partial_{ \ast } A $, the latter can not be true since $ A $ and $ B $ are disjoint and therefore
	\begin{equation*}
		0
		<
		\limsup_{ r \to 0 } \frac{ \lm^{ d } ( B_{ r } ( x ) \cap A ) }{ r^{ d } } 
		\leq
		\limsup_{ r \to 0 } \frac{ \lm^{ d } ( B_{ r } ( x ) \setminus B ) }{ 
		r^{ d } },
	\end{equation*}
	thus the former from (\ref{one_equation_has_to_hold}) has to be true. We 
	now estimate
	\begin{align*}
		\limsup_{ r \to 0 }
		\frac{ \lm^{ d } ( B_{ r } ( x ) \cap ( A \cup B ) ) }{ r^{ d } }
		& \geq
		\limsup_{ r \to 0 }
		\frac{ \lm^{ d } ( B_{ r } ( x ) \cap A ) }{ r^{ d } }
		> 0
		\shortintertext{and}
		\limsup_{ r \to 0 }
		\frac{ \lm^{ d } ( B_{ r } ( x ) \setminus ( A \cup B ) ) }{ r^{ d } }
		& =
		\limsup_{ r \to 0 }
		\frac{ \lm^{ d } ( B_{ r } ( x ) \setminus A ) - \lm^{ d } ( B_{ r } ( x ) \cap B ) }{ r^{ d } }
		\\
		& =
		\limsup_{ r \to 0 } \frac{ \lm^{ d } ( B_{ r } ( x ) \setminus A ) }{ r^{ d } }
		> 0,
	\end{align*}
	which proves $ x \in \partial_{ \ast } ( A \cup B ) $.
	
	The second equality is an immediate consequence of observation (\ref{measure_theoretic_boundary_almost_half_boundary}).
	
	Now let us prove \Cref{rewriting_boundary_via_interfaces}. It follows immediately that  
	\begin{equation*}
		\partial_{ \ast } \Omega_{ i }
		\supseteq
		\bigcup_{ j \neq i }
		\partial_{ \ast } \Omega_{ i }
		\cap
		\partial_{ \ast } \Omega_{ j }.
	\end{equation*}
	For the other inclusion, suppose $ x \in \partial_{ \ast } \Omega_{ i } $. Since
	\begin{equation*}
		0 <
		\limsup_{ r \to 0 }
		\frac{ \lm^{ d } ( B_{ r } ( x ) \setminus \Omega_{ i }) }{r^{ d } }
		=
		\limsup_{ r \to 0 } \frac{ \lm^{ d } \left( B_{ r } ( x ) \cap \bigcup_{ j \neq i } \Omega_{ j } \right) }{ r^{ d } },
	\end{equation*}
	we find some $ j \neq i $ such that
	\begin{equation*}
		\limsup_{ r \to 0 } \frac{ \lm^{ d } ( B_{ r } ( x ) \cap \Omega_{ j } ) }{ r^{ d } } > 0.
	\end{equation*}
	Since
	\begin{equation*}
		\limsup_{ r \to 0 } \frac{ \lm^{ d } ( B_{ r } ( x ) \setminus \Omega_{ j } ) }{ r^{ d } } 
		\geq
		\limsup_{ r \to 0 } \frac{ \lm^{ d } ( B_{ r } ( x ) \cap \Omega_{ i } ) }{ r^{ d } } > 0,
	\end{equation*}
	we therefore have $ x \in \partial_{ \ast } \Omega_{ j } $, which finishes the proof.
\end{proof}

\begin{proof}[Proof of \Cref{rewriting_variation_of_psi_i}]
	Without loss of generality we can assume that $ i = 1 $ and $ \sigma_{ 1, j } \leq \sigma_{ 1, j + 1 } $ for all $ 1 \leq j \leq P - 1 $. 
	All equalities for sets in this proof hold up to a set of $ 
	(d-1)$-dimensional Hausdorff measure zero. First let us apply the 
	Fleming--Rishel co-area formula to find that for a given open set $ U 
	\subseteq \flattorus $, we have
	\begin{align*}
		\abs{ \nabla \psi_{ 1 } } ( U ) 
		& =
		\int_{ 0 }^{ \infty }
		\hm^{ d - 1 } \left(
		\partial_{ \ast } \left(
		\left\{
		x \in U 
		\, \colon \,
		\psi_{ 1 } \leq s 
		\right\}
		\right)
		\right)
		\dd{ s }
		\\
		& =
		\sum_{ j = 1 }^{ P -1 }
		\int_{ \sigma_{ 1, j } }^{ \sigma_{ 1, j + 1 } }
		\hm^{ d - 1 } \left(
		\partial_{ \ast } \left(
		\bigcup_{ k = 1 }^{ j }
		U \cap \Omega_{ k }
		\right)
		\right)
		\dd{ s }
		\\
		& =
		\sum_{ j = 1 }^{ P - 1 }
		\left(\sigma_{ 1, j +1} - \sigma_{ 1 , j } \right)
		\hm^{ d - 1 } \left(
		\partial_{ \ast } \left(
		\bigcup_{ k = 1 }^{ j }
		U \cap \Omega_{ k }
		\right)
		\right)
		\eqqcolon I. 
	\end{align*}
	We claim that we have
	\begin{equation}
		\label{rewriting_measure_theoretic_boundary_of_union}
		\partial_{ \ast } \left(
		\bigcup_{ k = 1 }^{ j }
		U \cap \Omega_{ k }
		\right)
		=
		\bigcup_{ k = 1 }^{ j }
		\bigcup_{ l = j + 1 }^{ P }
		U
		\cap
		\partial_{ \ast } \Omega_{ k }
		\cap
		\partial_{ \ast } \Omega_{ l }.
	\end{equation}
	Since $ ( \Omega_{ k } \cap U )_{ k } $ is again a partition of $ U $, we 
	omit taking the intersection with $ U $ for a briefer notation. The proof 
	will be done via an induction over $ j $. For $ j = 1 $, this is exactly 
	\Cref{rewriting_boundary_via_interfaces} from 
	\Cref{identities_for_measure_theoretic_boundaries}. For the induction step, 
	assume that the claim (\ref{rewriting_measure_theoretic_boundary_of_union}) 
	holds for all $ j' \leq j $. Again by 
	\Cref{identities_for_measure_theoretic_boundaries},
	\Cref{item_disjoint_sets} and \Cref{rewriting_boundary_via_interfaces}, we 
	can compute that
	\begin{align*}
		& \partial_{ \ast } \left(
		\bigcup_{ k = 1 }^{ j + 1 }
		\Omega_{ k }
		\right)
		=
		\left(
		\partial_{ \ast } \left(
		\bigcup_{ k = 1 }^{ j }
		\Omega_{ k }
		\right)
		\cup
		\partial_{ \ast } \Omega_{ j + 1 }
		\right)
		\setminus
		\left(
		\partial_{ \ast } \left(
		\bigcup_{ k = 1 }^{ j }
		\Omega_{ k }
		\right)
		\cap
		\partial_{ \ast } \Omega_{ j + 1 }
		\right)
		\\
		={} &
		\left(
		\left(
		\bigcup_{ k = 1 }^{ j }
		\bigcup_{ l = j + 1 }^{ P }
		\partial_{ \ast } \Omega_{ k }
		\cap
		\partial_{ \ast } \Omega_{ l }
		\right)
		\cup
		\bigcup_{ k \neq j + 1 }
		\partial_{ \ast } \Omega_{ j + 1 }
		\cap
		\partial_{ \ast } \Omega_{ k }
		\right)
		\setminus
		\left(
		\bigcup_{ k = 1 }^{ j }
		\bigcup_{ l = j + 1 }^{ P }
		\partial_{ \ast } \Omega_{ k }
		\cap
		\partial_{ \ast } \Omega_{ l }
		\cap
		\partial_{ \ast }\Omega_{ j + 1 }
		\right).
	\end{align*}
	By \Cref{identities_for_measure_theoretic_boundaries} \Cref{item_disjoint_sets}, we can write the second term as 
	\begin{equation*}
		\bigcup_{ k = 1 }^{ j }
		\bigcup_{ l = j + 1 }^{ P }
		\partial_{ \ast } \Omega_{ k }
		\cap
		\partial_{ \ast } \Omega_{ l }
		\cap
		\Omega_{ j + 1 }
		=
		\bigcup_{ k = 1 }^{ j } 
		\partial_{ \ast } \Omega_{ k }
		\cap
		\partial_{ \ast } \Omega_{ j + 1 }.
	\end{equation*}
	By carefully considering both equations, we get
	\begin{align}
		& 
		\notag
		\partial_{ \ast }
		\left(
		\bigcup_{ k = 1 }^{ j + 1 }
		\Omega_{ k }
		\right)
		\\
		\notag
		={} &
		\left(
		\left(
		\bigcup_{ k = 1 }^{ j }
		\bigcup_{ l = j + 2 }^{ P }
		\partial_{ \ast } \Omega_{ k }
		\cap
		\partial_{ \ast } \Omega_{ l }
		\right)
		\cup
		\bigcup_{ k = j + 2 }^{ P }
		\partial_{ \ast } \Omega_{ j + 1 }
		\cap
		\partial_{ \ast } \Omega_{ k }
		\right)
		\setminus
		\bigcup_{ k = 1 }^{ j }
		\partial_{ \ast } \Omega_{ j + 1 }
		\cap
		\partial_{ \ast } \Omega_{ k }
		\\
		\label{here_complement_obscolete}
		={} &
		\left(
		\bigcup_{ k = 1 }^{ j + 1 }
		\bigcup_{ l = j + 2 }^{ P }
		\partial_{ \ast } \Omega_{ k }
		\cap
		\partial_{ \ast } \Omega_{ l }
		\right)
		\setminus
		\bigcup_{ k = 1 }^{ j }
		\partial_{ \ast } \Omega_{ j + 1 }
		\cap
		\partial_{ \ast } \Omega_{ k }.
	\end{align}
	Since by \Cref{identities_for_measure_theoretic_boundaries} 
	\Cref{item_disjoint_sets}, we have for $ 1 \leq k \leq j +1 $, $ j+ 2 \leq 
	l \leq P $ and $ 1 \leq m \leq j $ that
	\begin{equation*}
		\hm^{ d - 1 }
		\left(
		\partial_{ \ast } \Omega_{ k }
		\cap
		\partial_{ \ast } \Omega_{ l } 
		\cap
		\partial_{ \ast } \Omega_{ j + 1 }
		\cap
		\partial_{ \ast } \Omega_{ m }
		\right) = 0,
	\end{equation*}
	taking the complement in the term (\ref{here_complement_obscolete}) becomes 
	obsolete and thus we obtain the desired result 
	(\ref{rewriting_measure_theoretic_boundary_of_union}).
	Therefore we have again by \Cref{identities_for_measure_theoretic_boundaries} \Cref{item_disjoint_sets} that
	\begin{align*}
		I &=
		\sum_{ j = 1 }^{ P - 1 }
		\left(
		\sigma_{ 1 , j + 1 } - \sigma_{ 1 , j }
		\right)
		\left(
		\sum_{ k = 1 }^{ j }
		\sum_{ l = j + 1 }^{ P }
		\hm^{ d - 1 } \left( 
		U 
		\cap 
		\partial_{ \ast } \Omega_{ k } 
		\cap 
		\partial_{ \ast } \Omega_{ l }
		\right)
		\right)
		\\
		& =
		\sum_{ 1 \leq k < l \leq P }
		\hm^{ d - 1 } \left(
		U \cap \partial_{ \ast } \Omega_{ k }
		\cap
		\partial_{ \ast } \Omega_{ l }
		\right)
		\left(
		\sum_{ j = k }^{ l - 1 }
		\sigma_{ 1 , j + 1 } - \sigma_{ 1 , j }
		\right)
		\\
		& =
		\sum_{ 1 \leq k < l \leq P }
		\hm^{ d - 1 } \left(
		U \cap 
		\partial_{ \ast } \Omega_{ k }
		\cap
		\partial_{ \ast } \Omega_{ l }
		\right)
		\left(
		\sigma_{ 1, l } - \sigma_{ 1 , k }
		\right).
	\end{align*}
	Since we assumed without loss of generality that $ \sigma_{ 1 , j } \leq 
	\sigma_{ 1, j + 1 } $, we know that
	for $ k < l $,
	\begin{equation*}
		\sigma_{ 1 , l } - \sigma_{ 1 , k }
		=
		\abs{ \sigma_{ 1 , l } - \sigma_{ 1 , k } }
	\end{equation*}
	which finishes the proof.
\end{proof}

\begin{lemma}
	\label{supremum_of_measures_lemma}
	Let $ \mu $ be a regular positive Borel measure on some open set $ \Omega $ and let $ B_{ 1 }, \dotsc, B_{ m } $ be $ \mu $-finite disjoint Borel subsets of $ \Omega $. Moreover let $ c_{ i }^{ h } $ for $ 1 \leq i \leq m $ and $ 1 \leq h \leq k $ be non-negative coefficients. Define the measures
	\begin{equation*}
		\mu_{ h } 
		\coloneqq
		\sum_{ i = 1 }^{ m }
		c_{ i }^{ h }
		\mu |_{ B_{ i } }
		\quad
		\text{ and }
		\quad 
		\nu 
		\coloneqq
		\sum_{ i = 1 }^{ m }
		\max_{ h } c_{ i }^{ h }
		\mu |_{ B_{ i } }.
	\end{equation*}
	Then we have
	\begin{equation*}
		\nu 
		=
		\bigvee_{ h = 1 }^{ k }
		\mu_{ h }.
	\end{equation*}
\end{lemma}

\begin{proof}
	We first note that for all Borel sets $ A $ and all $ \tilde{h} $, we have
	\begin{equation*}
		\nu ( A ) 
		=
		\sum_{ i = 1 }^{ m }
		\max_{ h }	
		c_{ i }^{ h }
		\mu ( A \cap B_{ i } ) 
		\geq
		\sum_{ i = 1 }^{ m }
		c_{ i }^{ \tilde{ h } }
		\mu ( A \cap B_{ i } )
		=
		\mu_{ \tilde{ h } } ( A ).
	\end{equation*}
	On the other hand we have that for all $ 1 \leq i \leq m $, we find an 
	index $ h( i ) $ such that
	$ \max_{ h } c_{ i }^{ h } = c_{ i }^{ h( i ) } $.
	Since the sets $ B_{ i } $ are disjoint, we therefore have 
	\begin{equation*}
		\nu ( B_{ i } ) = \max_{ h } c_{ i }^{ h } \mu ( B_{ i } )
		= \mu_{ h ( i ) } ( B_{ i } ),
	\end{equation*}
	thus we have for all $ 1 \leq i \leq m $ that
	\begin{equation*}
		\nu ( B_{ i } ) 
		\leq
		\left(
		\bigvee_{ h = 1 }^{ k }
		\mu_{ h }
		\right) ( B_{ i } ).
	\end{equation*}
	Since the support of $ \nu $ is contained in $ \bigcup_{ 1 \leq i \leq m } B_{ i } $, this finishes the proof.
\end{proof}