\chapter{Introduction}

Allgemeine Notizen: Energiekonvergenz kann bei positiver Krümmung gezeigt 
werden, siehe introduciton sebastian tim. 
Etwas zu Eindeutigkeit sagen; wir gehen sehr oft zu Teilfolgen über. Hilft uns 
hierbei die Weak strong uniqueness?

Multiphase mean curvature is an important geometric evolution equation which 
has been studied for a long time, bearing not only mathematical importance, but 
also for the applied sciences. Originally it was proposed to study the 
evolution of grain boundaries in annealed recrystallized metal, as described by
Mullins in \cite{mullins_two_dimensional_motion_of_idealized_grain_boundaries}, 
who cites Beck in \cite{beck_metal_interfaces} as already having observed such 
a behaviour in 1952. 

Over the years a number of different solution concepts for multiphase mean 
curvature flow have been proposed. Classically we have smooth solutions, where 
we require the evolution of sets to be smooth, for example described by Huisken 
in \cite{huisken_asymptotic_behavior_for_singuliarities_of_mcf}. 
Another description of smoothly evolving mean curvature flow can be found in 
the work of Gage and Hamilton 
\cite{gage_hamilton_the_heat_equation_shrinking_convex_plane_curves}, who 
proved the \enquote{shrinking conjecture} for convex planar curves.
Brakke describes in his book 
\cite{brakke_kenneth_motion_of_surface_by_mean_curvature} the motion by mean 
curvature using varifolds, which yields a quite abstract and general notion for 
mean curvature flow and is based on the gradient flow structure of mean 
curvature flow. 
Luckhaus and Sturzenhecker introduced a 
distributional solution to mean curvature flow in their work 
\cite{luckhaus_sturzenhecker_implicit_time_discretization_for_mcf}. Another 
solution concept is the viscosity solution concept, for example presented in 
(\cite{chen_giga_goto_uniqueness_and_existence_of_generalized_mcf_equations},
\cite{evans_spruck_motion_of_level_sets_by_mean_curvature}), where it is shown 
that solutions of a certain parabolic equation have the property that if they 
are smooth, the corresponding level sets move by mean curvature.

The Allen--Cahn equation
\begin{equation}
	\label{ac_intro}
	\partial_{ t } u_{ \varepsilon }
	=
	\Delta u_{ \varepsilon }
	-
	\frac{ 1 }{ \varepsilon^{ 2 } }
	\nabla W ( u_{ \varepsilon } )
\end{equation}
is commonly used as a phase-field approximation for mean curvature flow and was 
first discovered by Allen and Cahn in their paper 
\cite{allen_cahn_microscopig_theory_for_antiphase_boundary_motion}. Here $ W 
\colon \mathbb{ R }^{ N } \to [ 0 , \infty ) $ is some smooth potential with 
finitely many zeros $ \alpha_{ 1 } , \dotsc, \alpha_{ P } $.
Since then it has been a topic of long study, both from a numerical and 
analytical standpoint. 

If we consider the two-phase case, which corresponds to setting $ N = 1 $ and $ 
P = 2 $, then 
the behaviour of the solutions to (\ref{ac_intro}) are thoroughly researched.
Chen has proven in 
\cite{chen_generation_and_propagation_of_interfaces_for_reaction_diffusion_equations}
that as long as the interface of the limit evolves smoothly, we have 
convergence to mean curvature flow.
The authors Bronsard and Kohn showed by utilization of the gradient flow 
structure of (\ref{ac_intro}), more precisely by mainly utilizing the 
corresponding energy dissipation inequality, in 
\cite{bronsard_kohn_motion_by_mean_curvature_as_singular_limit} the compactness 
of solutions and regularity for the limit. Moreover they studied the behaviour 
for radially symmetric initial data and showed motion by mean curvature in this 
case. 
Ilmanen made fundamental contributions in 
\cite{ilmanen_convergence_of_ac_to_brakkes_mcf} by proving the convergence to 
Brakke's mean curvature flow as long as the initial conditions are 
well-prepared by exploiting the equipartition of energies. However his methods 
seem to only work in the two-phase case since he uses a comparison principle 
whose generalization to the vectorial case is not clear.

The multiphase case for both the behaviour of the Allen--Cahn equation and mean 
curvature is much more involved and a topic of current research. For the 
convergence of the Allen--Cahn equation, it has by shown by Bronsard and 
Reitich in 
\cite{bronsard_reitich_on_three_phase_boundary_motion_and_singular_limit} that 
for the three-phase case, we have short-time existence of a solution (in what 
sense????). 

The analysis of multiphase mean curvature flow has been studied for example by 
Mantegazza, Novaga and Tortorelli in 
\cite{mantegazza_novaga_tortorelli_motion_by_curvature_of_planar_networks}, 
where they considered the planar case when only a single triple junction 
appears, and their work has been extended to several triple junctions in 
\cite{mantegazza_novaga_pluda_schule_evolution_of_networks_with_multiple_junctions}.

Asymptotic expansions  with Neumann boundary conditions 
have been studied by Keller, Rubinstein and Sternberg in 
\cite{keller_rubinstein_sternberg_fast_reaction_slow_diffusion}.
