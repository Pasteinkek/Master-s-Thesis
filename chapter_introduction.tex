\chapter{Introduction}

\section{Abstract}

This thesis presents a conditional convergence result of solutions to the 
Allen--Cahn equation with arbitrary potentials to a De Giorgi type $ \bv 
$-solution of multiphase mean 
curvature flow. For this we will recall the proof for the conditional 
convergence to $ \bv $-solutions in 
the sense of Laux and Simon. Lastly we show that De Giorgi type $ \bv 
$-solutions are De Giorgi type varifold solutions, and thus our solution 
is unique in a weak-strong sense.

\section{History and main results}

Multiphase mean curvature flow is an important geometric evolution equation 
which 
has been studied for a long time, bearing not only mathematical importance, but 
also for the applied sciences. Originally it was proposed to study the 
evolution of grain boundaries in annealed recrystallized metal, as described by
Mullins in \cite{mullins_two_dimensional_motion_of_idealized_grain_boundaries}, 
who cites Beck in \cite{beck_metal_interfaces} as already having observed such 
a behaviour in 1952. 

Over the years numerous different solution concepts for multiphase mean 
curvature flow have been proposed. Classically we have smooth solutions, where 
we require the evolution of the interfaces to be smooth, for example described 
by Huisken 
in \cite{huisken_asymptotic_behavior_for_singuliarities_of_mcf}. 
Another description of smoothly evolving mean curvature flow can be found in 
the work of Gage and Hamilton 
\cite{gage_hamilton_the_heat_equation_shrinking_convex_plane_curves}, who 
proved the \enquote{shrinking conjecture} for convex planar curves.
Brakke describes in his book 
\cite{brakke_kenneth_motion_of_surface_by_mean_curvature} the motion by mean 
curvature using varifolds, which yields a quite abstract and general notion for 
mean curvature flow and is based on the gradient flow structure of mean 
curvature flow. 
Luckhaus and Sturzenhecker introduced a 
distributional solution concept for mean curvature flow in their work 
\cite{luckhaus_sturzenhecker_implicit_time_discretization_for_mcf}. Another 
approach is the viscosity solution concept presented in 
\cite{chen_giga_goto_uniqueness_and_existence_of_generalized_mcf_equations} and
\cite{evans_spruck_motion_of_level_sets_by_mean_curvature}, where it is shown 
that solutions of a certain parabolic equation have the property that if they 
are smooth, the corresponding level sets move by mean curvature.

For some smooth potential $ W 
\colon \mathbb{ R }^{ N } \to [ 0 , \infty ) $ with 
finitely many zeros $ \alpha_{ 1 } , \dotsc, \alpha_{ P } $, the Allen--Cahn 
equation
\begin{equation}
	\label{ac_intro}
	\partial_{ t } u_{ \varepsilon }
	=
	\Delta u_{ \varepsilon }
	-
	\frac{ 1 }{ \varepsilon^{ 2 } }
	\nabla W ( u_{ \varepsilon } )
\end{equation}
is commonly used as a phase-field approximation for mean curvature flow and was 
first proposed by Allen and Cahn in their paper 
\cite{allen_cahn_microscopig_theory_for_antiphase_boundary_motion}.
Since then it has been an extensively researched topic, both from a numerical 
and 
analytical standpoint. 

If we consider the two-phase case, which corresponds to setting $ N = 1 $ and $ 
P = 2 $, then 
the behaviour of the solutions to (\ref{ac_intro}) as $ \varepsilon $ tends to 
zero are thoroughly researched.
The authors Bronsard and Kohn showed in 
\cite{bronsard_kohn_motion_by_mean_curvature_as_singular_limit} the compactness 
of solutions and regularity for the limit by exploiting the gradient flow 
structure of (\ref{ac_intro}), more precisely by mainly utilizing the 
corresponding energy dissipation inequality. Moreover they studied the 
behaviour 
for radially symmetric initial data and showed that the limit moves by mean 
curvature in this 
case. 
Chen has proven in 
\cite{chen_generation_and_propagation_of_interfaces_for_reaction_diffusion_equations}
that as long as the interface of the limit evolves smoothly, we have 
convergence to (classical) mean curvature flow. 
A similar result has been shown 
by De Mottoni and Schatzmann in 
\cite{de_mottoni_schatzmann_geometrical_evolution_of_developed_interfaces}, 
where they showed short time convergence. Their strategy was to prove through a 
spectral estimate that the solution $ u_{ 
\varepsilon } $ already coincides with the formal asymptotic expansion up to an 
error. Similar strategies, but for nonlinear Robin 
boundary conditions with angle close to 90 degrees, can be found in the paper
\cite{abels_moser_convergence_of_ac_with_nonlinear_robin_boundary_condition_to_mcf}
by Abels and Moser.

Ilmanen made fundamental contributions in 
\cite{ilmanen_convergence_of_ac_to_brakkes_mcf} by proving the convergence to 
Brakke's mean curvature flow as long as the initial conditions are 
well-prepared by exploiting the equipartition of energies. However his methods 
seem to only work in the two-phase case since he uses a comparison principle 
whose generalization to the vectorial case is not clear.

The multiphase case for both the convergence of the Allen--Cahn equation and 
mean 
curvature flow is much more involved and a topic of current research. For the 
convergence of the Allen--Cahn equation, asymptotic expansions  with Neumann 
boundary conditions 
have been studied by Keller, Rubinstein and Sternberg in 
\cite{keller_rubinstein_sternberg_fast_reaction_slow_diffusion}. Even though 
they considered the vector-valued Allen--Cahn equation with a multiwell 
potential, their analysis restricts to the parts of the interfaces where no 
triple junctions appear. Bronsard and 
Reitich however considered in 
\cite{bronsard_reitich_on_three_phase_boundary_motion_and_singular_limit} a 
formal asymptotic expansion which also considers triple junctions
for the three-phase case. Moreover they proved short-time existence for the 
three-phase boundary problem.

The analysis of multiphase mean curvature flow has been studied for example by 
Mantegazza, Novaga and Tortorelli in 
\cite{mantegazza_novaga_tortorelli_motion_by_curvature_of_planar_networks}, 
where they considered the planar case when only a single triple junction 
appears, and their work has been extended to several triple junctions in 
\cite{mantegazza_novaga_pluda_schule_evolution_of_networks_with_multiple_junctions}.
Ilmanen, Neves and Schulze furthermore proved in 
\cite{ilmanen_neves_schulze_on_short_time_existence_for_the_planar_network_flow}
that even for non-regular initial data in the sense that Herring's angle 
condition is not satisfied, we have short time existence by approximation 
through regular networks. For long time existence, it has been shown by Kim and 
Tonegawa in \cite{kim_tonegawa_on_the_mean_curvature_flow_of_grain_boundaries} 
through a modification of Brakke's approximation scheme that one can obtain 
non-trivial mean curvature flow even with singular initial data.

The first main goal of this thesis is to prove a conditional convergence result 
of solutions to the vectorial Allen--Cahn equation (\ref{ac_intro}) to a De 
Giorgi type $ \bv $-solution of multiphase mean curvature flow in the sense of 
\Cref{de_giorgi_solution_to_mmcf}. The proof is based mainly on a duality 
argument and the results of Laux and Simon in 
\cite{convergence_of_allen_cahn_equation_to_multiphase_mean_curvature_flow}.
However the strong assumption we make here is that the Cahn-Hilliard energies 
(\ref{cahn_hilliard_energy}) of the solutions to (\ref{ac_intro}) converge to 
the perimeter functional (\ref{definition_of_multiphase_energy}) applied to the 
limit. 
This prevents that as $ \varepsilon $ tends to zero, the approximate interfaces 
collapse. This would mean that energy dissipates, see also the discussion in
\Cref{subsection_de_giorgi_type_varifold_solutions_for_mcf}.  In general we are 
inclined to believe that this assumption could fail. For example it has been 
shown by Bronsard and Stoth in 
\cite{bronsard_stoth_on_the_existence_of_high_multiplicity_interfaces}
that for the volume preserving Allen--Cahn equation and for radial-symmetric 
initial data, we can have any number of higher 
multiplicity transition layers which are at most $ C \varepsilon^{ \alpha } $ 
apart, at least for times of order one.
Here $ \alpha $ is some exponent between zero and one third.
Nonetheless the energy 
convergence assumption provides us with the important equipartition of 
energies, whose proof under milder assumption was the main obstacle of Ilmanen 
in \cite{ilmanen_convergence_of_ac_to_brakkes_mcf}. Moreover it is the key for 
our localization estimates and lets us localize on the different phases of the 
limit. Lastly it assures that the differential $ \nabla u_{ 
\varepsilon } $ can locally up to an error be written as a rank-one matrix. In 
fact it
is the tensor of the approximate frozen unit normal and the gradient of the 
geodesic 
distance function associated to the majority phase evaluated at $ u_{ 
\varepsilon } $, see the proof of \Cref{convergence_of_curvature_multiphase}.

The second main goal is to compare De Giorgi type $ \bv $-solutions to De 
Giorgi type varifold solutions.
The latter were 
proposed by Hensel and Laux in 
\cite{hensel_laux_varifold_solution_concept_for_mean_curvature_flow}. We will 
show that the De Giorgi type $ \bv $-solution concept is stronger in the sense 
that every $ \bv 
$-solution is also a varifold solution. 
Since Hensel and Laux have shown 
weak-strong uniqueness for their varifold solution concept, it follows that we 
also obtain weak-strong uniqueness for our De Giorgi type $ \bv $-solution 
concept, which is the best we can expect. 
In fact it has been shown on a 
numerical basis by Angenent, Chopp and Ilmanen in 
\cite{angenent_chopp_ilmanen_a_computed_example_of_nonuniqueness_of_mcf}
that even in three dimensions, there exists a smooth hypersurface whose 
evolution by 
mean curvature 
flow admits a singularity at a certain time after which we have nonuniqueness. 
A rigorous analysis of this phenomenon has been done in dimensions 4, 5, 6, 7 
and 8 by Angenent, Ilmanen and Velázquez in 
\cite{angenent_ilmanen_velázquez_fattening_from_smooth_initial_data_in_mcf}.

Let us also mention some of the closely related unanswered questions. For one 
Hensel and Laux have shown in 
\cite{hensel_laux_varifold_solution_concept_for_mean_curvature_flow}
that in the two-phase case and under well prepared initial conditions, the
solutions of the Allen--Cahn equation (\ref{ac_intro}) converge to a De Giorgi 
type varifold solution. However their methods have no obvious generalization to 
the multiphase case without the energy convergence assumption, and one even 
struggles to find an approximate sequence 
which constructs the desired varifolds. And even then one would have to find 
suitable substitutions for the localization estimates explained in 
\Cref{section_localization_estimates}. 
These are based on De Giorgi's structure 
theorem and thus only work for $ \bv $-functions.

Another possible question would be how to generalize the results to the case 
of 
arbitrary mobilities: Throughout the thesis, and also the main background 
papers 
(\cite{convergence_of_allen_cahn_equation_to_multiphase_mean_curvature_flow},
\cite{hensel_laux_varifold_solution_concept_for_mean_curvature_flow}), it is 
always assumed that the mobilities are fixed through the relation $ \mu_{ i j } 
= 1/ \sigma_{ i j } $.
Here $ \sigma_{ i j } $ denotes the surface tension of 
the $ ( i , j ) $-th interface and $ \mu_{ i j } $ its mobility. As proposed by
Bretin, Danescu, Penuelas and Masnou in 
\cite{bretin_dansecu_penuelas_masnou_a_metric_based_approach_to_mmcf_with_mobilities},
passing
to arbitrary mobilities should 
amount to multiplying an appropriate \enquote{mobility matrix} $ M $ onto the 
right-hand side of (\ref{ac_intro}) and changing the metric of the underlying 
space accordingly to $ \langle u , v \rangle = \int \inner*{M u}{ v} \dd{ x } $.
The difference in their approach is to first uncouple their system so that 
they arrive at the scalar Allen--Cahn equation and then couple the components 
through a Lagrange-multiplier, which assures that the limit is a partition.

\section{Structure of the thesis}

In \Cref{chapter_gradient_flows_and_mcf} we will give a soft mathematical 
introduction into the topic of gradient flows. We will derive De Giorgi's 
optimal energy dissipation inequality in a simple example and discuss its 
usefulness for 
reformulating the gradient flow equation. Moreover we will apply our 
observations 
to (multiphase) mean curvature flow.

Afterwards in \Cref{chapter_ac_equation}, we will consider the Allen--Cahn 
equation 
(\ref{ac_intro}) already mentioned above. Here we will focus again on its 
gradient flow structure. We then propose a suitable solution concept and prove 
the existence of a solution through De Giorgi's minimizing movements scheme.

Continuing with \Cref{chapter_convergence_to_evolving_partition}, we are going 
to take a look at the behaviour of solutions to the Allen--Cahn equation as $ 
\varepsilon $ tends to zero. First we will study the simple two-phase case in 
order 
to get a better feeling for the equation and highlight important techniques 
like the Modica--Mortola trick. We will show that we have convergence to an 
evolving set of finite perimeter in space and time. 
Afterwards we will also prove 
precompactness of the sequence in $ \cont ( [ 0 , T ] ; \lp^{ 2 } ) $ which 
lets us show that the initial data is attained. Next up is the multiphase case, 
which will require more finesse, but we will be able to show similar results as 
in the 
two-phase case. 
For this we will need a generalized chain rule for distributional 
derivatives and a careful analysis of the geodesic distance functions with 
respect to the potential $ W $ from equation (\ref{ac_intro}).

\Cref{chapter_conditional_convergence_of_ac} is concerned with the conditional 
convergence result proven by Laux and Simon in 
\cite{convergence_of_allen_cahn_equation_to_multiphase_mean_curvature_flow}. 
We will show that the in \Cref{chapter_convergence_to_evolving_partition} 
observed limit is a $ \bv $-solution to mean curvature flow under the crucial 
assumption of energy convergence. 
Again we will first consider the simpler two-phase case in 
order to simplify some of the arguments. Here one has to show the existence of 
normal velocities followed by the equipartition of the Cahn--Hilliard energies. 
Then we are going 
to separately show the convergence of the velocity term and the curvature term 
of the Allen--Cahn equation to the corresponding terms for mean curvature flow. 
For the multiphase equivalent our core strategy will be to reduce the 
multiphase case to the two-phase 
case through a localization argument, which we detail in 
\Cref{section_localization_estimates}. 
Afterwards we will show the same results as in the two-phase case.

Lastly in \Cref{chapter_de_giorgis_mcf} we are going to present new results 
building on the previous insights. We start off by presenting a De Giorgi type 
$ \bv $-solution concept for multiphase mean curvature flow. This is followed 
by proving a similar conditional convergence result as in the previous chapter. 
In the final 
\Cref{subsection_de_giorgi_type_varifold_solutions_for_mcf}, we are going to 
discuss the assumption of energy convergence. Moreover we show that every De 
Giorgi type $ \bv $-solution is a De Giorgi type varifold solution in the sense 
of Hensel and Laux in 
\cite{hensel_laux_varifold_solution_concept_for_mean_curvature_flow}, whose 
solution concept does not rely on the assumption of energy convergence.



