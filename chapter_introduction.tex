\chapter{Introduction}

Allgemeine Notizen: Energiekonvergenz kann bei positiver Krümmung gezeigt 
werden, siehe introduciton sebastian tim.

Multiphase mean curvature is an important geometric evolution equation which 
has been studied for a long time, bearing not only mathematical importance, but 
also for the applied sciences. Originally it was proposed to study the 
evolution of grain boundaries in annealed recrystallized metal, as described by
Mullins in \cite{mullins_two_dimensional_motion_of_idealized_grain_boundaries}, 
who cites Beck in \cite{beck_metal_interfaces} as already having observed such 
a behaviour in 1952. 

Over the years a number of different solution concepts for multiphase mean 
curvature flow have been proposed. Classically we have smooth solutions, where 
we require the evolution of sets to be smooth, for example described by Huisken 
in \cite{huisken_asymptotic_behavior_for_singuliarities_of_mcf}. 
Another description of smoothly evolving mean curvature flow can be found in 
the work of Gage and Hamilton 
\cite{gage_hamilton_the_heat_equation_shrinking_convex_plane_curves}, who 
proved the \enquote{shrinking conjecture} for convex planar curves.
Brakke describes in his book 
\cite{brakke_kenneth_motion_of_surface_by_mean_curvature} the motion by mean 
curvature using varifolds, which yields a quite abstract and general notion for 
mean curvature flow. 
Luckhaus and Sturzenhecker introduced a 
distributional solution to mean curvature flow in their work 
\cite{luckhaus_sturzenhecker_implicit_time_discretization_for_mcf}. Another 
solution concept is the viscosity solution concept