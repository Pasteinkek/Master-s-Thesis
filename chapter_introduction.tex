\chapter{Introduction}

Allgemeine Notizen: 
Etwas zu Eindeutigkeit sagen; wir gehen sehr oft zu Teilfolgen über. Hilft uns 
hierbei die Weak strong uniqueness?
De Giorgis originial paper finden?

\section{Abstract}

This thesis presents a conditional convergence result of the solutions to the 
Allen--Cahn equation to a De Giorgi type $ \bv $-solution to multiphase mean 
curvature flow for arbitrary potentials, but fixed mobilities. For this we will 
also recall the proof for the conditional convergence to $ \bv $-solutions in 
the sense of Laux and Simon. Lastly we show that De Giorgi type $ \bv 
$-solutions are De 
Giorgi type varifold solutions, and thus our solution concept exhibits a 
weak-strong uniqueness.

\section{History and main goals}

Multiphase mean curvature is an important geometric evolution equation which 
has been studied for a long time, bearing not only mathematical importance, but 
also for the applied sciences. Originally it was proposed to study the 
evolution of grain boundaries in annealed recrystallized metal, as described by
Mullins in \cite{mullins_two_dimensional_motion_of_idealized_grain_boundaries}, 
who cites Beck in \cite{beck_metal_interfaces} as already having observed such 
a behaviour in 1952. 

Over the years a number of different solution concepts for multiphase mean 
curvature flow have been proposed. Classically we have smooth solutions, where 
we require the evolution of sets to be smooth, for example described by Huisken 
in \cite{huisken_asymptotic_behavior_for_singuliarities_of_mcf}. 
Another description of smoothly evolving mean curvature flow can be found in 
the work of Gage and Hamilton 
\cite{gage_hamilton_the_heat_equation_shrinking_convex_plane_curves}, who 
proved the \enquote{shrinking conjecture} for convex planar curves.
Brakke describes in his book 
\cite{brakke_kenneth_motion_of_surface_by_mean_curvature} the motion by mean 
curvature using varifolds, which yields a quite abstract and general notion for 
mean curvature flow and is based on the gradient flow structure of mean 
curvature flow. 
Luckhaus and Sturzenhecker introduced a 
distributional solution to mean curvature flow in their work 
\cite{luckhaus_sturzenhecker_implicit_time_discretization_for_mcf}. Another 
solution concept is the viscosity solution concept, for example presented in 
(\cite{chen_giga_goto_uniqueness_and_existence_of_generalized_mcf_equations},
\cite{evans_spruck_motion_of_level_sets_by_mean_curvature}), where it is shown 
that solutions of a certain parabolic equation have the property that if they 
are smooth, the corresponding level sets move by mean curvature.

The Allen--Cahn equation
\begin{equation}
	\label{ac_intro}
	\partial_{ t } u_{ \varepsilon }
	=
	\Delta u_{ \varepsilon }
	-
	\frac{ 1 }{ \varepsilon^{ 2 } }
	\nabla W ( u_{ \varepsilon } )
\end{equation}
is commonly used as a phase-field approximation for mean curvature flow and was 
first discovered by Allen and Cahn in their paper 
\cite{allen_cahn_microscopig_theory_for_antiphase_boundary_motion}. Here $ W 
\colon \mathbb{ R }^{ N } \to [ 0 , \infty ) $ is some smooth potential with 
finitely many zeros $ \alpha_{ 1 } , \dotsc, \alpha_{ P } $.
Since then it has been a topic of long study, both from a numerical and 
analytical standpoint. 

If we consider the two-phase case, which corresponds to setting $ N = 1 $ and $ 
P = 2 $, then 
the behaviour of the solutions to (\ref{ac_intro}) are thoroughly researched.
Chen has proven in 
\cite{chen_generation_and_propagation_of_interfaces_for_reaction_diffusion_equations}
that as long as the interface of the limit evolves smoothly, we have 
convergence to mean curvature flow.
The authors Bronsard and Kohn showed by utilization of the gradient flow 
structure of (\ref{ac_intro}), more precisely by mainly utilizing the 
corresponding energy dissipation inequality, in 
\cite{bronsard_kohn_motion_by_mean_curvature_as_singular_limit} the compactness 
of solutions and regularity for the limit. Moreover they studied the behaviour 
for radially symmetric initial data and showed motion by mean curvature in this 
case. 
Ilmanen made fundamental contributions in 
\cite{ilmanen_convergence_of_ac_to_brakkes_mcf} by proving the convergence to 
Brakke's mean curvature flow as long as the initial conditions are 
well-prepared by exploiting the equipartition of energies. However his methods 
seem to only work in the two-phase case since he uses a comparison principle 
whose generalization to the vectorial case is not clear.

The multiphase case for both the behaviour of the Allen--Cahn equation and mean 
curvature is much more involved and a topic of current research. For the 
convergence of the Allen--Cahn equation, asymptotic expansions  with Neumann 
boundary conditions 
have been studied by Keller, Rubinstein and Sternberg in 
\cite{keller_rubinstein_sternberg_fast_reaction_slow_diffusion} (NOCHMAL MIT 
TIM BESPREHCEN). Moreover it has by shown by Bronsard and 
Reitich in 
\cite{bronsard_reitich_on_three_phase_boundary_motion_and_singular_limit} that 
for the three-phase case, we have short-time existence of a solution (in what 
sense????). 

The analysis of multiphase mean curvature flow has been studied for example by 
Mantegazza, Novaga and Tortorelli in 
\cite{mantegazza_novaga_tortorelli_motion_by_curvature_of_planar_networks}, 
where they considered the planar case when only a single triple junction 
appears, and their work has been extended to several triple junctions in 
\cite{mantegazza_novaga_pluda_schule_evolution_of_networks_with_multiple_junctions}.
Ilmanen, Neves and Schulze furthermore proved in 
\cite{ilmanen_neves_schulze_on_short_time_existence_for_the_planar_network_flow}
that even for non-regular initial data in the sense that not all angles at 
triple junctions are equal, we have short time existence by approximation 
through regular networks. For long time existence, it has been shown by Kim and 
Tonegawa in \cite{kim_tonegawa_on_the_mean_curvature_flow_of_grain_boundaries} 
through a modification of Brakke's approximation scheme that one can obtain 
non-trivial mean curvature flow even with singular initial data.

The first main goal of this thesis is to prove a conditional convergence result 
of solutions to the vectorial Allen--Cahn equation (\ref{ac_intro}) to a De 
Giorgi type $ \bv $-solution of multiphase mean curvature flow in the sense of 
\Cref{de_giorgi_solution_to_mmcf}. The proof is based mainly on a duality 
argument and the results of Laux and Simon in 
\cite{convergence_of_allen_cahn_equation_to_multiphase_mean_curvature_flow}.
However the strong assumption (MIT TIM ÜBER SIEN PAPER DISKUTIEREN; WANN MAN 
DIES ZEIGEN KANN) we make here is that the Cahn-Hilliard energies 
(\ref{cahn_hilliard_energy}) of the solutions to (\ref{ac_intro}) converge to 
the perimeter functional (\ref{definition_of_multiphase_energy}) of the limit. 
This prevents that as $ \varepsilon $ tends to zero, approximate interfaces 
collapse, which would mean that energy dissipates, see also the discussion in
\Cref{subsection_de_giorgi_type_varifold_solutions_for_mcf}. The energy 
convergence assumption provides us with the important equipartition of 
energies, whose proof under milder assumption was the main obstacle of Ilmanen 
in \cite{ilmanen_convergence_of_ac_to_brakkes_mcf}. Moreover it is the key for 
our localization estimates and lets us localize on the different phases of mean 
curvature flow. Lastly it assures that the differential $ \nabla u_{ 
\varepsilon } $ can locally up to an error be written as a rank-one matrix which
is the tensor of the approximate frozen unit normal and the gradient of the 
geodesic 
distance function associated to the majority phase evaluated at $ u_{ 
\varepsilon } $, see the proof of \Cref{convergence_of_curvature_multiphase}.

The second main goal is to compare De Giorgi type $ \bv $-solutions to De 
Giorgi type varifold solutions, which were 
proposed by Hensel and Laux in 
\cite{hensel_laux_varifold_solution_concept_for_mean_curvature_flow}. We will 
show that our solution concept is stronger in the sense that every $ \bv 
$-solution is also a varifold solution. Since Hensel and Laux have shown 
weak-strong uniqueness for their varifold solution concept, it follows that we 
also obtain weak-strong uniqueness for our De Giorgi type $ \bv $-solution 
concept, which is the best we can expect. In fact it has been shown on a 
numerical basis by Angenent, Chopp and Ilmanen in 
\cite{angenent_chopp_ilmanen_a_computed_example_of_nonuniqueness_of_mcf}
(AUCH MIT TIM BESPRECHEN)
that even in three dimensions, there exists a hypersurface whose evolution by 
mean curvature 
flow admits a singularity at a certain time after which we have nonuniqueness.

Let us also mention some of the closely related unanswered questions. For one 
Hensel and Laux have shown in 
\cite{hensel_laux_varifold_solution_concept_for_mean_curvature_flow}
that in the two-phase case and under well prepared initial conditions, then 
solutions of the Allen--Cahn equation (\ref{ac_intro}) converge to a De Giorgi 
type varifold solution. However their methods have no obvious generalization to 
the multiphase case without the energy convergence assumption, and one even 
struggles to find an approximate sequence 
which constructs the desired varifolds. And even then one would have to find 
suitable substitutions for the localization estimates explained in 
\Cref{section_localization_estimates}, which are based on De Giorgi's structure 
theorem and thus only work for $ \bv $-functions.
One other possible question would be how to generalize the results to the case 
of 
arbitrary mobilities: Throughout the thesis, and also the main background 
papers 
(\cite{convergence_of_allen_cahn_equation_to_multiphase_mean_curvature_flow},
\cite{hensel_laux_varifold_solution_concept_for_mean_curvature_flow}), it is 
always assumed that the mobilities are fixed through the relation $ \mu_{ i j } 
= 1/ \sigma_{ i j } $, where $ \sigma_{ i j } $ denotes the surface tension of 
the $ ( i , j ) $-th interface, and $ \mu_{ i j } $ its mobility. As proposed by
Bretin, Danescu, Penuelas, and Masnou in 
\cite{bretin_dansecu_penuelas_masnou_a_metric_based_approach_to_mmcf_with_mobilities},
passing
to arbitrary mobilities should 
amount to multiplying an appropriate \enquote{mobility matrix} $ M $ onto the 
right-hand side of (\ref{ac_intro}) and changing the metric of the underlying 
space accordingly to $ \langle u , v \rangle = \int \inner*{M u}{ v} \dd{ x } $.
The difference in their approach is that first uncouple their system so that 
they arrive at the scalar Allen--Cahn equation and then couple they components 
through a Lagrange-multiplier, which assures that the limit is a partition.

\section{Structure of the thesis}

