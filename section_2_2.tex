\section{Convergence in the multiphase case}

Let us now turn to the much more interesting and more challenging case where we consider systems of the Allen--Cahn equation (\ref{allen_cahn_eq}), or in other words, we want to consider the case where $ u_{\varepsilon } $ maps to $ \mathbb{ R }^{ N } $ and therefore our potential $ W $ is a map from $ \mathbb{ R }^{ N } $ to $ [ 0 , \infty ) $. The for us most relevant case is when $ W $ has exactly $ P = N + 1 $ zeros given by $ \alpha_{ 1 } , \dotsc, \alpha_{ P } $, but it is no limitation for us to allow more general amount of zeros.

One of the many difficulties in the vectorial case is that that there is no easy choice of a primitive for $ \sqrt{ 2 W ( u ) } $ compared to the scalar case. We there saw for example through the Modica-Mortula trick (\ref{modica_mortula_trick}) that this provided a very powerful tool for us, and comparing the composition $ \phi \circ u_{ \varepsilon } $ to $ u_{ \varepsilon } $ was quite simple since the map $ \phi $ was invertible as a consequence of the non-negativity of $ W $.

As a suitable replacement, we shall consider the \emph{geodesic distance} defined as 
\begin{equation*}
		\geodesic_distance ( u, v )
		\coloneqq
		\inf
		\left\{
		\int_{ 0 }^{ 1 }
		\sqrt{ 2 W ( \gamma ) }
		\abs{ \dot{ \gamma }  }
		\dd{t}
		\,
		\colon
		\, \gamma \in \mathrm{ C }^{ 1 } \left( [0, 1 ] ; \mathbb{ R }^{ N } \right) \text{ with } \gamma( 0 ) = u,\, \gamma( 1 )= v 
		\right\}.
\end{equation*}
This indeed defines a metric on $ \mathbb{ R }^{ N } $: If $ \geodesic_distance ( v, w ) = 0 $, then by the continuity of $ W $ and since it only has a discrete set of zeros, we may deduce that $ v = w $. Symmetry can be seen by reversing a given path between two points and the triangle inequality follows from concatenation (and smoothing) of two paths and rescaling.

The \emph{geodesic distances} generated by $ W $  are defined as
\begin{equation*}
	\sigma_{ i , j } 
	\coloneqq
	\geodesic_distance ( \alpha_{ i } , \alpha_{ j } )
\end{equation*}
and as a consequence of $ \geodesic_distance $ being a metric satisfy
\begin{equation*}
	\sigma_{ i , k } \leq \sigma_{ i , j } + \sigma_{ j , k },
\end{equation*}
$ \sigma_{ i , j } = 0 $ if and only if $ i $ is equal to $ j $ and $ \sigma_{ i , j } = \sigma_{ j, i } $.
