\chapter{The Allen--Cahn equation}

This chapter follows \cite{convergence_of_allen_cahn_equation_to_multiphase_mean_curvature_flow}, but since the authors decided to only sketch some proofs. we want to go into more detail.

Let $ \Lambda > 0 $ and define the flat torus 
$ \mathbb{T} = [0, \Lambda )^{ d } \subset \mathbb{ R }^{ d } $, 
where we work with periodic boundary conditions and write $ \int \dd{x} $ instead of $ \int_{ \mathbb{ T } } \dd{x} $.
Then for 
$ u \colon [ 0 , \infty ) \times \mathbb{ T } \to \mathbb{ R }^{ N } $ 
and some potential 
$ W \colon \mathbb{ R }^{ N } \to [0, \infty ) $,
the \emph{Allen--Cahn equation} with parameter $ \varepsilon > 0 $ is given by
\begin{equation}
	\label{allen_cahn_eq}
	\partial_{ t } u 
	=
	\Delta u - \frac{1 }{ \varepsilon^{ 2 } } \nabla W ( u ).
\end{equation}

To understand this equation better, we consider the \emph{Cahn--Hilliard energy} which assigns to $ u $ for a fixed time the real number
\begin{equation}
	\label{cahn_hilliard_energy}
	\energy_{ \varepsilon } 
		(u)
	\coloneqq
	\int
		\frac{ 1 }{ \varepsilon }
		W ( u )
		+
		\frac{ \varepsilon }{ 2 }
		\abs{ \nabla u }^{ 2 }
	\dd{x}.
\end{equation}

If everything is nice and smooth, we can compute that under the assumption that $ u $ satisfies 
equation (\ref{allen_cahn_eq}), we have that
\begin{align*}
	\dv{t} \energy_{ \varepsilon } ( u )
	&=
	\int
		\frac{ 1 }{ \varepsilon }
		\inner{\nabla W ( u )}{ \partial_{ t } u}
		+
		\varepsilon
		\inner{\nabla u}{ \nabla \partial_{ t } u}
	\dd{x}
	\\
	&=
	\int
		\inner*{ \frac{ 1 }{\varepsilon } \nabla W ( u ) - \varepsilon \Delta u  }{ \partial_{ t } u }
	\dd{x}
	\\
	&=
	\int - \varepsilon \abs{ \partial_{ t } u }^{ 2 } \dd{ x }
	\tag{\ref{allen_cahn_eq}}.
\end{align*}

This calculation suggests that equation (\ref{allen_cahn_eq}) is the $ \lp^{ 2 } $ gradient-flow (rescaled by $ \sqrt{\varepsilon} $) of the Cahn--Hilliard energy. Thus we can try to construct a solution to the PDE (\ref{allen_cahn_eq}) via De Giorgis minimizing movements scheme, which we will do in theorem \Cref{existence_of_ac_solution}.

But first we need to clarify what our potential $ W $ should look like. Classic examples in the scalar case are given by $ W ( u ) = \left( u^{ 2 } - 1 \right)^{ 2 } $ or $ W( u ) = u^{ 2 } ( u - 1 )^{ 2 } $, and we call functions like these \emph{doublewell potentials}, see also \ref{graph_of_doublewell_potential}.

\begin{figure}[h]
\label{graph_of_doublewell_potential}
\begin{tikzpicture}
	\label{Plot of the doublewell potential $ W(u) = (u^2 - 1 )^{ 2 } $}
	\begin{axis}[
		axis lines = left,
		xlabel = \(u\),
		ylabel = {\(W(u)= (u^2 - 1)^{ 2 }\)},
		]
		%Below the red parabola is defined
		\addplot [
		domain=-2:2, 
		samples=100, 
		color=black,
		]
		{x^4-2*x^2 + 1};
	\end{axis}
\end{tikzpicture}
\caption{The graph of a doublewell potential}
\end{figure}

In higher dimensions, we want to accept the following potentials: $ W \colon \mathbb{ R }^{ N } \to [0, \infty ) $ has to be a a smooth multiwell potential with finitely many zeros at $ u = \alpha_{ 1 }, \dotsc , \alpha_{ P } \in \mathbb{ R }^{ N } $. Furthermore we aks for polynomial growth in the sense that there exists some $ p \geq 2 $ such that
\begin{equation}
	\label{polynomial_growth}
	\abs{ u }^{ p } \lesssim W(u) \lesssim \abs{ u }^{ p }
\end{equation}
and
\begin{equation}
	\label{polynomial_growth_derivative}
	\abs{ \nabla W ( u ) } \lesssim \abs{ u }^{ p -1 }
\end{equation}
for all $ u $ sufficiently large. Lastly we want $ W $ to be convex up to a small perturbation in the sense that there exist smooth functions 
$ W_{ \mathrm{conv} }$, $ W_{ \mathrm{pert} } \colon \mathbb{ R }^{ N } \to [ 0 , \infty ) $ such that
\begin{equation}
	\label{decomposition_of_w}
	W = W_{ \mathrm{conv}} + W_{ \mathrm{pert}},
\end{equation}
$ W_{ \mathrm{conv} } $ is convex and
\begin{equation}
	\sup_{ x \in \mathbb{ R }^{ N } }
	\abs{ \nabla^{ 2 } W_{ \mathrm{pert} } } < \infty.
\end{equation}
These assumptions are in particular satisfied by our two examples for doublewell potentials and therefore seem to be plausible.




