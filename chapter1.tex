\chapter{The Allen--Cahn equation}

Let $ \Lambda > 0 $ and define the flat torus 
$ \mathbb{T} = [0, \Lambda )^{ d } \subset \mathbb{ R }^{ d } $, 
where we as usual assume periodic boundary conditions. 
Then for 
$ u \colon [ 0 , \infty ) \times \mathbb{ T } \to \mathbb{ R }^{ N } $ 
and some potential 
$ W \colon \mathbb{ R }^{ N } \to [0, \infty ) $,
the \emph{Allen--Cahn equation} with parameter $ \varepsilon > 0 $ is given by
\begin{equation}
	\label{allen_cahn_eq}
	\partial_{ t } u 
	=
	\Delta u - \frac{1 }{ \varepsilon^{ 2 } } \partial_{ u } W ( u ).
\end{equation}

To understand this equation better, we consider the \emph{Cahn--Hilliard energy} assigns to $ u $ for a fixed time the real number
\begin{equation}
	\label{cahn_hilliard_energy}
	\energy_{ \varepsilon } 
		(u)
	\coloneqq
	\int_{ \mathbb{ T } }
		\frac{ 1 }{ \varepsilon }
		W ( u )
		+
		\frac{ \varepsilon }{ 2 }
		\norm{ \nabla u }^{ 2 }
	\dd{x}.
\end{equation}

If everything is nice and smooth, we can compute that under the assumption that $ u $ satisfies 
equation \ref{allen_cahn_eq}, we have that
\begin{align*}
	\frac{ \dd }{ \dd{ t } } \energy_{ \varepsilon } ( u )
	&=
	\int
	stuff
\end{align*}


