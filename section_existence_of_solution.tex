\section{Existence of a solution}

With \Cref{solution_to_ac}, we are able to state our existence result for a
solution of the Allen--Cahn equation. The proof uses De Giorgis minimizing 
movements scheme and arguments from the theory of gradient flows and has been 
very 
briefly sketched in 
\cite{convergence_of_allen_cahn_equation_to_multiphase_mean_curvature_flow}. 
Note moreover that one would expect even more regularity for a solution 
of the Allen--Cahn equation. In fact, it has been shown by De Mottoni and 
Schatzmann in 
\cite{de_mottoni_schatzmann_geometrical_evolution_of_developed_interfaces} that 
in the scalar case, a solution u is 
smooth for positive times, at least for bounded initial data. 
Their Ansatz is a 
variation of parameters and the 
regularity then follows from the smoothing effect of the heat kernel. However 
De 
Giorgi's minimizing movements scheme has broad applications to a wide class of 
gradient flows, which is why we want to present this technique here.

\begin{theorem}
	\label{existence_of_ac_solution}
	Let $ u^{ 0 } \colon \flattorus \to \mathbb{ R }^{ N } $ be such that 
	$ \energy_{ \varepsilon } ( u^{ 0 } ) < \infty $.
	Then there exists a weak solution $ u_{ \varepsilon} $ to the Allen--Cahn 
	equation (\ref{allen_cahn_eq}) in the sense of \Cref{solution_to_ac} with 
	initial data $ u^{ 0 } $.
	Furthermore the solution satisfies the energy dissipation inequality
	\begin{equation}
		\label{energy_dissipation_sharp}
		\energy_{ \varepsilon } ( u_{ \varepsilon } ( t ) )
		+
		\int_{ 0 }^{ t }
		\int
		\varepsilon \abs{ \partial_{ t } u_{ \varepsilon } }^{ 2 }
		\dd{ x }
		\dd{ s }
		\leq
		\energy_{ \varepsilon } ( u^{ 0 } )
	\end{equation}
	for every $ t \in [ 0 , T ] $. We additionally have
	$
	\partial_{ i , j }^{ 2 } u_{ \varepsilon }, \nabla W ( u_{ \varepsilon } ) \in \lp^{ 2 } ( [0, T] \times \flattorus ; \mathbb{ R }^{ N } ) 
	$
	for all $ 1 \leq i, j \leq d $. In particular we can test the weak form (\ref{ac_weak_equation}) with $ \partial_{ i , j }^{ 2 } u_{ \varepsilon } $.
\end{theorem}

\begin{proof}
	The proof is divided into several steps. Therefore we want to give an 
	overview of our strategy. We will first consider a minimization problem for 
	functions in space. By a piecewise linear interpolation in time of 
	iteratively chosen minimizers of the minimization problem, we are going to 
	construct approximations to the solution. Through a convexity argument, we 
	get an energy dissipation inequality for these approximations. This enables 
	us to apply the Arzelà--Ascoli theorem and thus we obtain 
	a limit, which will be our solution. 
	We then show additional regularity by 
	using compactness arguments and a finite differences argument. Lastly we 
	show the optimal energy dissipation inequality 
	(\ref{energy_dissipation_sharp}) by using a lower semicontinuity argument 
	for the energy dissipation inequality of the approximations.
	\begin{description}[wide=0pt]
		\item[Step 1:] A minimization problem.
		
		Fix some $ h > 0 $, $ u_{ n - 1 } \in \wkp^{1 , 2 }\cap \lp^{ p } ( \flattorus; \mathbb{ R }^{ N } ) $ and consider the functional
		\begin{align}
			\label{de_giorgi_min_functional}
			\mathcal{ F } \colon \wkp^{ 1, 2 } \cap \lp^{ p } ( \flattorus ; \mathbb{ R }^{ N } )
			& \to
			\mathbb{ R }
			\\
			u & \mapsto 
			\energy_{ \varepsilon } ( u ) + 
			\frac{ 1 }{ 2h } \int \varepsilon \abs{ u - u_{ n - 1 } }^{ 2 } \dd{x }
			\notag .
		\end{align}
		Then $ \mathcal{ F } $ is coercive with respect to 
		$ \norm{ \cdot }_{ \wkp^{ 1, 2 } } $ 
		and bounded from below by zero. 
		Thus we may take a $ \wkp^{ 1 , 2 } $-bounded sequence  
		$ (v_{ k } )_{ k \in \mathbb{ N } } $ in $ \wkp^{ 1 , 2} \cap \lp^{ p } ( \flattorus ; \mathbb{ R }^{ N } ) $ such that 
		$ \mathcal{ F } ( v_{ k } ) \to \inf \mathcal{ F } $ as $ k \to \infty $. 
		These $ v_{ k } $ have a non-relabelled subsequence which converges weakly in $ \wkp^{ 1 , 2 } ( \flattorus ; \mathbb{ R }^{ N } ) $ and strongly in $ \lp^{ 2 } ( \flattorus, \mathbb{ R }^{ N } ) $ to some $ u \in \wkp^{ 1, 2 } ( \flattorus; \mathbb{ R }^{ N } ) $.
		Moreover $ v_{ k } $ is bounded in $ \lp^{ p } \left( \flattorus ; 
		\mathbb{ R }^{ N } \right)$ by the lower growth assumption 
		(\ref{polynomial_growth}) on $ W $.
		Therefore we obtain by the lower semicontinuity of the $ \lp^{ p } 
		$-norm  that $ u \in \wkp^{ 1 , 2 } 
		\cap \lp^{ p } ( \flattorus ; \mathbb{ R }^{ N } ) $.
		Lastly we have 
		\begin{align*}
			\frac{ 1 }{ 2 h }
			\int \varepsilon \abs{ u - u_{ n - 1 } }^{ 2 } \dd{ x}
			& =
			\liminf_{ k \to \infty }
			\frac{ 1 }{ 2h }
			\int \varepsilon \abs{ v_{ k } - u_{ n -1 } }^{ 2 } \dd{ x }
			\shortintertext{and}
			\int \frac{ \varepsilon }{ 2 } \abs{ \nabla u }^{ 2 } \dd{ x }
			& \leq
			\liminf_{ k \to \infty }
			\int \frac{ \varepsilon }{ 2 } \abs{ \nabla v_{ k } }^{ 2 } \dd{ x }
		\end{align*}
		by the weak lower semicontinuity of the norm.
		We pass to another non-relabelled subsequence which converges 
		pointwise almost everywhere. Then we achieve by the continuity of $ 
		W $ and Fatou's Lemma that
		\begin{equation*}
			\int \frac{ 1 }{ \varepsilon } W ( u ) \dd{ x }
			=
			\int \liminf_{ k \to \infty } \frac{ 1 }{ \varepsilon } W ( v_{k } ) \dd{ x }
			\leq
			\liminf_{ k \to \infty } \int \frac{ 1 }{ \varepsilon } W ( v_{ k } ) \dd{x},
		\end{equation*}
		which proves that u is a minimizer of $ \mathcal{ F } $.
		
		\item[Step 2:] Minimizing movements scheme.
		
		By iteratively choosing minimizers $ u_{ n }^{ h } $ from Step 1, we 
		obtain a sequence of functions $ u_{ 0 }^{ h } = u^{ 0 }, u_{ 1 }^{ h } 
		, \dotsc $. Thus we may define a function $ u^{ h } $ as the piecewise 
		linear interpolation at the time-steps $ 0, h , 2h, \dotsc $ of these 
		functions. More precisely we define for a natural number $ n \in 
		\mathbb{ N } $ 
		and $ t \in [ nh, (n+1)h ) $ our function to be  
		\begin{equation*}
		u^{ h } ( t , x ) 
		\coloneqq 
		\frac{ ( n + 1 ) h  - t }{ h }
		u_{ n }^{ h } ( x )
		+
		\frac{ t - n h }{ h }
		u_{ n +1 }^{ h } ( x ). 
		\end{equation*}
		
		\item[Step 3:] Sharp energy dissipation inequality for $ u_{ n }^{ h } 
		$.
		
		We claim that there exists some constant $ C > 0 $ such that for all $ h > 0 $ and $ n \in \mathbb{ N } $, we have
		\begin{equation}
			\label{discrete_optimal_energy_dissipation}
			\energy_{ \varepsilon } ( u_{ n }^{ h } )
			+
			\left( \frac{ 1 }{ h } - \frac{ C }{ 2 \varepsilon^{ 2 } } \right)
			\int \varepsilon \abs{ u_{ n }^{ h } - u_{ n - 1 }^{ h } }^{ 2 } \dd{ x }
			\leq
			\energy_{ \varepsilon } ( u_{ n - 1 }^{ h } ).
		\end{equation}
		In order to prove this inequality, we notice that since $ \abs{ 
		\nabla^{ 2 } W_{ \mathrm{pert} } } \leq C $, the function $ W_{ 
		\mathrm{pert} } + C \abs{ u }^{ 2 }/2 $ is convex for $ C > 0 $ 
		sufficiently large.
		Hence the functional 
		\begin{equation*}
			\tilde{\energy}_{ \varepsilon }  ( u )
			\coloneqq
			\int 
			\frac{ 1 }{ \varepsilon }
			\left( W ( u ) + \frac{ C }{ 2 } \abs{ u }^{ 2 } \right)
			+
			\frac{ \varepsilon}{ 2 }
			\abs{ \nabla u }^{ 2 }
			\dd{ x }
		\end{equation*}
		is convex on $ \wkp^{ 1 , 2 } \cap \lp^{ p } ( \flattorus ; \mathbb{ R 
		}^{ N } ) $. For a given $ \varphi \in \wkp^{ 1 , 2 } \cap \lp^{ p } ( 
		\flattorus ; \mathbb{ R }^{ N } ) $, we thus have that the function
		$ t \mapsto \ \tilde{\energy}_{ \varepsilon } ( u_{ n }^{ h } + t 
		\varphi ) $ is convex and differentiable, which yields that
		\begin{equation}
			\label{convexity_at_time_1}
			\tilde{\energy}_{ \varepsilon } ( u_{ n }^{ h } + \varphi )
			\geq
			\tilde{\energy}_{ \varepsilon } ( u_{ n }^{ h } ) + 
			\left.\dv{ t }\right|_{ t = 0 } \tilde{\energy}_{ \varepsilon } ( 
			u_{ n }^{ h } + t \varphi ).
		\end{equation} 
		But since $ u_{ n }^{ h } $ is a minimizer of the functional $\mathcal{ 
		F }$ defined by (\ref{de_giorgi_min_functional}), we have
		\begin{align*}
			&\left. \dv{ t } \right|_{ t= 0 }
			\tilde{\energy}_{ \varepsilon} ( u_{ n }^{ h } + t \varphi )
			\\
			={}&
			\left. \dv{ t } \right|_{ t = 0 } 
			\left(
			\mathcal{ F } ( u_{ n }^{ h } + t \varphi )
			+ 
			\frac{ C }{ 2 \varepsilon } 
			\int 
			\abs{ u_{ n }^{ h } + t \varphi }^{ 2 } 
			\dd{ x }
			-
			\frac{ 1 }{ 2h } 
			\int 
			\varepsilon
			\abs{ u_{n }^{ h } + t \varphi - u_{ n - 1 }^{ h } }^{ 2 } 
			\dd{ x }
			\right)
			\\
			={}& 
			\int
			\frac{ C }{ \varepsilon }
			\inner*{u_{ n }^{ h }}{\varphi}
			-
			\frac{ 1 }{ h }
			\varepsilon
			\inner*{ u_{ n }^{ h } - u_{ n - 1 } ^{ h } }{ \varphi }
			\dd{x}.
		\end{align*}
		By choosing $ \varphi = u_{ n - 1 }^{ h } - u_{ n }^{ h } $ 
		and using inequality (\ref{convexity_at_time_1}), we obtain
		\begin{equation*}
			\tilde{\energy}_{ \varepsilon } ( u_{ n - 1 }^{ h } ) 
			\geq
			\tilde{\energy}_{ \varepsilon} ( u_{ n }^{ h } )
			+
			\int
			\frac{ C }{ \varepsilon } 
			\left(
			\inner*{ u_{n }^{ h }}{ u_{ n - 1 }^{ h } }
			-
			\abs{ u_{ n }^{ h }}^{ 2 }
			\right)
			+
			\frac{ 1 }{ h }
			\varepsilon
			\abs{ u_{n }^{ h } - u_{ n -1 }^{ h } }^{ 2 }
			\dd{ x }.
		\end{equation*}
		By the definition of $ \tilde{ \energy }_{ \varepsilon }  $, this is 
		equivalent to
		\begin{align*}
			& \energy_{ \varepsilon } (u_{ n - 1 }^{ h } )
			\\
			\geq {} &
			\energy_{ \varepsilon } (u_{ n }^{ h } ) +
			\int
			\left(
			\frac{ C }{ 2 \varepsilon }
			-
			\frac{ C }{ \varepsilon }
			\right)
			\abs{ u_{ n }^{ h } }^{ 2 }
			-
			\frac{ C }{ 2 \varepsilon }
			\abs{ u_{n - 1 }^{ h } }^{ 2 }
			+ 
			\frac{ C }{ \varepsilon }
			\inner*{ u_{ n }^{ h } }{ u_{ n - 1 }^{ h } }
			+ 
			\frac{ 1 }{ h }
			\varepsilon
			\abs{ u_{ n }^{ h } - u_{ n - 1 }^{ h } }^{ 2 }
			\dd{ x }
			\\
			= {} &
			\energy_{ \varepsilon } ( u_{n }^{ h } )
			+
			\left( 
			\frac{ 1 }{ h }
			- 
			\frac{ C }{ 2 \varepsilon^{ 2 } }
			\right)
			\int 
			\varepsilon \abs{ u_{n }^{ h } - u_{ n - 1 }^{ h } }^{ 2 }
			\dd{ x },
		\end{align*}
		where we recognized the square in the last equality.
		This is the claimed estimate 
		(\ref{discrete_optimal_energy_dissipation}).
		
		\item[Step 4:] Hölder bounds for $ u_{ h } $.
		
		By iteratively applying the energy estimate 
		(\ref{discrete_optimal_energy_dissipation}), we 
		deduce that for every $ n \in \mathbb{ N } $, we have
		\begin{align}
			& \energy_{ \varepsilon } ( u_{ n }^{ h } )
			+
			\left(
			1
			-
			\frac{ C h  }{ 2 \varepsilon^{ 2 } }
			\right)
			\int_{ 0 }^{ n h }
			\int
			\varepsilon
			\abs{ \partial_{ t } u^{ h } }^{ 2 }
			\dd{ x }
			\dd{ t }
			\notag
			\\
			={} &
			\energy_{ \varepsilon } ( u_{ n }^{ h } )
			+
			\left(
			h - \frac{ C h^{ 2 }}{ 2 \varepsilon^{ 2 } }
			\right)
			\sum_{k = 1 }^{ n }
			\int
			\varepsilon
			\abs{
				\frac{ u_{ k }^{ h } - u_{ k - 1 }^{ h } }{ h }
			}^{ 2 }
			\dd{ x }
			\notag
			\\
			\label{energy_and_time_derivative_bound}
			\leq {} &
			\energy_{ \varepsilon } ( u^{ 0 } ).
		\end{align}
		This gives us with the use of Jensen's inequality for $ 0\leq s \leq t 
		\leq T $ and $ h > 0 $ sufficiently small that
		\begin{align}
			\norm{ u^{ h } ( t ) - u^{ h } ( s ) }_{ \lp^{ 2 } }
			& =
			\norm{ 
				\int_{ s }^{ t }
				\partial_{ t } u_{ h } ( \tau )
				\dd{ \tau }
			}_{ \lp^{ 2 } }
			\notag
			\\
			& \leq
			\sqrt{ t - s }
			\left(	
			\int_{ 0 }^{ T }
			\int
			\abs{ \partial_{ t } u^{ h } ( \tau, x ) }^{ 2 }
			\dd{ x }
			\dd{ \tau }
			\right)^{ 1/2 }
			\notag
			\\
			\label{hölder_continuity_of_uh}
			& \leq
			\sqrt{ t - s }
			\left(
			\varepsilon - \frac{ C h }{ 2 \varepsilon }
			\right)^{ - 1/2 }
			\left(  E_{ \varepsilon } ( u_{ \varepsilon }^{ 0 } ) \right)^{ 1/ 
			2 }.
		\end{align}
		Note that for the first inequality, we applied Fubini and enlarged the 
		time domain.
		We therefore obtain a uniform bound on the $ \lp^{ 2 }$-Hölder 
		continuity of 
		$ u^{ h } $ in time as $ h $ tends to zero since $ \varepsilon $ is 
		fixed.
		
		\item[Step 5:] Compactness.
		
		In order to apply the Arzelà--Ascoli theorem to the sequence $ (u_{ h 
		} ) $ in $ \cont \left( [ 0 , T ] ; \lp^{ 2 } ( \flattorus ; \mathbb{ R 
		}^{ N } ) \right) $ as $ h $ tends to zero, we need to check the 
		pointwise precompactness of the image and the equicontinuity of the 
		sequence. 
		The equicontinuity follows from the previous uniform Hölder estimate 
		(\ref{hölder_continuity_of_uh}).
		In order to check the pointwise precompactness, we need to verify that 
		for $ \delta > 0 $ sufficiently small and all $ t \in [ 0 , T ] $, the 
		set
		$ \{ u^{ h } ( t ) \}_{ \delta > h > 0 } $ is a precompact subset of $ 
		\lp^{ 2 } ( \flattorus ; \mathbb{ R }^{ N } ) $. 
		But this follows from the energy bound $ \energy_{ \varepsilon } ( u_{ 
		n }^{ h } ) \leq \energy_{ \varepsilon } ( u_{ 0 } ) $ given by 
		inequality (\ref{discrete_optimal_energy_dissipation}). In fact it 
		gives us a time-uniform bound on $ \norm{ \nabla u^{ h }( t ) }_{\lp^{ 
		2 }( \flattorus; \mathbb{ R }^{ N } ) } $ and on $ \norm{ u^{ h } ( t 
		)}_{ \lp^{ p } ( \flattorus; \mathbb{ R }^{ N } ) } $ and thus on $ 
		\norm{ u^{ h } ( t )}_{ \wkp^{ 1 , 2 } ( \flattorus; \mathbb{ R }^{ N } 
		) } $. 
		The compactness of the embedding
		$ \wkp^{ 1 , 2 } ( \flattorus ; \mathbb{ R }^{ N } ) \hookrightarrow \lp^{ 2 } ( \flattorus; \mathbb{ R }^{ N } ) $ thus yields the desired pointwise precompactness.
		
		Therefore we may apply Arzelà--Ascoli to obtain some $ u \in \cont^{ 1 
		/ 2 } \left( [ 0, T ] ; \lp^{ 2 } ( \flattorus ; \mathbb{ R }^{ N } 
		\right) $ and some  sequence $ h_{ n } \to 0 $ such that
		$ u^{ h_{ n } } $ converges uniformly to $ u $ on $ [ 0 , T ] $ with 
		respect to $ \norm{ \cdot }_{ \lp^{ 2 } ( \flattorus ; \mathbb{ R }^{ N 
		} ) } $.
		Note tat $ u $ is already $ 1/2 $-Hölder continuous sincce the Hölder 
		seminorm of $ u^{ h } $ stays bounded.
		
		\item[Step 6:] Additional regularity 1.
		
		We first want to argue that from our construction, one already obtains 
		that $ u $ stays bounded in time with respect to $ \wkp^{ 1 , 2 } 
		\left( \flattorus ; \mathbb{ R }^{ N } \right) $.
		We start by noticing that for a fixed $ t \in [ 0 , T ] $, we can find 
		by the pointwise precompactness of $ u^{ h } $ some
		non-relabelled subsequence and $ v \in \wkp^{ 1 , 2 } ( \flattorus ; 
		\mathbb{ R }^{ N } ) $ such that $ u^{ h_{ n } } ( t ) $ converges 
		weakly to $ v $ in $ \wkp^{ 1 , 2 } ( \flattorus ; \mathbb{ R }^{ N } ) 
		$. By the uniqueness of the limit, we already have $ u ( t ) = v $ 
		almost 
		everywhere, which yields $ u ( t ) \in \wkp^{ 1 , 2 } ( \flattorus, 
		\mathbb{ R }^{ N } ) $. Applying the lower semicontinuity of the norm, 
		we may also deduce 
		that
		\begin{equation*}
			\norm{ u ( t ) }_{ \wkp^{ 1 , 2 } }
			\leq
			\liminf_{ n \to \infty }
			\norm{ u^{ h_{ n } } ( t ) }_{ \wkp^{ 1 , 2 } }
			\leq
			C \energy_{ \varepsilon } ( u^{ 0 } ),
		\end{equation*}
		from which we deduce the desired bounded.	
		Secondly the boundedness of the energies
		\begin{equation*}
			\sup_{ 0 \leq t \leq T }
			\energy_{ \varepsilon } ( u ( t ) ) 
			< \infty
		\end{equation*}
		follows from the lower semicontinuity of the energy, the pointwise $ 
		\lp^{ 2 } $-convergence and the pointwise weak convergence in $ \wkp^{ 
		1 , 
		2 } $. Note that this is similar to Step 1.
		Lastly we want to argue that $ \partial_{ t } u \in \lp^{2 } \left( [ 0 
		, T ] \times \flattorus ; \mathbb{ R }^{ N } \right) $. 
		From the energy dissipation inequality 
		(\ref{energy_and_time_derivative_bound}) in Step 4, we deduce that $ 
		\partial_{ t } u^{ h } $ is a bounded sequence in $ \lp^{ 2 } \left( [ 
		0 , T ] 
		\times \flattorus; \mathbb{ R }^{ N } \right) $. Thus we find a 
		non-relabelled subsequence of $ h_{ n } $ and some $ w \in \lp^{ 2 } ( 
		[ 0 , T ] \times \flattorus ; \mathbb{ R }^{ N } ) $ such that $ u^{ 
		h_{n } } $ converges weakly to $ w $ in $ \lp^{ 2 } ( [0, T ] \times 
		\flattorus ; \mathbb{ R }^{ N } ) $. But then $ w $ is already the weak 
		time derivative of u since for any test function $ \varphi $, we have
		\begin{align*}
			\int_{ [ 0 , T ] \times \flattorus }
			\inner*{ u }{ \partial_{ t } \varphi }
			\dd{ x } \dd{ t }
			& =
			\lim_{ n \to \infty }
			\int_{ [ 0 , T ] \times \flattorus }
			\inner*{ u^{ h_{ n } } }{ \partial_{ t } \varphi }
			\dd{ x } \dd{ t }
			\\
			& =
			\lim_{n \to \infty }
			- \int_{ [ 0 , T ] \times \flattorus }
			\inner*{ \partial_{ t } u^{ h_{ n } } }{ \varphi }
			\dd{ x }\dd{ t }
			\\
			& =
			- \int_{ [ 0 , T ] \times \flattorus }
			\inner*{ w }{ \varphi }
			\dd{ x } \dd{ t }.
		\end{align*}
		
		\item[Step 7:] $ u $ is a weak solution.
		
		Going back to Step 1, we see that $ u_{ n }^{ h } $ solves the 
		Euler--Lagrange equation
		\begin{equation*}
			\int
			\frac{ 1 }{ \varepsilon^{ 2 } }
			\inner*{ \nabla W ( u_{ n }^{ h } ) }  { \varphi }
			+
			\inner*{ \nabla u_{ n }^{ h } }{ \nabla \varphi }
			+
			\inner*{ \frac{ u_{ n }^{ h } - u_{ n - 1 }^{ h } }{ h } }{ \varphi 
			}
			\dd{ x }
			=
			0
		\end{equation*}
		for any test function $ \varphi \in \cont^{ \infty } ( \flattorus ; 
		\mathbb{ 
		R }^{ N } )  $.
		Let $ \tilde{ u }^{ h } $ denote the piecewise constant interpolation 
		of the functions $ u^{ 0 }, u_{ 1 }^{ h } , \dotsc $ similar to the 
		construction in Step 2. 
		Then we have for any test function $ \psi \in \cont_{ \mathrm{ c } }^{ 
		\infty } \left( ( 0 , T ) \times \flattorus ; \mathbb{ R }^{ N } 
		\right) $ that
		\begin{equation}
				\label{el_eq_for_unh}
			\int_{ 0 }^{ T }
				\int
					\frac{ 1 }{ \varepsilon^{ 2 } }
					\inner*{ \nabla W ( \tilde{ u }^{ h } ) }{ \psi }
					+
					\inner*{ \nabla \tilde{ u }^{ h } }{ \nabla \psi }
					+
					\inner*{ \partial_{ t } u^{ h } }{ \psi }
				\dd{ x }
			\dd{ t }
			= 0.
		\end{equation}
		Let $ t \in [ 0 , T ] $. Since $ u_{ h } $ is defined as the piecewise 
		linear interpolation of the functions $ u_{ n}^{ h } $, we can find for 
		the sequence $ h_{ n } $ corresponding sequences $ \lambda_{ n } \in [ 
		0 , 1 ] $ and $ k_{ n } \in \mathbb{ N }$ such that 
		\begin{equation*}
			t = \lambda_{ n } ( k_{ n } - 1 ) h_{ n } + ( 1 - \lambda_{ n } ) 
			k_{ n } h_{ n } .
		\end{equation*}
		Therefore we can write
		\begin{equation*}
			u_{ h } ( t ) 
			=
			\frac{ k_{ n } h_{ n } - t }{ h_{ n } } u_{ k_{ n } - 1 }^{ h }
			+
			\frac{ t - ( k_{ n } - 1 ) h_{ n } }{ h_{ n } }
			u_{ k_{ n } }^{ h_{ n } }.
		\end{equation*}
		In order to pass to the limit in equation (\ref{el_eq_for_unh}), we first note that $ u_{ k_{ n } }^{ h_{ n } } $ converges to $ u ( t ) $ in $ \lp^{ 2 } ( \flattorus ; \mathbb{ R }^{ N } ) $ since
		\begin{align*}
			\norm{ u ( t ) - u_{ k_{ n } }^{ h_{ n } } }_{ \lp^{ 2 } ( \flattorus ; \mathbb{ R }^{ N } ) }
			& \leq
			\norm{ u ( t ) - u^{ h_{ n } } ( t ) }_{ \lp^{ 2 } ( \flattorus ; \mathbb{ R }^{ N } ) }
			+
			\norm{ u^{ h_{ n } } ( t ) - u^{ h_{ n } } ( k_{n } h_{ n } ) }_{ \lp^{ 2 } ( \flattorus, \mathbb{ R }^{ N } ) }
			\\
			& \lesssim
			\norm{ u ( t ) - u^{ h_{ n } } ( t ) }_{ \lp^{ 2 } ( \flattorus ; \mathbb{ R }^{ N } ) }
			+ 
			\sqrt{ h_{ n } },
		\end{align*}
		which goes to zero as $ n $ tends to infinity. Here we used the Hölder 
		estimate
		(\ref{hölder_continuity_of_uh}) for the second inequality.
		This implies that
		\begin{equation*}
			\int
			\inner*{ \nabla u ( t ) }{ \nabla \psi ( t ) }
			\dd{ x }
			=  
			\lim_{ n \to \infty }
			\int
			\inner*{ \nabla u_{ k_{ n } }^{ h_{ n } } }{ \nabla \psi( t ) }
			\dd{ x }.
		\end{equation*}
		Using the energy estimate (\ref{energy_and_time_derivative_bound}), we 
		obtain a majorant in time and therefore have
		\begin{equation*}
			\int_{ 0 }^{ T }
			\int
			\inner*{ \nabla u }{ \nabla \psi }
			\dd{ x }
			\dd{ t }
			=
			\lim_{ n \to \infty }
			\int_{0 }^{ T }
				\int
					\inner*{ \nabla \tilde{ u }^{ h_{ n } } }{ \nabla \psi }
				\dd{ x }
			\dd{ t }.
		\end{equation*}
		Now onto the first summand. We note that 
		\begin{equation*}
			\abs{ \inner*{ \nabla W ( \tilde{ u }^{ h_{ n } } ) }{ \psi 
			} }
			\lesssim
			\left( 1 + \abs{ \tilde{ u }^{ h_{ n } } }^{ p - 1 } \right) 
			\abs{ \psi  },
		\end{equation*}
		which is a bounded sequence in $ \lp^{ p'} \left( ( 0 , T ) \times 
		\flattorus ; \mathbb{ R 
		}^{ N } \right) $ by the energy estimate 
		(\ref{energy_and_time_derivative_bound}). 
		Therefore $ \inner*{ \nabla W 
		( \tilde{ u }^{ h_{ n } } ) }{ \psi } $ is equiintegrable. 
		Since we may also pass to a non-relabelled subsequence 
		which converges almost everywhere, we thus obtain by the continuity of 
		$ \nabla W $ that
		\begin{equation*}
			\int_{ 0 }^{ T }
			\int
			\inner*{ \nabla W ( u ) }{ \psi } 
			\dd{ x }
			\dd{ t }
			=
			\lim_{ n \to \infty }
			\int_{ 0 }^{ T }
			\int
			\inner*{ \nabla W ( \tilde{ u}^{ h_{ n } } ) }{ \psi }
			\dd{ x }
			\dd{ t }.
		\end{equation*}
			In Step 6, we moreover have shown that $ \partial_{ t } u^{ h_{ n } 
			} $ 
			converges weakly to $ \partial_{t } u $ in the space $ \lp^{ 2 } 
			\left( ( 0, 
			T ) 
		\times \flattorus ; \mathbb{ R }^{ N } \right) $. This yields the 
		convergence
		\begin{equation*}
			\int_{ 0 }^{ T }
			\int
			\inner*{ \partial_{ t } u }{ \psi }
			\dd{ x }
			=
			\lim_{ n \to \infty }
			\int_{ 0 }^{ T }
			\int
			\inner*{ \partial_{ t } u^{ h_{ n } }  }{ \psi }
			\dd{ x }
			\dd{ t }.
		\end{equation*}
		Combining our arguments, it follows from the Euler--Lagrange equation 
		(\ref{el_eq_for_unh}) that
		\begin{equation*}
			\int_{ 0 }^{ T }
			\int
			\frac{ 1 }{ \varepsilon^{ 2 } }
			\inner*{ \nabla W ( u ) }{ \psi }
			+
			\inner*{ \nabla u }{ \nabla \psi }
			+
			\inner*{ \partial_{ t } u }{ \psi }
			\dd{ x }
			\dd{ t }
			=
			0
		\end{equation*}
		for all $ \psi \in \cont_{ \mathrm{ c } }^{ \infty } ( ( 0 , T ) \times 
		\flattorus 
		; \mathbb{ R }^{ N } ) $. Moreover we can apply the fundamental theorem 
		of 
		calculus of variations that for almost all $ t \in ( 0 , T ) $ and all 
		$ \varphi \in \cont_{ \mathrm{c} }^{ \infty } \left( \flattorus ; 
		\mathbb{ R }^{ N } \right) $, 
		we have
		\begin{equation*}
			\int
			\frac{ 1 }{ \varepsilon^{ 2 } }
			\inner*{ \nabla W ( u( t ) ) }{ \varphi}
			+
			\inner*{ \nabla u( t ) }{ \nabla \varphi }
			+
			\inner*{ \partial_{ t } u ( t ) }{ \varphi }
			\dd{ x }
			=
			0
		\end{equation*}
		By continuity this extends to all $ \varphi \in 
		\wkp^{ 1 , 2 } \cap \lp^{ p } ( \mathbb{ T } ; \mathbb{ R }^{ N } ) $.
		
		\item[Step 8:] Additional regularity 2.
		
		In order to complete the proof, we still have to show that $ \partial_{ 
		i , j }^{ 2 } u$ and $ \nabla W ( u) $ are square-integrable.
		To this end, we test the weak formulation (\ref{ac_weak_equation}) with 
		finite differences. We therefore define the finite differences as
		\begin{equation*}
			\Delta_{ h }^{ + } w ( t , x ) \coloneqq \frac{ w ( t , x + h v ) - 
			w ( t , x ) }{ h }
			\quad\text{and}\quad
			\Delta_{ h }^{ - } w ( t , x ) \coloneqq \frac{ w ( t , x - h v ) - 
			w ( t , x ) }{ h }
		\end{equation*}
		for some $ h > 0 $ and $ v \in \mathbb{ R }^{ d } $. Thus plugging $ 
		\Delta_{ h }^{ - } \Delta_{ h }^{ + } u $ into the weak formulation 
		(\ref{ac_weak_equation}) yields by the transformation formula that
		\begin{equation*}
			0
			=
			\int_{ 0 }^{ T }
			\int
			\frac{ 1 }{ \varepsilon^{ 2 } }
			\inner*{ \Delta_{ h }^{ + } \nabla W ( u ) }{ \Delta_{ h }^{ + } u }
			+
			\inner*{ \nabla \Delta_{ h }^{ + } u }{ \nabla \Delta_{ h }^{ + } u }
			+
			\inner*{ \partial_{ t } \Delta_{ h }^{ + } u }{ \Delta_{ h }^{ + } u }
			\dd{ x }
			\dd{ t }.
		\end{equation*}
		This is equivalent to
		\begin{align*}
			& \int_{ 0 }^{ T }
			\int
			\abs{ \Delta_{ h }^{ + } \nabla u }^{ 2 }
			\dd{ x }
			\dd{ t }
			\\
			={} &
			-
			\int_{ 0 }^{ T }
			\int
			\partial_{ t } \left( \frac{ \abs{ \Delta_{ h }^{ + } u }^{ 2 } }{ 2 } \right)
			+
			\frac{ 1 }{ \varepsilon^{ 2 } }
			\inner*{ \Delta_{ h }^{ + } \nabla W ( u ) }{ \Delta_{ h }^{ + } u }
			\dd{ x }
			\dd{ t }
			\\
			={} &
			\int  
			\frac{ \abs{ \Delta_{ h }^{ + } u ( 0 ) }^{ 2 } - \abs{ \Delta_{ h }^{ + } u ( T ) }^{ 2 } }{ 2 }
			\dd{ x }
			\\
			& -
			\int_{ 0 }^{ 1 }
			\int_{ 0 }^{ T }
			\int
			\inner*{ 
				\diff^{ 2 } W \left( ( 1 - s ) u ( t, x  ) + s u( t,  x + h v ) ) \right) 
				\Delta_{ h }^{ + } u
			}{
				\Delta_{ h }^{ + } u 
			}
			\dd{ x }
			\dd{ t }
			\dd{ s }.
		\end{align*}
		The last equality follows from the fundamental theorem of calculus.
		Here the first summand is bounded uniformly in $ h $ by the uniform $ 
		\wkp^{ 1 , 2 } \left( 
		\flattorus ; \mathbb{ R }^{ N } \right) $-bound proven in Step 6. For 
		the second 
		summand, we partition $ W $ into the sum of $ W_{ \mathrm{conv} } $ and 
		$ W_{ \mathrm{pert} } $. The term involving the convex summand can then 
		be estimated by
		\begin{equation*}
			\int_{ 0 }^{ 1 }
			\int_{ 0 }^{ T }
			\int
			\inner*{ 
				\diff^{ 2 } W_{ \mathrm{conv} } \left( ( 1- s ) u ( t, x  ) + s u( t,  x + h v ) \right) 
				\Delta_{ h }^{ + } u
			}{
				\Delta_{ h }^{ + } u 
			}
			\dd{ x }
			\dd{ t }
			\dd{ s }
			\geq 0. 
		\end{equation*}
		For the term involving $ W_{ \mathrm{pert} } $, we get via the bound on 
		its second derivative that
		\begin{align*}
			& \abs{
				\int_{ 0 }^{ 1 }
				\int_{ 0 }^{ T }
				\int
				\inner*{ 
					\diff^{ 2 } W_{ \mathrm{pert} } \left( (1 - s ) u ( t, x  ) + s  u( t,  x + h v )  \right) 
					\Delta_{ h }^{ + } u
				}{
					\Delta_{ h }^{ + } u 
				}
				\dd{ x }
				\dd{ t }
				\dd{ s }
			}
			\\
			\lesssim {} &
			\int_{ 0 }^{ T }
			\int
			\abs{ \nabla u }^{ 2 }
			\dd{ x }
			\dd{ t },
		\end{align*}
		which is also finite. 
		Combining these estimates, we obtain that $ \int_{ 0 }^{ T } \int \abs{ \Delta_{ h }^{ + } \nabla u }^{ 2 } \dd{ x } \dd{ t } $ is uniformly bounded in $ h $. Applying our calculation to all directions $ v \in \mathbb{ R }^{ d } $, we get by the finite-differences theorem for all $ 1 \leq i, j \leq d $ that $ \partial_{ i , j }^{ 2 } u \in \lp^{ 2 } ( [ 0 , T ] \times \flattorus ; \mathbb{ R }^{ N } ) $.
		
		In order to obtain $ \nabla W ( u ) \in \lp^{ 2 } ( [ 0 , T ] \times 
		\flattorus ; \mathbb{ R }^{ N } ) $. We again consider the weak 
		formulation (\ref{ac_weak_equation}) and notice that since we have 
		already shown that both the time derivative and second space 
		derivatives of $ u $ are square-integrable, our claim follows from a 
		duality argument.
		
		\item[Step 9:] Sharp energy dissipation inequality for $ u $.
		
		Let $ 0 \leq t \leq T $ and define $ k_{ n } $ as in Step 7, where we 
		also established that $ u_{ k_{ n } }^{ h_{ n } } $ converges to $ u ( 
		t ) $ in $ \lp^{ 2 } \left( \flattorus ; \mathbb{ R }^{ N } \right) $.
		Thus we may pass to another non-relabelled subsequence to obtain 
		pointwise convergence almost everywhere. By Fatou's Lemma, we therefore 
		get
		\begin{equation*}
			\int
			\frac{ 1 }{ \varepsilon }
			W ( u ( t ) ) 
			\dd{ x }
			\leq
			\liminf_{ n \to \infty }
			\int
			\frac{ 1 }{ \varepsilon }
			W ( u_{ k_{ n } }^{ h_{ n } } )
			\dd{ x }.
		\end{equation*}
		Moreover we can deduce from the $ \lp^{ 2 } $-convergence that $ \nabla 
		u_{ k_{ n } }^{ h_{ n } } $ converges to $ \nabla u( t ) $ in the 
		distributional sense. But $ \nabla u_{ k_{ n } }^{ h_{ n } } $ is 
		uniformly bounded in $ \lp^{ 2 } ( \flattorus ; \mathbb{ R }^{ N } ) $ 
		by the energy dissipation inequality 
		(\ref{discrete_optimal_energy_dissipation}), thus we already obtain 
		that $ \nabla u_{ k_{ n } }^{ h_{ n } } $ converges weakly to $ \nabla 
		u ( t ) $ in $ \lp^{ 2 } ( \flattorus; \mathbb{ R }^{ N } ) $. 
		By the weak lower semicontinuity of the norm, this yields
		\begin{equation*}
			\int 
			\frac{ \varepsilon }{ 2 }
			\abs{ \nabla u ( t ) }^{ 2 }
			\dd{ x }
			\leq
			\liminf_{ n \to \infty }
			\int
			\frac{ \varepsilon }{ 2 }
			\abs{ \nabla u_{ k_{ n } }^{ h_{ n } } }^{ 2 }
			\dd{ x }.
		\end{equation*}
		Lastly we have by the weak convergence of $ \partial_{ t } u^{ h_{n } } 
		$ to $ \partial_{ t} u $ in $\lp^{ 2 } ( [ 0, T ] \times  \flattorus ; 
		\mathbb{ R }^{ N } ) $ proven in Step 6 that
		\begin{align*}
			\int_{ 0 }^{ t }
			\int
			\varepsilon
			\abs{ \partial_{ t } u }^{ 2 }
			\dd{ x }
			\dd{ t }
			& \leq
			\liminf_{ n \to \infty }
			\left( 1 - \frac{ C h_{ n } }{ 2 \varepsilon^{ 2 } } \right)
			\int_{ 0 }^{ t }
			\int
			\varepsilon
			\abs{ \partial_{ t } u^{ h_{ n } } }^{ 2 }
			\dd{ x }
			\dd{ t }
			\\
			& \leq
			\liminf_{ n \to \infty }
			\left( 1 - \frac{ C h_{ n } }{ 2 \varepsilon^{ 2 } } \right)
			\int_{ 0 }^{ k_{ n } h_{ n } }
			\int
			\varepsilon
			\abs{ \partial_{ t } u^{ h_{ n } } }^{ 2 }
			\dd{ x }
			\dd{ t }.		
		\end{align*}
		Summarizing these estimates, we obtain by the energy dissipation 
		inequality for $ u^{ h } $ (\ref{energy_and_time_derivative_bound}) 
		that for all $ 0 \leq t \leq T $, we have
		\begin{align*}
			& \energy_{ \varepsilon } ( u ( t ) )
			+
			\int_{ 0 }^{ t }
			\int
			\varepsilon
			\abs{ \partial_{ t } u }^{ 2 }
			\dd{ x }
			\dd{ t }
			\\
			\leq{} &
			\liminf_{ n \to \infty }
			\left(
			\energy_{ \varepsilon } ( u_{ k_{ n } } )
			+
			\left( 1 - \frac{ C h_{ n } }{ 2 \varepsilon^{ 2 } } \right)
			\int_{ 0 }^{ k_{ n } h_{ n } }
			\int
			\varepsilon
			\abs{ \partial_{ t } u^{ h } }^{ 2 }
			\dd{ x }
			\dd{ t }
			\right)
			\\
			\leq{} &
			\energy_{ \varepsilon } ( u_{ \varepsilon }^{ 0 } ).
		\end{align*}
		This is the desired estimate 
	\end{description}
	We have thus proven all of our claims.
\end{proof}

\begin{remark}
	The energy dissipation inequality 
	(\ref{discrete_optimal_energy_dissipation}) with the factor 
	$ 1/2h $ instead of $ 1/h- C/2\varepsilon^{ 2 } $ follows immediately from 
	the definition of our optimization problem.
	But is not optimal for fixed $ 
	\varepsilon $ if we want to study the behaviour as $ h $ tends to zero.
\end{remark}

\begin{remark}
	The optimal energy dissipation inequality (\ref{energy_dissipation_sharp}) 
	can be deduced via the formal calculation 
	\begin{align*}
		\dv{ t } \energy_{ \varepsilon } ( u )
		& =
		\int
		\frac{ 1 }{\varepsilon }
		\inner*{ \nabla W ( u ) }{ \partial_{ t } u }
		+
		\varepsilon
		\inner*{ \nabla u  }{ \nabla \partial_{ t } u }
		\dd{ x }
		\\
		& = 
		\int
		\inner*{ \frac{ 1 }{ \varepsilon } \nabla W ( u ) - \varepsilon \Delta u }{ \partial_{ t } u }
		\dd{ x }
		\\
		& =
		- \varepsilon \int \abs{ \partial_{ t } u_{ \varepsilon } }^{ 2 } \dd{ x }.
	\end{align*}
	In order to make this calculation rigorous, we however need to show some 
	additional regularity of $ u $.
	Of course, this is especially true if $ u $ is smooth for positive times, 
	and is thus to be expected, see again 
	\cite{de_mottoni_schatzmann_geometrical_evolution_of_developed_interfaces}.
	This  would give us equality in the energy dissipation inequality 
	(\ref{energy_dissipation_sharp}).
	Since we will only need the inequality, our proof will however suffice.
\end{remark}