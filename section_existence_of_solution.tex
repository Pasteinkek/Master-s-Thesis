\section{Existence of a solution}

With \Cref{solution_to_ac}, we are able to state our existence results for solutions of the Allen--Cahn equation. The proof uses De Giorgis minimizing movements scheme and arguments from the theory of gradient flows and has been briefly sketched in \cite{convergence_of_allen_cahn_equation_to_multiphase_mean_curvature_flow}.

\begin{theorem}
	\label{existence_of_ac_solution}
	Let $ u_{ \varepsilon}^{ 0 } \colon \flattorus \to \mathbb{ R }^{ N } $ be such that 
	$ \energy_{ \varepsilon } ( u_{ \varepsilon}^{ 0 } ) < \infty $.
	Then there exists a weak solution $ u_{ \varepsilon} $ to the Allen--Cahn equation (\ref{allen_cahn_eq}) in the sense of \Cref{solution_to_ac} with initial data $ u_{ \varepsilon}^{ 0 } $.
	Furthermore the solution satisfies the energy dissipation inequality
	\begin{equation}
		\label{energy_dissipation_sharp}
		\energy_{ \varepsilon } ( u_{ \varepsilon } ( t ) )
		+
		\int_{ 0 }^{ t }
		\int
		\varepsilon \abs{ \partial_{ t } u_{ \varepsilon } }^{ 2 }
		\dd{ x }
		\dd{ s }
		\leq
		\energy_{ \varepsilon } ( u_{ \varepsilon}^{ 0 } )
	\end{equation}
	for every $ t \in [ 0 , T ] $ and we additionally have
	$
	\partial_{ i , j }^{ 2 } u_{ \varepsilon }, \nabla W ( u_{ \varepsilon } ) \in \lp^{ 2 } ( [0, T] \times \flattorus ; \mathbb{ R }^{ N } ) 
	$
	for all $ 1 \leq i, j \leq d $. In particular we can test the weak form (\ref{ac_weak_equation}) with $ \partial_{ i , j }^{ 2 } u_{ \varepsilon } $.
\end{theorem}

\begin{proof}
	\begin{description}[wide=0pt]
		\item[Step 1:] A minimization problem
		
		Fix some $ h > 0 $, $ u_{ n - 1 } \in \wkp^{1 , 2 }\cap \lp^{ p } ( \flattorus; \mathbb{ R }^{ N } ) $ and consider the functional
		\begin{align}
			\label{de_giorgi_min_functional}
			\mathcal{ F } \colon \wkp^{ 1, 2 } \cap \lp^{ p } ( \flattorus ; \mathbb{ R }^{ N } )
			& \to
			\mathbb{ R }
			\\
			u & \mapsto 
			\energy_{ \varepsilon } ( u ) + 
			\frac{ 1 }{ 2h } \int \varepsilon \abs{ u - u_{ n - 1 } }^{ 2 } \dd{x }
			\notag .
		\end{align}
		Then $ \mathcal{ F } $ is coercive with respect to 
		$ \norm{ \cdot }_{ \wkp^{ 1, 2 } } $ 
		and bounded from below by zero, thus we may take a $ \wkp^{ 1 , 2 } $-bounded sequence  
		$ (v_{ k } )_{ k \in \mathbb{ N } } $ in $ \wkp^{ 1 , 2} \cap \lp^{ p } ( \flattorus ; \mathbb{ R }^{ N } ) $ such that 
		$ \mathcal{ F } ( v_{ k } ) \to \inf \mathcal{ F } $ as $ k \to \infty $. 
		These $ v_{ k } $ have a non-relabelled subsequence which converges weakly in $ \wkp^{ 1 , 2 } ( \flattorus ; \mathbb{ R }^{ N } ) $ and strongly in $ \lp^{ 2 } ( \flattorus, \mathbb{ R }^{ N } ) $ to some $ u \in \wkp^{ 1, 2 } ( \flattorus; \mathbb{ R }^{ N } ) $.
		Moreover $ v_{ k } $ is bounded in $ \lp^{ p } \left( \flattorus ; \mathbb{ R }^{ N } \right)$ by the lower growth assumption (\ref{polynomial_growth}) on $ W $
		and thus obtain by a duality argument that $ u \in \wkp^{ 1 , 2 } \cap \lp^{ p } ( \flattorus ; \mathbb{ R }^{ N } ) $.
		
		Lastly we have 
		\begin{align}
			\frac{ 1 }{ 2 h }
			\int \varepsilon \abs{ u - u_{ n - 1 } }^{ 2 } \dd{ x}
			& \leq
			\liminf_{ k \to \infty }
			\frac{ 1 }{ 2h }
			\int \varepsilon \abs{ v_{ k } - u_{ n -1 } }^{ 2 } \dd{ x },
			\\
			\int \frac{ \varepsilon }{ 2 } \abs{ \nabla u }^{ 2 } \dd{ x }
			& \leq
			\liminf_{ k \to \infty }
			\int \frac{ \varepsilon }{ 2 } \abs{ \nabla v_{ k } }^{ 2 } \dd{ x }
		\end{align}
		by the weak convergence in $ \wkp^{ 1 , 2 } ( \flattorus , \mathbb{ R }^{ N } )$.
		By passing to another non-relabelled subsequence which converges pointwise almost everywhere, we moreover achieve by continuity of $ W $ that
		\begin{equation*}
			\int \frac{ 1 }{ \varepsilon } W ( u ) \dd{ x }
			=
			\int \liminf_{ k \to \infty } \frac{ 1 }{ \varepsilon } W ( v_{k } ) \dd{ x }
			\leq
			\liminf_{ k \to \infty } \int \frac{ 1 }{ \varepsilon } W ( v_{ k } ) \dd{x},
		\end{equation*}
		which proves that u is a minimizer of $ \mathcal{ F } $.
		
		\item[Step 2:] Minimizing movements scheme 
		
		By iteratively choosing minimizers $ u_{ n }^{ h } $ from Step 1, we obtain a sequence of functions $ u_{ \varepsilon }^{ 0 }, u_{ 1 }^{ h } , \dotsc $. Thus we may define a function $ u \in \mathrm{ C } $ as the piecewise linear interpolation at the time-steps $ 0, h , 2h, \dotsc $ of these functions.
		
		\item[Step 3:] Sharp energy dissipation inequality for $ u_{ n }^{ h } $
		
		We claim that there exists some constant $ C > 0 $ such that for all $ h > 0 $ and $ n \in \mathbb{ N } $, we have
		\begin{equation}
			\label{discrete_optimal_energy_dissipation}
			\energy_{ \varepsilon } ( u_{ n }^{ h } )
			+
			\left( \frac{ 1 }{ h } - \frac{ C }{ 2 \varepsilon^{ 2 } } \right)
			\int \varepsilon \abs{ u_{ n }^{ h } - u_{ n - 1 }^{ h } }^{ 2 } \dd{ x }
			\leq
			\energy_{ \varepsilon } ( u_{ n - 1 }^{ h } ).
		\end{equation}
		In order to prove this inequality, we notice that since $ \abs{ \nabla^{ 2 } W_{ \mathrm{pert} } } \leq C $, the function $ W_{ \mathrm{pert} } + C \abs{ u }^{ 2 }/2 $ is convex for $ C > 0 $ sufficiently large, thus the functional 
		\begin{equation*}
			\tilde{\energy}_{ \varepsilon }  ( u )
			\coloneqq
			\int 
			\frac{ 1 }{ \varepsilon }
			\left( W ( u ) + \frac{ C }{ 2 } \abs{ u }^{ 2 } \right)
			+
			\frac{ \varepsilon}{ 2 }
			\abs{ \nabla u }^{ 2 }
			\dd{ x }
		\end{equation*}
		is convex on $ \wkp^{ 1 , 2 } \cap \lp^{ p } ( \flattorus ; \mathbb{ R }^{ N } ) $. For a given $ \xi \in \wkp^{ 1 , 2 } \cap \lp^{ p } ( \flattorus ; \mathbb{ R }^{ N } ) $, we thus have that the function
		$ t \mapsto \ \tilde{\energy}_{ \varepsilon } ( u_{ n }^{ h } + t \xi ) $ is convex and differentiable, which yields that
		\begin{equation}
			\label{convexity_at_time_1}
			\tilde{\energy}_{ \varepsilon } ( u_{ n }^{ h } + \xi )
			\geq
			\tilde{\energy}_{ \varepsilon } ( u_{ n }^{ h } ) + 
			\left.\dv{ t }\right|_{ t = 0 } \tilde{\energy}_{ \varepsilon } ( u_{ n }^{ h } + t \xi).
		\end{equation} 
		But since $ u_{ n }^{ h } $ is a minimizer of the functional $\mathcal{ 
		F }$ defined by (\ref{de_giorgi_min_functional}), we have
		\begin{align*}
			\left. \dv{ t } \right|_{ t= 0 }
			\tilde{\energy}_{ \varepsilon} ( u_{ n }^{ h } + t \xi )
			={}&
			\left. \dv{ t } \right|_{ t = 0 } 
			\mathcal{ F } ( u_{ n }^{ h } + t \xi )
			\\
			{}&+ 
			\frac{ C }{ 2 \varepsilon } 
			\int 
			\abs{ u_{ n }^{ h } + t \xi }^{ 2 } 
			\dd{ x }
			-
			\frac{ 1 }{ 2h } 
			\int 
			\varepsilon
			\abs{ u_{n }^{ h } + t \xi - u_{ n - 1 }^{ h } }^{ 2 } 
			\dd{ x }
			\\
			={}& 
			\int
			\frac{ C }{ \varepsilon }
			\inner*{u_{ n }^{ h }}{\xi}
			-
			\frac{ 1 }{ h }
			\varepsilon
			\inner*{ u_{ n }^{ h } - u_{ n - 1 } ^{ h } }{ \xi }
			\dd{x}.
		\end{align*}
		Plugging $ \xi = u_{ n - 1 }^{ h } - u_{ n }^{ h } $ into this equation and using inequality (\ref{convexity_at_time_1})  thus yields
		\begin{equation*}
			\tilde{\energy}_{ \varepsilon } ( u_{ n - 1 }^{ h } ) 
			\geq
			\tilde{\energy}_{ \varepsilon} ( u_{ n }^{ h } )
			+
			\int
			\frac{ C }{ \varepsilon } 
			\left(
			\inner*{ u_{ n - 1 }^{ h } }{ u_{n }^{ h }}
			-
			\abs{ u_{ n }^{ h }}^{ 2 }
			\right)
			+
			\frac{ 1 }{ h }
			\varepsilon
			\abs{ u_{n }^{ h } - u_{ n -1 }^{ h } }^{ 2 }
			\dd{ x },
		\end{equation*}
		which is equivalent to
		\begin{align*}
			\energy_{ \varepsilon } (u_{ n - 1 }^{ h } )
			\geq {} &
			\energy_{ \varepsilon } (u_{ n }^{ h } )
			\\
			& +
			\int
			\left(
			\frac{ C }{ 2 \varepsilon }
			-
			\frac{ C }{ \varepsilon }
			\right)
			\abs{ u_{ n }^{ h } }^{ 2 }
			-
			\frac{ C }{ 2 \varepsilon }
			\abs{ u_{n - 1 }^{ h } }^{ 2 }
			+ 
			\frac{ C }{ \varepsilon }
			\inner*{ u_{ n - 1 }^{ h } }{ u_{ n }^{ h } }
			+ 
			\frac{ 1 }{ h }
			\varepsilon
			\abs{ u_{ n }^{ h } - u_{ n - 1 }^{ h } }^{ 2 }
			\dd{ x }
			\\
			= {} &
			\energy_{ \varepsilon } ( u_{n }^{ h } )
			+
			\left( 
			\frac{ 1 }{ h }
			- 
			\frac{ C }{ 2 \varepsilon^{ 2 } }
			\right)
			\int 
			\varepsilon \abs{ u_{n }^{ h } - u_{ n - 1 }^{ h } }^{ 2 }
			\dd{ x },
		\end{align*}
		which is the claimed estimate (\ref{discrete_optimal_energy_dissipation}).
		
		\item[Step 4:] Hölder bounds for $ u_{ h } $
		
		From the energy estimate (\ref{discrete_optimal_energy_dissipation}) we deduce via an induction that
		\begin{align}
			& \energy_{ \varepsilon } ( u_{ n }^{ h } )
			+
			\left(
			1
			-
			\frac{ C h  }{ 2 \varepsilon^{ 2 } }
			\right)
			\int_{ 0 }^{ n h }
			\int
			\varepsilon
			\abs{ \partial_{ t } u^{ h } }^{ 2 }
			\dd{ x }
			\dd{ t }
			\notag
			\\
			={} &
			\energy_{ \varepsilon } ( u_{ n }^{ h } )
			+
			\left(
			h - \frac{ C h^{ 2 }}{ 2 \varepsilon^{ 2 } }
			\right)
			\sum_{k = 1 }^{ n }
			\int
			\varepsilon
			\abs{
				\frac{ u_{ n }^{ h } - u_{ n - 1 }^{ h } }{ h }
			}^{ 2 }
			\dd{ x }
			\notag
			\\
			\label{energy_and_time_derivative_bound}
			\leq {} &
			\energy_{ \varepsilon } ( u_{ \varepsilon }^{ 0 } ).
		\end{align}
		This gives us with the use of Jensen's inequality for $ 0\leq s \leq t 
		\leq T $ and $ h > 0 $ sufficiently small that
		\begin{align}
			\norm{ u^{ h } ( t ) - u^{ h } ( s ) }_{ \lp^{ 2 } }
			& =
			\norm{ 
				\int_{ s }^{ t }
				\partial_{ t } u_{ h } ( \tau )
				\dd{ \tau }
			}_{ \lp^{ 2 } }
			\notag
			\\
			& \leq
			\sqrt{ t - s }
			\left(	
			\int_{ 0 }^{ T }
			\int
			\abs{ \partial_{ t } u^{ h } ( \tau, x ) }^{ 2 }
			\dd{ x }
			\dd{ \tau }
			\right)^{ 1/2 }
			\notag
			\\
			\label{hölder_continuity_of_uh}
			& \leq
			\sqrt{ t - s }
			\left(
			\varepsilon - \frac{ C h }{ 2 \varepsilon }
			\right)^{ - 1/2 }
			\left(  E_{ \varepsilon } ( u_{ \varepsilon }^{ 0 } ) \right)^{ 1/ 2 },
		\end{align}
		which gives us a uniform bound on the $ \lp^{ 2 }$-Hölder continuity of $ u^{ h } $ in time as $ h $ tends to zero.
		
		\item[Step 5:] Compactness
		
		In order to apply Arzelà-Ascoli for the sequence $ (u_{ h } ) $ as $ h $ tends to zero, we need to check pointwise precompactness of the image and equicontinuity of the sequence. 
		The equicontinuity follows from the previous estimate (\ref{hölder_continuity_of_uh}).
		In order to check the pointwise precompactness, we need to verify that for all $ t \in [ 0 , T ] $, the set
		$ \{ u^{ h } ( t ) \}_{ \delta > h > 0 } $ is a precompact subset of $ \lp^{ 2 } ( \flattorus ; \mathbb{ R }^{ N } ) $ (for $ \delta > 0 $ sufficiently small). 
		But this follows from the energy bound $ \energy_{ \varepsilon } ( u_{ n }^{ h } ) \leq \energy_{ \varepsilon } ( u_{ 0 } ) $ (given by inequality (\ref{discrete_optimal_energy_dissipation})), which gives us a time-uniform bound on $ \norm{ \nabla u^{ h }( t ) }_{\lp^{ 2 }( \flattorus; \mathbb{ R }^{ N } ) } $ and on $ \norm{ u^{ h } ( t )}_{ \lp^{ p } ( \flattorus; \mathbb{ R }^{ N } ) } $ and thus on $ \norm{ u^{ h } ( t )}_{ \wkp^{ 1 , 2 } ( \flattorus; \mathbb{ R }^{ N } ) } $. The compact embedding
		$ \wkp^{ 1 , 2 } ( \flattorus ; \mathbb{ R }^{ N } ) \hookrightarrow \lp^{ 2 } ( \flattorus; \mathbb{ R }^{ N } ) $ thus yields the desired pointwise precompactness.
		
		Therefore we may apply Arzelà--Ascoli to obtain some $ u \in \cont^{ 1 , 2 } \left( [ 0, T ] ; \lp^{ 2 } ( \flattorus ; \mathbb{ R }^{ N } \right) $ and some  sequence $ h_{ n } \to 0 $ such that
		$ u^{ h_{ n } } $ converges uniformly to $ u $ on $ [ 0 , T ] $ with respect to $ \norm{ \cdot }_{ \lp^{ 2 } ( \flattorus ; \mathbb{ R }^{ N } ) } $.
		
		\item[Step 6:] Additional regularity 1
		
		We first want to argue that from our construction, one already obtains that $ u \in \lp^{ \infty } \left( [ 0 , T ] ; \wkp^{ 1, 2 } ( \flattorus ; \mathbb{ R }^{ N } ) \right) $
		For this we first notice that for a fixed $ t \in [ 0 , T ] $, the 
		sequence $ u^{ h_{ n } } ( t ) $ is by the energy bound 
		(\ref{discrete_optimal_energy_dissipation}) a bounded sequence in $ 
		\wkp^{ 1 , 2 } ( \flattorus ; \mathbb{ R }^{ N } ) $, thus we find some 
		non-relabelled subsequence and $ v \in \wkp^{ 1 , 2 } ( \flattorus ; 
		\mathbb{ R }^{ N } ) $ such that $ u^{ h_{ n } } ( t ) $ converges 
		weakly to $ v $ in $ \wkp^{ 1 , 2 } ( \flattorus ; \mathbb{ R }^{ N } ) 
		$. By uniqueness of the limit, we already have $ u ( t ) = v $ almost 
		everywhere, which yields $ u ( t ) \in \wkp^{ 1 , 2 } ( \flattorus, 
		\mathbb{ R }^{ N } ) $, and by lower semicontinuity, we may also deduce 
		that
		\begin{equation*}
			\norm{ u ( t ) }_{ \wkp^{ 1 , 2 } }
			\leq
			\liminf_{ n \to \infty }
			\norm{ u^{ h_{ n } } ( t ) }_{ \wkp^{ 1 , 2 } }
			\leq
			C \energy_{ \varepsilon },
		\end{equation*}
		from which we deduce that $ u \in \lp^{ \infty } \left( [ 0 , T ] ; \wkp^{ 1 , 2 } ( \flattorus ; \mathbb{ R }^{ N } ) \right) $.	
		
		Secondly the boundedness of the energies
		\begin{equation*}
			\sup_{ 0 \leq t \leq T }
			\energy_{ \varepsilon } ( u ( t ) ) 
			< \infty
		\end{equation*}
		follows from the lower semicontinuity of the energy and the pointwise $ \lp^{ 2 } $ convergence and pointwise weak convergence in $ \wkp^{ 1 , 2 } $ as described in step 1.
		
		Lastly we want to argue that $ \partial_{ t } u \in \lp^{2 } \left( [ 0 , T ] \times \flattorus ; \mathbb{ R }^{ N } \right) $. From inequality (\ref{energy_and_time_derivative_bound}) in step 4, we deduce that $ \partial_{ t } u^{ h } $ is a bounded sequence in $ \lp^{ 2 } [ 0 , T ] \times \flattorus; \mathbb{ R }^{ N } ) $. Thus we find a non-relabelled subsequence of $ h_{ n } $ and some $ w \in \lp^{ 2 } ( [ 0 , T ] \times \flattorus ; \mathbb{ R }^{ N } ) $ such that $ u^{ h_{n } } $ converges weakly to $ w $ in $ \lp^{ 2 } ( [0, T ] \times \flattorus ; \mathbb{ R }^{ N } ) $. But then $ w $ is already the weak time derivative of u since for any testfunction $ \xi $, we have
		\begin{align*}
			\int_{ [ 0 , T ] \times \flattorus }
			\inner*{ u }{ \partial_{ t } \xi }
			\dd{ x } \dd{ t }
			& =
			\lim_{ n \to \infty }
			\int_{ [ 0 , T ] \times \flattorus }
			\inner*{ u^{ h_{ n } } }{ \partial_{ t } \xi }
			\dd{ x } \dd{ t }
			\\
			& =
			\lim_{n \to \infty }
			- \int_{ [ 0 , T ] \times \flattorus }
			\inner*{ \partial_{ t } u^{ h_{ n } } }{ \xi }
			\dd{ x }\dd{ t }
			\\
			& =
			- \int_{ [ 0 , T ] \times \flattorus }
			\inner*{ w }{ \xi }
			\dd{ x } \dd{ t }.
		\end{align*}
		
		\item[Step 7:] $ u $ is a weak solution
		
		Going back to step 1, we see that $ u_{ n }^{ h } $ solves the Euler--Lagrange equation
		\begin{equation}
			\label{el_eq_for_unh}
			\int
			\frac{ 1 }{ \varepsilon^{ 2 } }
			\inner*{ \nabla W ( u_{ n }^{ h } ) }  { \xi }
			+
			\inner*{ \nabla u_{ n }^{ h } }{ \nabla \xi }
			+
			\inner*{ \frac{ u_{ n }^{ h } - u_{ n - 1 }^{ h } }{ h } }{ \xi }
			\dd{ x }
			=
			0
		\end{equation}
		for any testfunction $ \xi \in \cont^{ \infty } ( \flattorus ; \mathbb{ R }^{ N } )  $.
		Let $ t \in [ 0 , T ] $. Since $ u_{ h } $ is defined as the pointwise linear interpolation of the functions $ u_{ n}^{ h } $, we find for the sequence $ h_{ n } $ corresponding sequences $ \lambda_{ n } \in [ 0 , 1 ] $ and $ k_{ n } \in \mathbb{ N }$ such that 
		\begin{equation*}
			t = \lambda_{ n } ( k_{ n } - 1 ) h_{ n } + ( 1 - \lambda_{ n } ) k_{ n } h_{ n } 
		\end{equation*}
		and therefore we can write
		\begin{equation*}
			u_{ h } ( t ) 
			=
			\frac{ k_{ n } h_{ n } - t }{ h_{ n } } u_{ k_{ n } - 1 }^{ h }
			+
			\frac{ t - ( k_{ n } - 1 ) h_{ n } }{ h_{ n } }
			u_{ k_{ n } }^{ h_{ n } }.
		\end{equation*}
		In order to pass to the limit in equation (\ref{el_eq_for_unh}), we first note that $ u_{ k_{ n } }^{ h_{ n } } $ converges to $ u ( t ) $ in $ \lp^{ 2 } ( \flattorus ; \mathbb{ R }^{ N } ) $ since
		\begin{align*}
			\norm{ u ( t ) - u_{ k_{ n } }^{ h_{ n } } }_{ \lp^{ 2 } ( \flattorus ; \mathbb{ R }^{ N } ) }
			& \leq
			\norm{ u ( t ) - u^{ h_{ n } } ( t ) }_{ \lp^{ 2 } ( \flattorus ; \mathbb{ R }^{ N } ) }
			+
			\norm{ u^{ h_{ n } } ( t ) - u^{ h_{ n } } ( k_{n } h_{ n } ) }_{ \lp^{ 2 } ( \flattorus, \mathbb{ R }^{ N } ) }
			\\
			& \lesssim
			\norm{ u ( t ) - u^{ h_{ n } } ( t ) }_{ \lp^{ 2 } ( \flattorus ; \mathbb{ R }^{ N } ) }
			+ 
			\sqrt{ h_{ n } }
			\tag{\ref{hölder_continuity_of_uh}},
		\end{align*}
		which goes to zero as $ n $ tends to infinity.
		This implies that 
		\begin{align*}
			\int
			\inner*{ \nabla u ( t ) }{ \nabla \xi }
			\dd{ x }
			& = 
			-
			\int
			\inner*{ u ( t ) }{ \divg \nabla \xi  }
			\dd{ x }
			\\
			& = 
			-
			\lim_{ n \to \infty }
			\int
			\inner*{ u_{ k_{ n } }^{ h_{ n } } }{ \divg \nabla \xi  }
			\dd{x }
			\\
			& = 
			\lim_{ n \to \infty }
			\int
			\inner*{ \nabla u_{ k_{ n } }^{ h_{ n } } }{ \nabla \xi }
			\dd{ x }.
		\end{align*}
		In step 6, we moreover have shown that $ \partial_{ t } u^{ h } $ converges weakly to $ \partial_{t } u $ in $ \lp^{ 2 } ( [ 0, T ] \times \flattorus ; \mathbb{ R }^{ N } ) $. This yields, by choosing cylindrical testfunctions, that $ \partial_{ t } u^{ h } ( t ) $ converges weakly to $ \partial_{ t } u ( t ) $ in $ \lp^{ 2 } ( \flattorus ; \mathbb{ R }^{ N } ) $ for almost every $ t \in [ 0 , T ] $.
		Thus we obtain for almost every $ t\in [ 0 , T ] $ the convergence
		\begin{equation*}
			\int
			\inner*{ \partial_{ t } u ( t ) }{ \xi }
			\dd{ x }
			=
			\lim_{ n \to \infty }
			\int
			\inner*{ \partial_{ t } u ( t ) }{ \xi }
			\dd{ x }
			=
			\lim_{ n \to \infty }
			\int
			\inner*{ \frac{ u_{ k_{ n } }^{ h_{ n } } - u_{ k_{ n } - 1 }^{ h_{ n } } }{ h_{ n } } }{ \xi }
			\dd{ x }.
		\end{equation*}
		To obtain the weak equation, we still need to prove convergence of remaining term. For this we note that 
		\begin{equation*}
			\frac{ 1 }{ \varepsilon^{ 2 } }
			\abs{ \inner*{ \nabla W ( u_{ k_{ n } }^{ h_{ n } } ) }{ \xi } }
			\lesssim
			\left( 1 + \abs{ u_{  k_{ n } }^{ h_{ n } } }^{ p - 1 } \right) \abs{ \xi },
		\end{equation*}
		which is a bounded sequence in $ \lp^{ p'} ( \flattorus ; \mathbb{ R }^{ N } ) $. Since we also may pass to a non-relabelled subsequence which converges almost everywhere, we thus obtain that
		\begin{equation*}
			\int
			\inner*{ \nabla W ( u ( t ) ) }{ \xi } 
			\dd{ x }
			=
			\lim_{ n \to \infty }
			\int
			\inner*{ \nabla W ( u_{ k_{ n } }^{ h_{ n } } ) }{ \xi }
			\dd{ x }.
		\end{equation*}
		Therefore it follows from the Euler--Lagrange equation (\ref{el_eq_for_unh}) that
		\begin{equation*}
			\int
			\frac{ 1 }{ \varepsilon^{ 2 } }
			\inner*{ \nabla W ( u ( t )) }{ \xi }
			+
			\inner*{ \nabla u }{ \nabla \xi }
			+
			\inner*{ \partial_{ t } u }{ \xi }
			\dd{ x }
			=
			0
		\end{equation*}
		for all $ \xi \in \cont^{ \infty } ( \flattorus ; \mathbb{ R }^{ N } ) $. By continuity this extends to all $ \xi \in \wkp^{ 1 , 2 } \cap \lp^{ p } ( \mathbb{ T } ; \mathbb{ R }^{ N } ) $.
		
		Thus the last thing to check is that
		\begin{equation*}
			\sup_{ 0 \leq t \leq T }
			\energy_{ \varepsilon } ( u( t ) )
			<
			\infty,
		\end{equation*}
		which follows from the next step.
		
		\item[Step 8:] Sharp energy dissipation inequality for $ u $
		
		Let $ 0 \leq t \leq T $ and define $ k_{ n } $ as in step 7, where we also established that $ u_{ k_{ n } }^{ h_{ n } } $ converges to $ u ( t ) $ in $ \lp^{ 2 } \left( \flattorus ; \mathbb{ R }^{ N } \right) $, and thus we may pass to another non-relabelled subsequence to obtain pointwise convergence almost everywhere. By Fatou's Lemma, we thus obtain
		\begin{equation*}
			\int
			\frac{ 1 }{ \varepsilon }
			W ( u ( t ) ) 
			\dd{ x }
			\leq
			\liminf_{ n \to \infty }
			\int
			\frac{ 1 }{ \varepsilon }
			W ( u_{ k_{ n } }^{ h_{ n } } )
			\dd{ x }.
		\end{equation*}
		Moreover we can deduce from the $ \lp^{ 2 } $-convergence that $ \nabla u_{ k_{ n } }^{ h_{ n } } $ converges to $ u( t ) $ in the distributional sense. But $ \nabla u_{ k_{ n } }^{ h_{ n } } $ is uniformly bounded in $ \lp^{ 2 } ( \flattorus ; \mathbb{ R }^{ N } ) $ by the energy dissipation inequality (\ref{discrete_optimal_energy_dissipation}), thus we already obtain that $ \nabla u_{ k_{ n } }^{ h_{ n } } $ converges weakly to $ \nabla u ( t ) $ in $ \lp^{ 2 } ( \flattorus; \mathbb{ R }^{ N } ) $ which yields
		\begin{equation*}
			\int 
			\frac{ \varepsilon }{ 2 }
			\abs{ \nabla u ( t ) }^{ 2 }
			\dd{ x }
			\leq
			\liminf_{ n \to \infty }
			\int
			\frac{ \varepsilon }{ 2 }
			\abs{ \nabla u_{ k_{ n } }^{ h_{ n } } }^{ 2 }
			\dd{ x }.
		\end{equation*}
		Lastly we have by the weak convergence of $ \partial_{ t } u^{ h_{n } } $ to $ \partial{ t} u $ in $\lp^{ 2 } ( \flattorus ; \mathbb{ R }^{ N } ) $ proven in step 6 that
		\begin{align*}
			\int_{ 0 }^{ t }
			\int
			\varepsilon
			\abs{ \partial_{ t } u }^{ 2 }
			\dd{ x }
			\dd{ t }
			& \leq
			\liminf_{ n \to \infty }
			\left( 1 - \frac{ C h_{ n } }{ 2 \varepsilon^{ 2 } } \right)
			\int_{ 0 }^{ t }
			\int
			\varepsilon
			\abs{ \partial_{ t } u^{ h_{ n } } }^{ 2 }
			\dd{ x }
			\dd{ t }
			\\
			& \leq
			\liminf_{ n \to \infty }
			\left( 1 - \frac{ C h_{ n } }{ 2 \varepsilon^{ 2 } } \right)
			\int_{ 0 }^{ k_{ n } h_{ n } }
			\int
			\varepsilon
			\abs{ \partial_{ t } u^{ h_{ n } } }^{ 2 }
			\dd{ x }
			\dd{ t }.		
		\end{align*}
		Summarizing these estimates, we obtain by the energy dissipation inequality for $ u^{ h } $ (\ref{energy_and_time_derivative_bound}) that for all $ 0 \leq t \leq T $ we have
		\begin{align*}
			\energy_{ \varepsilon } ( u ( t ) )
			+
			\int_{ 0 }^{ t }
			\int
			\varepsilon
			\abs{ \partial_{ t } u }^{ 2 }
			\dd{ x }
			\dd{ t }
			& \leq
			\liminf_{ n \to \infty }
			\energy_{ \varepsilon } ( u_{ k_{ n } } )
			+
			\left( 1 - \frac{ C h_{ n } }{ 2 \varepsilon^{ 2 } } \right)
			\int_{ 0 }^{ k_{ n } h_{ n } }
			\int
			\varepsilon
			\abs{ \partial_{ t } u^{ h } }^{ 2 }
			\dd{ x }
			\dd{ t }
			\\
			& \leq
			\energy_{ \varepsilon } ( u_{ \varepsilon }^{ 0 } ).
		\end{align*}
		
		\item[Step 8:] Additional regularity 2
		
		In order to complete the proof, we still have to show that $ \partial_{ i , j }^{ 2 } u$ and$ \nabla W ( u) $ are elements of $ \lp^{ 2 } ( [ 0 , T ] \times \flattorus ; \mathbb{ R }^{ N } ) $.
		In order to show that the second partial derivatives of $ u $ are square-integrable, we test the weak formulation (\ref{ac_weak_equation}) with the finite differences. To this end, define the finite differences as
		\begin{equation*}
			\Delta_{ h }^{ + } v ( t , x ) \coloneqq \frac{ v ( t , x + h v ) - v ( t , x ) }{ h },
			\quad
			\Delta_{ h }^{ - } v ( t , x ) \coloneqq \frac{ v ( t , x - h v ) - v ( t , x ) }{ h }
		\end{equation*}
		for some $ h > 0 $ and $ v \in \mathbb{ R }^{ d } $. Thus plugging $ \Delta_{ h }^{ - } \Delta_{ h }^{ + } u $ into (\ref{ac_weak_equation}) yields by the transformation formula that
		\begin{equation*}
			0
			=
			\int_{ 0 }^{ T }
			\int
			\frac{ 1 }{ \varepsilon^{ 2 } }
			\inner*{ \Delta_{ h }^{ + } \nabla W ( u ) }{ \Delta_{ h }^{ + } u }
			+
			\inner*{ \nabla \Delta_{ h }^{ + } u }{ \nabla \Delta_{ h }^{ + } u }
			+
			\inner*{ \partial_{ t } \Delta_{ h }^{ + } u }{ \Delta_{ h }^{ + } u }
			\dd{ x }
			\dd{ t },
		\end{equation*}
		which is equivalent to
		\begin{align*}
			& \int_{ 0 }^{ T }
			\int
			\abs{ \Delta_{ h }^{ + } \nabla u }^{ 2 }
			\dd{ x }
			\dd{ t }
			\\
			={} &
			-
			\int_{ 0 }^{ T }
			\int
			\partial_{ t } \left( \frac{ \abs{ \Delta_{ h }^{ + } u }^{ 2 } }{ 2 } \right)
			+
			\frac{ 1 }{ \varepsilon^{ 2 } }
			\inner*{ \Delta_{ h }^{ + } \nabla W ( u ) }{ \Delta_{ h }^{ + } u }
			\dd{ x }
			\dd{ t }
			\\
			={} &
			\int  
			\frac{ \abs{ \Delta_{ h }^{ + } u ( 0 ) }^{ 2 } - \abs{ \Delta_{ h }^{ + } u ( T ) }^{ 2 } }{ 2 }
			\dd{ x }
			\\
			& -
			\int_{ 0 }^{ 1 }
			\int_{ 0 }^{ T }
			\int
			\inner*{ 
				\diff^{ 2 } W \left( ( 1 - s ) u ( t, x  ) + s u( t,  x + h v ) ) \right) 
				\Delta_{ h }^{ + } u
			}{
				\Delta_{ h }^{ + } u 
			}
			\dd{ x }
			\dd{ t }
			\dd{ s }.
		\end{align*}
		The first summand can be estimated by $ \norm{ u }_{ \lp^{ \infty } ( [0 , T ] , \wkp^{ 1 , 2 } ( \flattorus ; \mathbb{ R }^{ N } ) ) } $. For the second summand, we partition $ W $ into the sum of $ W_{ \mathrm{conv} } $ and $ W_{ \mathrm{pert} } $. The term involving the convex summand can then by estimated by
		\begin{equation*}
			\int_{ 0 }^{ 1 }
			\int_{ 0 }^{ T }
			\int
			\inner*{ 
				\diff^{ 2 } W_{ \mathrm{conv} } \left( ( 1- s ) u ( t, x  ) + s u( t,  x + h v ) \right) 
				\Delta_{ h }^{ + } u
			}{
				\Delta_{ h }^{ + } u 
			}
			\dd{ x }
			\dd{ t }
			\dd{ s }
			\geq 0 
		\end{equation*}
		and the for the pertubation term, we get via the bound on its second derivative that
		\begin{align*}
			& \abs{
				\int_{ 0 }^{ 1 }
				\int_{ 0 }^{ T }
				\int
				\inner*{ 
					\diff^{ 2 } W_{ \mathrm{pert} } \left( (1 - s ) u ( t, x  ) + s  u( t,  x + h v )  \right) 
					\Delta_{ h }^{ + } u
				}{
					\Delta_{ h }^{ + } u 
				}
				\dd{ x }
				\dd{ t }
				\dd{ s }
			}
			\\
			\lesssim {} &
			\int_{ 0 }^{ T }
			\int
			\abs{ \nabla u }^{ 2 }
			\dd{ x }
			\dd{ t },
		\end{align*}
		which is also finite. 
		Combining these estimates, we obtain that $ \int_{ 0 }^{ T } \int \abs{ \Delta_{ h }^{ + } \nabla u }^{ 2 } \dd{ x } \dd{ t } $ is uniformly bounded in $ h $. Applying our calculation to all directions $ v \in \mathbb{ R }^{ d } $, we get by the finite-differences theorem for all $ 1 \leq i, j \leq d $ that $ \partial_{ i , j }^{ 2 } u \in \lp^{ 2 } ( [ 0 , T ] \times \flattorus ; \mathbb{ R }^{ N } ) $.
		
		In order to obtain $ \nabla W ( u ) \in \lp^{ 2 } ( [ 0 , T ] \times \flattorus ; \mathbb{ R }^{ N } ) $, we again consider the weak formulation (\ref{ac_weak_equation}) and notice that since we have already shown that both the time derivative and second space derivatives of $ u $ are square-integrable, our claim follows from a duality argument. 
	\end{description}
\end{proof}

\begin{remark}
	The inequality (\ref{discrete_optimal_energy_dissipation}) with the factor $ 1/2h $ instead of $ 1/h- C/2\varepsilon^{ 2 } $ follows immediately from the definition of our optimization problem, but is not optimal for fixed $ \varepsilon $ if we want to study the behaviour as $ h $ tends to zero. Moreover this so called \emph{sharp energy dissipation inequality} is important for later.
\end{remark}

\begin{remark}
	The energy dissipation inequality (\ref{energy_dissipation_sharp}) can be deduced via the formal calculation 
	\begin{align*}
		\dv{ t } \energy_{ \varepsilon } ( u )
		& =
		\int
		\frac{ 1 }{\varepsilon }
		\inner*{ \nabla W ( u ) }{ \partial_{ t } u }
		+
		\varepsilon
		\inner*{ \nabla u  }{ \nabla \partial_{ t } u }
		\dd{ x }
		\\
		& = 
		\int
		\inner*{ \frac{ 1 }{ \varepsilon } \nabla W ( u ) - \varepsilon \Delta u }{ \partial_{ t } u }
		\dd{ x }
		\\
		& =
		- \varepsilon \int \abs{ \partial_{ t } u_{ \varepsilon } }^{ 2 } \dd{ x }.
	\end{align*}
	In order to make this calculation rigorous, we however need to show that $ t \mapsto \energy_{ \varepsilon } ( u ( t ) ) $ is absolutely continuous, which is non-trivial, but it would give us equality in energy dissipation inequality (\ref{energy_dissipation_sharp}).
	Since we will only need the inequality, our proof will however suffice.
\end{remark}