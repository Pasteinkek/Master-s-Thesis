\chapter{Convergence of the Allen--Cahn equations}

The main content of this chapter is to look at the behaviour of solutions of the Allen--Cahn equation (\ref{allen_cahn_eq}) as the parameter $ \varepsilon $ tends to zero. As it turns out the scalar case is significantly easier to handle than the vectorial case, thus we shall first focus on the case $ N = 1 $,
also called the \emph{twophase case}.

\section{The twophase case}

When want to consider the behaviour of solutions $ u_{ \varepsilon} \colon [0, T] \times \flattorus \to \mathbb{ R } $ to (\ref{allen_cahn_eq}) for $ \varepsilon \to 0 $. Let us for now assume that the energies of the initial functions $ \energy_{ \varepsilon } ( u_{ \varepsilon }^{ 0 } ) $ stay uniformly bounded as $ \varepsilon $ tends to zero. 
Then due to the energy dissipation inequality (\ref{energy_dissipation_sharp}), we already obtain that for all
$ 0 \leq t \leq T $, we have that $ \energy_{ \varepsilon } \left( u_{\varepsilon } ( t ) \right) $ stays uniformly  bounded.

Another important observation for the convergence is the classic Modica Mortula trick (hier Referenz einfügen): let $ \alpha < \beta $ be the two distinct zeros of the doublewell potential $ W $.
Then we define a primitive of $ \sqrt{ 2 W ( u ) } $ via
\begin{equation*}
	\phi ( u ) 
	\coloneqq
	\int_{ \alpha }^{ u }
		\sqrt{ 2 W ( s ) }
	\dd{ s }.
\end{equation*}
Thus Young's inequality tells us that for $ \psi_{ \varepsilon } \coloneqq \phi \circ u_{ \varepsilon } $, we have
\begin{align}
	\energy_{ \varepsilon } ( u_{ \varepsilon } )
	& =
	\int
		\frac{ 1 }{ \varepsilon }
		W ( u_{ \varepsilon } ) 
		+
		\frac{ \varepsilon }{ 2 }
		\abs{ \nabla u_{ \varepsilon } }^{ 2 }
	\dd{ x }
	\notag
	\\
	& \geq
	\int
		\sqrt{ 2 W ( u_{ \varepsilon } ) }
		\abs{ \nabla u_{ \varepsilon } }
	\dd{ x }
	\notag
	\\
	& =
	\int
		\abs{ \nabla \left( \phi \circ u_{ \varepsilon } \right) }
	\dd{ x }
	\label{modica_mortula_trick},
\end{align}
which suggests that we might hope for good compactness properties of $ \phi \circ u_{ \varepsilon } $.
We combine these two observations into the following Proposition.

\begin{proposition}
	Given initial data $ u_{ \varepsilon }^{ 0 } $ whose energies stay uniformly bounded in the sense that
	\begin{equation}
		\limsup_{ \ \varepsilon \to 0 }
			\energy_{ \varepsilon } ( u_{ \varepsilon }^{ 0 } ) 
		< 
		\infty,
	\end{equation}
	there exists for any sequence $ \varepsilon \to 0 $ some non-relabelled subsequence such that the solutions of the Allen--Cahn equation (\ref{allen_cahn_eq}) with initial condition $ u_{ \varepsilon }^{ 0 } $ converge in $ \lp^{ 1 } ( ( 0, T ) \times \flattorus ) $ to some $ u = \alpha ( 1 - \chi ) + \beta \chi $ with $ \chi \in \bv \left( ( 0 , T ) \times \flattorus ; \{ 0 , 1 \} \right) $. Moreover the compositions $ \psi_{ \varepsilon } $ are uniformly bounded in $ \bv \left( ( 0, T ) \times \flattorus \right) $ and converge to $ \phi \circ u $ in $ \lp^{ 1 } ( ( 0 , T ) \times \flattorus ) $.
\end{proposition}

\begin{proof}
	From the energy dissipation inequality (\ref{energy_dissipation_sharp}) in \Cref{existence_of_ac_solution} we infer that for all $ \varepsilon > 0 $, it holds that
	\begin{equation*}
		\sup_{ 0 \leq t \leq T }
			\energy_{ \varepsilon } ( u_{ \varepsilon } ( t ) ) 
		\leq
		\energy_{ \varepsilon } ( u_{ \varepsilon} ),
	\end{equation*}
	which is by assumption uniformly bounded as $ \varepsilon $ tends to zero.
	We want to use a similar calculation as (\ref{modica_mortula_trick}) we thus infer that $ \nabla \psi_{ \varepsilon } $ is uniformly bounded in $ \lp^{ 1 } \left( ( 0 , T ) \times \flattorus \right) $.
	Moreover we may estimate
	\begin{align*}
		\int_{ 0 }^{ T }
			\int
				\abs{ \psi_{ \varepsilon } }
			\dd{ x }
		\dd{ t }
		& =
		\int_{ 0 }^{ T }
			\int
				\abs{
					\int_{ \alpha }^{ u_{ \varepsilon } }
						\sqrt{ 2 W ( s ) }
				}
					\dd{ s }
			\dd{ x }
		\dd{ t }
		\\
		& \leq
		\int_{ 0 }^{ T }
			\int
				\abs{ u_{ \varepsilon } - \alpha }
				\sup_{ s \in [ \alpha, u_{ \varepsilon } }
					\sqrt{ 2 W ( s ) }
			\dd{ x }
		\dd{ t }
		\\
		& \lesssim
		1 + 
		\int_{ 0 }^{ T }
			\int
				\abs{ u_{ \varepsilon } }^{ 1 + p/ 2 }
		\tag{\ref{polynomial_growth}}
		\\
		& \lesssim
		1 + 
		\int_{ 0 }^{ T }
			\int
				W ( u_{ \varepsilon } )
			\dd{ x }
		\dd{ t },
		\tag{\ref{polynomial_growth}}
	\end{align*}
	which is uniformly bounded via the energy bound.
\end{proof}
