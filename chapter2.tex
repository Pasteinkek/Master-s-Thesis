\chapter{Convergence of the Allen--Cahn equations}

Baldo proved in his paper \cite{baldo_minimal_interface_criterion} that the Cahn-Hilliard energies $ \Gamma $-converge with respect to $ \norm{ \cdot }_{\lp^{ 1 }} $ to an \emph{optimal partition energy} given by
\begin{equation}
	\label{partition_energy}
	\energy ( \chi ) 
	\coloneqq
	\frac{ 1 }{ 2 }
	\sum_{ 1 \leq i, j \leq P }
	\sigma_{ i , j }
	\int
	\frac{ 1 }{ 2 }
	\left(
	\abs{ \nabla \chi_{ i } }
	+
	\abs{ \nabla \chi_{ j } }
	-
	\abs{ \nabla ( \chi_{ i } + \chi_{ j } ) }
	\right)
\end{equation}
for a partition $ \chi_{ 1 }, \dotsc, \chi_{ P } \colon \mathbb{ T } \to \{ 0, 1 \} $ satisfying $ \chi_{ 1 \leq i \leq P } \chi_{ i } = 1 $ almost everywhere. We may also define measurable sets $ \Omega_{ i } $ through the relation $ \chi_{ i } = \mathds{1}_{ \Omega_{ i } }  $. The link between a sequence $ u_{ \varepsilon} $ and $ \chi $ is given by $ u_{ \varepsilon} \to u \coloneqq \sum_{ 1 \leq j \leq P } \alpha_{ i } \chi_{ i } $ in $ \lp^{ 1 } $. 

Moreover if we denote by $ \rb \Omega_{ i } $ the reduced boundary of $ \Omega_{ i } $ and by $ \Sigma_{ i , j } \coloneqq \rb \Omega_{ i } \cap \Omega_{ j } $ the interface between $ \Omega_{ i } $ and $ \Omega_{ j } $, then we may rewrite equation (\ref{partition_energy}) as 
\begin{equation*}
	\energy ( \chi ) 
	=
	\frac{ 1 }{ 2 }
	\sum_{1 \leq i, j \leq P }
	\sigma_{ i , j } \hd^{ d - 1 } ( \Sigma_{ i , j } ).
\end{equation*}

Here, the surface tensions $ \sigma_{ i , j } $ are the geodesic distances between the wells $ \alpha_{ i } $ of $ W $ with respect to the metric $ 2W(u) \inner{\cdot}{\cdot} $, which can be written out as
\begin{equation*}
	\sigma_{ i , j } = \mathrm{ d }_{ W } ( \alpha_{ i } , \alpha_{ j } )
\end{equation*}
for the geodesic distance defined as
\begin{equation}
	\mathrm{ d }_{ W } ( u, v )
	\coloneqq
	\inf
	\left\{
	\int_{ 0 }^{ 1 }
	\sqrt{ 2 W ( \gamma ) }
	\abs{ \dot{ \gamma }  }
	\dd{t}
	\,
	\colon
	\, \gamma \in \mathrm{ C }^{ 1 }( [0, 1 ], \mathbb{ R }^{ N } ) \text{ with } \gamma( 0 ) = u,\, \gamma( 1 )= v 
	\right\}.
\end{equation}

Geometrically speaking, the partition energy $ \energy $ measures the surface tensions between the sets and penalizes larges interfaces. Also observe that the factor $ 1/2 $ can be left out if we only count each interface once.