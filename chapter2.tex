\chapter{Convergence of the Allen--Cahn equations}

The main content of this chapter is to look at the behaviour of solutions of the Allen--Cahn equation (\ref{allen_cahn_eq}) as the parameter $ \varepsilon $ tends to zero. As it turns out the scalar case is significantly easier to handle than the vectorial case, thus we shall first focus on the case $ N = 1 $,
also called the \emph{twophase case}.

\section{The twophase case}

When want to consider the behaviour of solutions $ u_{ \varepsilon} \colon [0, T] \times \flattorus \to \mathbb{ R } $ to (\ref{allen_cahn_eq}) for $ \varepsilon \to 0 $. Let us for now assume that the energies of the initial functions $ \energy_{ \varepsilon } ( u_{ \varepsilon }^{ 0 } ) $ stay uniformly bounded as $ \varepsilon $ tends to zero. 
Then due to the energy dissipation inequality (\ref{energy_dissipation_sharp}), we already obtain that for all
$ 0 \leq t \leq T $, we have that $ \energy_{ \varepsilon } \left( u_{\varepsilon } ( t ) \right) $ stays uniformly  bounded.

Another important observation for the convergence is the classic Modica Mortula trick (hier Referenz einfügen): let $ \alpha < \beta $ be the two distinct zeros of the doublewell potential $ W $.
Then we define a primitive of $ \sqrt{ 2 W ( u ) } $ via
\begin{equation*}
	\phi ( u ) 
	\coloneqq
	\int_{ \alpha }^{ u }
		\sqrt{ 2 W ( s ) }
	\dd{ s }.
\end{equation*}
Thus Young's inequality tells us that
\begin{align*}
	\energy_{ \varepsilon } ( u_{ \varepsilon } )
	& =
	\int
		\frac{ 1 }{ \varepsilon }
		W ( u_{ \varepsilon } ) 
		+
		\frac{ \varepsilon }{ 2 }
		\abs{ \nabla u_{ \varepsilon } }^{ 2 }
	\dd{ x }
	\\
	& \geq
	\int
		\sqrt{ 2 W ( u_{ \varepsilon } ) }
		\abs{ \nabla u_{ \varepsilon } }
	\dd{ x }
	\\
	& =
	\int
		\abs{ \nabla \left( \phi \circ u_{ \varepsilon } \right) }
	\dd{ x },
\end{align*}
which suggests that we might hope for good compactness properties of $ \phi \circ u_{ \varepsilon } $.
We combine these two observations into the following Proposition.

\begin{proposition}
	Given initial data $ u_{ \varepsilon }^{ 0 } $ 
\end{proposition}
