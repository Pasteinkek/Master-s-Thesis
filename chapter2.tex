\chapter{Convergence of the Allen--Cahn equations}

The main content of this chapter is to look at the behaviour of solutions of the Allen--Cahn equation (\ref{allen_cahn_eq}) as the parameter $ \varepsilon $ tends to zero. As it turns out the scalar case is significantly easier to handle than the vectorial case, thus we shall first focus on the case $ N = 1 $,
also called the \emph{twophase case}.

\section{Convergence in the twophase case}

When want to consider the behaviour of solutions $ u_{ \varepsilon} \colon [0, T] \times \flattorus \to \mathbb{ R } $ to (\ref{allen_cahn_eq}) for $ \varepsilon \to 0 $. Let us for now assume that the energies of the initial functions $ \energy_{ \varepsilon } ( u_{ \varepsilon }^{ 0 } ) $ stay uniformly bounded as $ \varepsilon $ tends to zero. 
Then due to the energy dissipation inequality (\ref{energy_dissipation_sharp}), we already obtain that for all
$ 0 \leq t \leq T $, we have that $ \energy_{ \varepsilon } \left( u_{\varepsilon } ( t ) \right) $ stays uniformly  bounded.

Another important observation for the convergence is the classic Modica Mortula trick (hier Referenz einfügen): let $ \alpha < \beta $ be the two distinct zeros of the doublewell potential $ W $.
Then we define a primitive of $ \sqrt{ 2 W ( u ) } $ via
\begin{equation*}
	\phi ( u ) 
	\coloneqq
	\int_{ \alpha }^{ u }
		\sqrt{ 2 W ( s ) }
	\dd{ s }.
\end{equation*}
For $ \psi_{ \varepsilon } \coloneqq \phi \circ u_{ \varepsilon } $, we can show that $ \psi_{ \varepsilon } \in \wkp^{ 1, 1 } ( ( 0 , T ) \times \flattorus ) $ with weak derivatives $ \nabla \psi_{ \varepsilon } = \sqrt{ 2 W ( u_{ \varepsilon } ) } \nabla u_{ \varepsilon } $ and $ \partial_{ t } \psi_{ \varepsilon } = \sqrt{ 2 W ( u_{ \varepsilon } ) } \partial_{ t } u_{ \varepsilon } $. (we will later show this in more generality (Referenz einfügen)).

Thus via Young's inequality, we can compute
\begin{align}
	\energy_{ \varepsilon } ( u_{ \varepsilon } )
	& =
	\int
		\frac{ 1 }{ \varepsilon }
		W ( u_{ \varepsilon } ) 
		+
		\frac{ \varepsilon }{ 2 }
		\abs{ \nabla u_{ \varepsilon } }^{ 2 }
	\dd{ x }
	\notag
	\\
	& \geq
	\int
		\sqrt{ 2 W ( u_{ \varepsilon } ) }
		\abs{ \nabla u_{ \varepsilon } }
	\dd{ x }
	\notag
	\\
	& =
	\int
		\abs{ \nabla \psi_{ \varepsilon} }
	\dd{ x }
	\label{modica_mortula_trick},
\end{align}
which suggests that we might hope for good compactness properties of $ \phi \circ u_{ \varepsilon } $.
We combine these two observations into the following Proposition.

\begin{proposition}
	\label{initial_convergence_result}
	Given initial data $ u_{ \varepsilon }^{ 0 } $ whose energies stay uniformly bounded in the sense that
	\begin{equation}
		\sup_{ \ \varepsilon > 0 }
			\energy_{ \varepsilon } ( u_{ \varepsilon }^{ 0 } ) 
		< 
		\infty,
	\end{equation}
	there exists for any sequence $ \varepsilon \to 0 $ some non-relabelled subsequence such that the solutions of the Allen--Cahn equation (\ref{allen_cahn_eq}) with initial condition $ u_{ \varepsilon }^{ 0 } $ converge in $ \lp^{ 1 } ( ( 0, T ) \times \flattorus ) $ to some $ u = \alpha ( 1 - \chi ) + \beta \chi $ with $ \chi \in \bv \left( ( 0 , T ) \times \flattorus ; \{ 0 , 1 \} \right) $. Moreover the compositions $ \psi_{ \varepsilon } $ are uniformly bounded in $ \bv \left( ( 0, T ) \times \flattorus \right) $ and converge to $ \phi \circ u $ in $ \lp^{ 1 } ( ( 0 , T ) \times \flattorus ) $.
\end{proposition}

\begin{proof}
	From the energy dissipation inequality (\ref{energy_dissipation_sharp}) in \Cref{existence_of_ac_solution} we infer that for all $ \varepsilon > 0 $, it holds that
	\begin{equation}
		\label{unif_bound_on_energies}
		\sup_{ 0 \leq t \leq T }
			\energy_{ \varepsilon } ( u_{ \varepsilon } ( t ) ) 
		\leq
		\energy_{ \varepsilon } ( u_{ \varepsilon}^{ 0 } ),
	\end{equation}
	whose right hand side is by assumption uniformly bounded in $ \varepsilon $.
	We want to use a similar calculation as (\ref{modica_mortula_trick}) we thus infer that $ \nabla \psi_{ \varepsilon } $ is uniformly bounded in $ \lp^{ 1 } \left( ( 0 , T ) \times \flattorus \right) $.
	Moreover we may estimate
	\begin{align}
		\int_{ 0 }^{ T }
			\int
				\abs{ \psi_{ \varepsilon } }
			\dd{ x }
		\dd{ t }
		& =
		\int_{ 0 }^{ T }
			\int
				\abs{
					\int_{ \alpha }^{ u_{ \varepsilon } }
						\sqrt{ 2 W ( s ) }
				}
					\dd{ s }
			\dd{ x }
		\dd{ t }
		\notag
		\\
		& \leq
		\int_{ 0 }^{ T }
			\int
				\abs{ u_{ \varepsilon } - \alpha }
				\sup_{ s \in [ \alpha, u_{ \varepsilon } ] }
					\sqrt{ 2 W ( s ) }
			\dd{ x }
		\dd{ t }
		\notag
		\\
		& \lesssim
		1 + 
		\int_{ 0 }^{ T }
			\int
				\abs{ u_{ \varepsilon } }^{ 1 + p/ 2 }
			\dd{ x }
		\dd{ t }
		\notag
		\\
		& \lesssim
		1 + 
		\int_{ 0 }^{ T }
			\int
				W ( u_{ \varepsilon } )
			\dd{ x }
		\dd{ t },
		\label{l1_estimate_for_psi_epsilon}
	\end{align}
	which is uniformly bounded via the energy bound. Here the last two inequalities follow from the upper respectively lower growth assumptions (\ref{polynomial_growth}) on $ W $. Thus we have indeed that $ \psi_{ \varepsilon } $ is a bounded sequence in $ \bv( ( 0 , T ) \times \flattorus ) $ and therefore there exists some non-relabelled subsequence and some $ \psi \in \bv ( ( 0 , T ) \times \flattorus ) $ such that $ \psi_{ \varepsilon } $ converges to $ \psi $ in $ \lp^{ 1 } ( ( 0 , T ) \times \flattorus ) $.
	
	We notice that since $ W $ is non-negative and only has a discrete set of zeros, the function $ \phi $ is strictly increasing and continuous on $ \mathbb{ R } $, and is thus invertible. Moreover we may pass to a further non-relabelled subsequence of $ \psi_{ \varepsilon } $ which converges almost everywhere to $ \psi $. 
	Thus defining $ u \coloneqq \phi^{ - 1 } ( \psi ) $, we obtain that
	\begin{equation*}
		u_{ \varepsilon } = \phi^{ - 1 } ( \psi_{ \varepsilon } ) \to \phi^{ - 1 } ( \psi ) = u
	\end{equation*}
	converges pointwise almost everywhere.
	
	Moreover we notice that by Fatou's Lemma and the boundedness of the energies that
	\begin{equation*}
		\int
			W ( u ) 
		\dd{ x }
		\leq
		\liminf_{ \varepsilon \to 0 }
			\int
				W ( u_{ \varepsilon } )
			\dd{ x }
		\leq
		\liminf_{ \varepsilon \to 0 }
			\varepsilon \energy_{ \varepsilon } ( u_{ \varepsilon } )
		=
		0.
	\end{equation*}
	Again using the non-negativity of $ W $, this yields that $ W ( u ) = 0 $ almost everywhere. 
	Thus $ u \in \{ \alpha , \beta \} $ almost everywhere and we may write $ u = \alpha ( 1 - \chi ) + \beta \chi $ for some $ \chi \colon ( 0 , T ) \times \flattorus \to \{ 0 , 1 \} $.
	Looking again at the definition of $ u $, we moreover obtain that 
	\begin{equation}
		\label{represenation_of_psi}
		\psi = \phi ( u ) = 
		\phi ( \alpha ) ( 1 - \chi )
		+ 
	 \phi ( \beta )	\chi
		= 
		\int_{ \alpha }^{ \beta } \sqrt{ 2 W ( s ) } \dd{ s } \chi 
		\eqqcolon
		\sigma \chi 
	\end{equation}
	and since $ \psi $ is a function of bounded variation, this implies that $ \chi $ is of bounded variation as well.
	
	Finally from the energy bound and the estimate $ \abs{ u_{ \varepsilon } } \lesssim 1 + W ( u ) $, we infer that $ u_{ \varepsilon } $ is $ \lp^{ p }$-bounded, and since $ u_{ \varepsilon } $ converges pointwise almost everywhere to $ u $, we obtain the desired $ \lp^{ 1 } $-convergence.
\end{proof}

Nextup, we want to make sure that $ u $ respectively $ \chi $ assume their initial data and on the way obtain a useful bound on the time derivative.

\begin{lemma}
	With the assumptions of \Cref{initial_convergence_result}, we have $ \psi_{ \varepsilon } \in \wkp^{ 1 , 2 } ( [ 0 , T ] ; \lp^{ 1 } ( \mathbb{ T } ) ) $ with the estimate
	\begin{equation}
		\label{bound_on_time_dv_of_psi_epsilon}
		\left(
			\int_{ 0 }^{ T }
				\left(
					\int
						\abs{ \partial_{ t } \psi_{ \varepsilon } }
					\dd{ x }
				\right)^{ 2 }
			\dd{ t }
		\right)^{ 1/2 }
		\lesssim
		\energy_{ \varepsilon } ( u_{ \varepsilon }^{ 0 } ).
	\end{equation}
	Furthermore the sequence $ u_{ \varepsilon } $ is precompact in $ \cont \left( [ 0 , T ] ; \lp^{ 2 } ( \flattorus ) \right) $. 
\end{lemma}

\begin{proof}
	\begin{description}[wide=0pt]
		\item[Step 1:] $ \psi_{ \varepsilon } \in \wkp^{ 1 , 2 } \left( [ 0 , T ] ; \lp^{ 1 } ( \flattorus ) \right) $ and satisfies the inequality (\ref{bound_on_time_dv_of_psi_epsilon})
		
		From estimate (\ref{l1_estimate_for_psi_epsilon}) we can infer that
		\begin{equation*}
			\int_{ 0 }^{ T }
				\left(
					\int
						\abs{ \psi_{ \varepsilon } }
					\dd{ x }
				\right)^{ 2 }
			\lesssim
			\int_{ 0 }^{ T }
				\left(
					1 + \int W ( u_{ \varepsilon } ) \dd{ x }
				\right)^{ 2 }
			\dd{ t } 
			<
			\infty
		\end{equation*}
		via the uniform boundedness of the energies (\ref{unif_bound_on_energies}). For the desired bound (\ref{bound_on_time_dv_of_psi_epsilon}), we estimate via Hölder's inequality and the uniform boundedness of the energies that
		\begin{align*}
			\int_{ 0 }^{ T }
				\left(
					\int
						\abs{ 
							\partial_{ t } \psi_{ \varepsilon }
						}
					\dd{ x }
				\right)^{ 2 }
			\dd{ t }
			& \leq
			\int_{ 0 }^{ T }
				\left(
					\int
						\sqrt{ 2 W ( u_{ \varepsilon } ) }
						\abs{ \partial_{ t } u_{ \varepsilon } }
					\dd{ x }
				\right)^{ 2 }
			\dd{ t }
			\\
			& =
			\int_{ 0 }^{ T }
				\left(
					\int
						\frac{ 1 }{ \sqrt{\varepsilon } } \sqrt{ 2 W ( u_{ \varepsilon } ) }
						\sqrt{ \varepsilon } \abs{ \partial_{ t } u_{ \varepsilon } }
					\dd{ x }
				\right)^{ 2 }
			\dd{ t }
			\\
			& \leq
			\int_{ 0 }^{ T }
				\int
					\frac{ 1 }{ \varepsilon }
					2 W ( u_{ \varepsilon } )
				\dd{ x }
				\int
					\varepsilon
					\abs{ \partial_{ t } u_{ \varepsilon } }^{ 2 }
				\dd{ x }
			\dd{ t }
			\\
			& \leq
			2 \energy_{ \varepsilon } ( u_{ \varepsilon }^{ 0 } )
			\int_{ 0 }^{ T }
				\int
					\varepsilon
					\abs{ \partial_{ t } u_{ \varepsilon } }^{ 2 }
				\dd{ x }
			\dd{ t }
			\tag{\ref{unif_bound_on_energies}}
			\\
			& \leq
			2 \left(\energy_{ \varepsilon } ( u_{ \varepsilon }^{ 0 } )\right)^{ 2 }
			\tag{\ref{energy_dissipation_sharp}},
		\end{align*}
		which completes step 1.
		
		\item[Step 2:] The sequence $ \psi_{ \varepsilon } $ is precompact in $ \cont \left( [ 0 , T ] ; \lp^{ 1 } ( \flattorus ) \right) $.
		
		As noted in embedding (\ref{w12_embeds_into_c1half}), we have
		\begin{equation*}
			\wkp^{ 1 , 2 } \left( [ 0 , T ] ; \lp^{ 2 } ( \flattorus ; \mathbb{ R }^{ N } ) \right)
			\hookrightarrow
			\cont^{ 1 / 2 } \left( [ 0 , T ] ; \lp^{ 2 } ( \flattorus ; \mathbb{ R }^{ N } ) \right)
		\end{equation*}
		from which the equicontinuity of the sequence follows with step 1.
		Moreover for fixed time $ t $, estimate (\ref{modica_mortula_trick}) yields that $ \nabla \psi_{ \varepsilon } ( t ) $ is bounded in $ \lp^{ 1 } ( \flattorus ) $, and combined estimate (\ref{l1_estimate_for_psi_epsilon}) without integrating in time yields that $ \psi_{ \varepsilon } ( t ) $ is a bounded sequence in $ \wkp^{ 1 , 1 } ( \flattorus ) $ and thus precompact in $ \lp^{ 1 } ( \flattorus ) $.
		Thus the Arzelà--Ascoli Theorem yields the desired precompactness.
		
		\item[Step 3:] The sequence $ u_{ \varepsilon } $ converges to $ u $ in measure uniformly in time.
		
		From step 2, we infer that we already have 
		\begin{equation*}
			\lim_{ \varepsilon \to 0 }
				\esssup_{ 0 \leq t \leq T }
					\int	
						\abs{ \psi_{ \varepsilon } ( t , x ) - \phi \circ u ( t , x ) }
					\dd{ x }
			= 0.
		\end{equation*}
		It especially follows in combination with equation (\ref{represenation_of_psi}) that
		\begin{align*}
			& \lim_{ \varepsilon \to 0 }
				\esssup_{ 0 \leq t \leq T }
					\int
						(1 - \chi)
						\abs{ \psi_{ \varepsilon } ( t , x ) }
					\dd{ x }
			=
			\lim_{ \varepsilon \to 0 }
				\esssup_{ 0 \leq t \leq T }
					\int
						(1 - \chi)
						\abs{ \psi_{ \varepsilon } ( t , x ) - \phi \circ u ( t , x ) }
					\dd{ x }
			=
			0	
			\shortintertext{and}
			& \lim_{ \varepsilon \to 0 }
				\esssup_{ 0 \leq t \leq T }
					\int
						\chi
							\abs{ \psi_{ \varepsilon } ( t , x ) - \sigma }
					\dd{ x }
			= 
			\lim_{ \varepsilon \to 0 }	
				\esssup_{ 0 \leq t \leq T }
					\int
						\chi
						\abs{ \psi_{ \varepsilon } ( t , x ) - \phi \circ u ( t , x ) }
					\dd{ x }
			= 0		
		\end{align*}
		Moreover we have by continuity and the $p$-growth of $ W $ (\ref{polynomial_growth}) that for a given $ \delta < \beta - \alpha $,
		\begin{equation*}
			\min \{ \abs{ \phi ( t ) } \, \colon \, \abs{ t - \alpha } > \delta / 2 \text{ or } \abs{ t - \beta } > \delta / 2 \}
			=
			\rho
			> 
			0.
		\end{equation*}
		Therefore we may estimate
		\begin{align*}
			& \esssup_{ 0 \leq t \leq T }
				\lm^{ d } \left( \left\{ x \in \flattorus \, \colon \, \abs{ u_{ \varepsilon } ( t, x ) - u ( t, x ) } > \delta \right\} \right)
			\\
			\leq {} &
			\esssup_{ 0 \leq t \leq T }
				\lm^{ d } \left(
					\left\{  ( 1- \chi ( t, x ) ) \abs{ u_{ \varepsilon } ( t , x ) - \alpha } > \delta / 2 \right\}
				\right)
				+
				\lm^{ d } \left( \left\{ \chi ( t , x ) \abs{ u_{ \varepsilon } ( t , x ) - \beta } > \delta / 2 \right\} \right)
			\\
			\leq {} &
			\esssup_{ 0 \leq t \leq T }
				\lm^{ d } \left(
					\left\{ ( 1 - \chi ( t, x ) ) \abs{ \psi_{ \varepsilon } ( t , x ) } > \rho \right\} 
				\right)
				+
				\lm^{ d } \left(
					\left\{ \chi ( t , x )  \abs{ \psi_{ \varepsilon } ( t , x ) - \sigma } > \rho \right\} 
				\right)
			\\
			\leq {} &
			\esssup_{ 0 \leq t \leq T }
				\frac{ 1 }{ \rho }
				\int
					( 1 - \chi ( t , x ) ) \abs{ \psi_{ \varepsilon } ( t , x ) } 
				\dd{ x }
				+
				\frac{ 1 }{ \rho }
				\int
					\chi ( t , x ) \abs{ \psi_{ \varepsilon } ( t , x ) - \sigma }
				\dd{ x },
		\end{align*}
		which goes to zero as $ \varepsilon $ goes to zero, proving our claim.
		
		\item[Step 4:] $ u_{ \varepsilon }^{ 2 } $ is equiintegrable uniformly in time
		
		We have 
		\begin{equation*}
			0 \leq u_{ \varepsilon }^{ 2 } \lesssim 1 + W ( u_{ \varepsilon } )
		\end{equation*}
		by the growth bounds (\ref{polynomial_growth}) on $ W $. Since $ W ( u_{ \varepsilon } ) $ converges to 0 in $ \lp^{ 1 } ( \flattorus ) $ uniformly in time by the energy bound (\ref{unif_bound_on_energies}), it is equiintegrable uniformly in time. Thus $ u_{ \varepsilon }^{ 2 } $ is equiintegrable uniformly in time as well.
		
		\item[Step 5:] $ u_{ \varepsilon } $ converges in $ \cont \left( [ 0 , T ] ; \lp^{ 2 } ( \flattorus ) \right) $.
		
		We somewhat repeat the proof the convergence in measure and integrability imply $ \lp^{ 1 } $-convergence, while keeping the uniformity in time.
		
		Take some $ \delta > 0 $. Then we decompose the integral
		\begin{equation*}
			\int \abs{ u_{ \varepsilon } - u }^{ 2 } \dd{ x }
			=
			\int_{ \{ \abs{ u_{ \varepsilon } - u } \geq \delta \} }
				\abs{ u_{\varepsilon } - u }^{ 2 }
			\dd{ x }
			+
			\int_{ \{ \abs{ u_{ \varepsilon } - u } < \delta \} }
				\abs{ u_{ \varepsilon } - u }^{ 2 }
			\dd{ x }.
		\end{equation*}
		For the first summand, we notice that
		\begin{equation*}
			\sup_{ 0 \leq t \leq T }
				\int_{ \{ \abs{ u_{ \varepsilon } - u } \geq \delta \} }
					\abs{ u_{\varepsilon } - u }^{ 2 }
				\dd{ x }
			\lesssim
			\sup_{ 0 \leq t \leq T }
				\int_{ \{ \abs{ u_{ \varepsilon } - u } \geq \delta \} }
					1 + u_{\varepsilon }^{ 2 }
				\dd{ x }
			\to 0
		\end{equation*}
		as $ \varepsilon \to 0 $ since $ \lm^{ d }\left( \abs{ u_{\varepsilon } - u } \geq \delta \right) \to 0 $ uniformly in time by step 3 and $ u_{\varepsilon }^{ 2 } + 1 $ is equiintegrable uniformly in time by step 4.
		
		For the second summand, we simply estimate
		\begin{equation*}
			\sup_{ 0 \leq t \leq T }
				\int_{ \{ \abs{ u_{ \varepsilon } - u } < \delta \} }
					\abs{ u_{ \varepsilon } - u }^{ 2 }
				\dd{ x }
			=
			\delta^{ 2 } \Lambda^{ d }
		\end{equation*}
		Taking the limes superior as $ \varepsilon $ tends to zero of this inequality yields that the right hand side can be made arbitrarily small, which yields
		\begin{equation*}
			\sup_{ 0 \leq t \leq T }
				\int
					\abs{ u_{ \varepsilon } - u }^{ 2 }
				\dd{ x }
			\to 
			0
		\end{equation*}
		as $ \varepsilon $ tends to zero.
	\end{description}
\end{proof}

\begin{remark}
	From the previous Lemma, it follows that if the initial conditions $ u_{ \varepsilon }^{ 0 } $ converge in $ \lp^{ 1 } $ or pointwise almost everywhere to the function $ \alpha ( 1 - \chi^{ 0 } ) + \beta \chi^{ 0 } $ (and we know that the limit is of this form since the energies of the initial values stay bounded), then $ u $ also assumes this initial condition in $ \lp^{ 2 } ( \flattorus ) $.
\end{remark}

\section{Convergence of the energies}

As often in the Calculus of Variations, convergence of energies boost our modulus of convergence and gives our limit therefore additional regularity. 
Thus let us assume 
\begin{equation}
	\label{energy_convergence}
	\int_{ 0 }^{ T }
		\energy_{ \varepsilon } ( u_{ \varepsilon } ) 
	\dd{ t }
	\to 
	\int_{ 0 }^{ T }
		\energy ( u )
	\dd{ t }
	\text{ as }
	\varepsilon \to 0,
\end{equation}
where the surface tension energy is defined for $ u = \alpha ( 1 - \chi ) + \beta \chi $ by 
\begin{equation}
	\energy ( u ) 
	\coloneqq
	\sigma \int \abs{ \nabla \chi }.
\end{equation}
Moreover we notice that the energy of $ u $ is exactly the total variation of $ \psi = \sigma \chi $.
Since for almost every time $ t $, we have that $ \psi_{ \varepsilon } ( t ) $ converges to $ \psi ( t ) $ in $ \lp^{ 1 }  ( \flattorus ) $ by \Cref{initial_convergence_result}, it follows from the lower semicontinuity of the variation measure and Young's inequality that
\begin{equation*}
	\energy ( u ) 
	\leq
	\liminf_{ \varepsilon \to 0 }
	\int
		\abs{ \nabla \psi_{ \varepsilon } }
	\dd{ x }
	\leq
	\liminf_{ \varepsilon \to 0 }
		\energy_{ \varepsilon } ( u_{ \varepsilon } ).
\end{equation*}
Moreover by the energy dissipation inequality (\ref{energy_dissipation_sharp}), the energies stay uniformly bounded in time. Thus using the dominated convergence theorem, we see that our time integrated energy convergence assumption is equivalent to saying that for almost every time $ t \in ( 0, T ) $, we have convergence of the energies $ \energy_{ \varepsilon } ( u_{ \varepsilon } ( t ) ) \to \energy ( u ( t ) ) $.

Since the energies themselves can be interpreted as measures on the flat torus, we define for a continuous function $ \varphi \in \cont ( \flattorus ) $ the corresponding energy measures by
\begin{align*}
	& \energy_{ \varepsilon } ( u_{ \varepsilon} ; \varphi )
	\coloneqq
	\int
		\varphi
		\left(
			\frac{ 1 }{ \varepsilon }
			+
			\frac{ \varepsilon }{ 2 }
			\abs{ \nabla u_{ \varepsilon } }^{ 2 }
		\right)
	\dd{x } 
	\text{ and }
	\\
	& \energy ( u ; \varphi )
	\coloneqq
	\sigma
	\int
		\varphi
		\abs{ \nabla \chi }.
\end{align*}
Then from the energy convergence and lower semicontinuity just discussed, it follows that 
\begin{equation}
	\label{convergence_of_energy_measures}
	\lim_{ \varepsilon \to 0 }
		\energy_{ \varepsilon } ( u_{ \varepsilon } ; \varphi )
	=
	\energy ( u ; \varphi ).
\end{equation}

A first additional regularity result under the energy convergence assumption is the following Proposition, which ensures that we have square-integrable normal velocities.

\begin{proposition}
	In the setting of \Cref{initial_convergence_result} and given the energy convergence assumption (\ref{energy_convergence}), the measure $ \partial_{ t } \chi $ is absolutely continuous with respect to the measure $ \abs{ \nabla \chi } \dd{ t } $ and the corresponding density $ V $ is square integrable with the estimate
	\begin{equation*}
		\int_{ 0 }^{ T }
			\int
				V^{ 2 }
			\abs{ \nabla \chi }
		\dd{ t }
		\lesssim
		\energy^{ 0 },
	\end{equation*}
	where $ \energy^{ 0 } \coloneqq \liminf_{ \varepsilon \to 0 } \energy_{ \varepsilon } ( u_{ \varepsilon }^{ 0 } ) $.
\end{proposition}

\begin{proof}
	Take a smooth test function $ \varphi \in \cont_{ \mathrm{ C } }^{ \infty } ( ( 0 , T ) \times \flattorus )$. 
	Then via the $ \lp^{ 1 } $-convergence of $ \psi_{ \varepsilon } \to \psi $, we have
	\begin{align*}
		\partial_{ t } \psi ( \varphi )
		& =
		\liminf_{ \varepsilon \to 0 }
			\partial_{ t } \psi_{ \varepsilon } ( \varphi )
		\\
		& =
		\liminf_{ \varepsilon \to 0 }
			\int_{ 0 }^{ T }
				\int
					\sqrt{ 2 W ( u_{ \varepsilon } ) }
					\partial_{ t } u_{ \varepsilon }
					\varphi
				\dd{ x }
			\dd{ t }
		\\
		& \leq
		\liminf_{ \varepsilon \to 0 }
		\left( 
			\int_{ 0 }^{ T }
				\int
					\frac{ 1 }{ \varepsilon } 2 W ( u_{ \varepsilon } )
					\varphi^{ 2 }
				\dd{ x }
			\dd{ t }
		\right)^{ 1/ 2 }
		\left(
			\int_{ 0 }^{ T }
				\int
					\varepsilon
					\abs{ \partial_{ t } u_{ \varepsilon } }^{ 2 }
				\dd{ x }
			\dd{ t }
		\right)^{ 1/2 }
		\\
		& \leq
		\liminf_{ \varepsilon \to 0 }
			\left(
				2 
				\int_{ 0 }^{ T }
					\energy_{ \varepsilon} \left( u_{ \varepsilon } ; \varphi^{ 2 } \right)
				\dd{ t }
			\right)^{ 1/ 2 }
			\left(
				\energy_{ \varepsilon } ( u_{ \varepsilon } )
			\right)^{ 1/2 }
		\tag{\ref{energy_dissipation_sharp}}
		\\
		& =
		\sqrt{ 2 \sigma }
		\norm{ \varphi }_{ \lp^{ 2 } ( \flattorus , \abs{ \nabla \chi } \dd{ t } ) }
		\sqrt{ \energy^{ 0 } }.
		\tag{\ref{convergence_of_energy_measures}}
	\end{align*}
	This proves both the absolute continuity and via a duality argument the desired bound since $ \partial_{ t } \psi = \sigma \partial_{ t } \chi $ and $ \sigma > 0 $.
\end{proof}

We finish this section with a proof for the equipartition of the energy, which tells us that both the summand involving the potential $ W ( u_{ \varepsilon } $ and the norm of the gradient contribute to the energy in similar parts.

\begin{lemma}
	\label{equipartition_of_energies}
	Under the energy convergence assumption (\ref{energy_convergence}), we have for any continuous function $ \varphi \in \cont^{ \infty } ( \flattorus ) $ that
	\begin{align*}
		\energy ( u ; \varphi )
		=
		\lim_{ \varepsilon \to 0 }
			\energy_{ \varepsilon } ( u_{ \varepsilon } ; \varphi )
		& = 
		\lim_{ \varepsilon \to 0 }
			\int
				\varphi
				\sqrt{ 2 W ( u_{ \varepsilon } ) }
				\abs{ \nabla u_{ \varepsilon } }
			\dd{ x }
		\\
		& =
		\lim_{ \varepsilon \to 0 }
			\int
				\varphi
				\varepsilon
				\abs{ \nabla u_{ \varepsilon } }^{ 2 }
			\dd{ x}
		\\
		& =
		\lim_{ \varepsilon \to 0 }
			\int
				\varphi
				\frac{ 1 }{ \varepsilon }
				2 W ( u_{ \varepsilon } )
			\dd{ x }
	\end{align*}
	for almost every time $ 0 \leq t \leq T $.
\end{lemma}

\begin{proof}
	We have already established the first equality before. For the second equality, we first assume that $ \varphi \in \cont \left( \flattorus ; [ 0 , \infty ) \right)$.
	By the lower semicontinuity of the variation measure, we immediately obtain
	\begin{equation*}
		\liminf_{ \varepsilon \to 0 }
			\int
				\varphi
				\sqrt{ 2 W ( u_{ \varepsilon } }
				\abs{ \nabla u_{ \varepsilon } }
			\dd{ x }
		\geq
		\energy ( u ; \varphi ).
	\end{equation*}
	But by Young's inequality, we also have 
	\begin{equation*}
		\limsup_{ \varepsilon \to 0 }
			\int
				\varphi
				\sqrt{ 2 W ( u_{ \varepsilon } ) }
				\abs{ \nabla u_{ \varepsilon } }
			\dd{ x }
		\leq
		\limsup_{ \varepsilon \to 0 }
			\energy_{ \varepsilon } ( u_{ \varepsilon } ; \varphi )
		= \energy ( u ; \varphi ).
	\end{equation*}
	For general $ \varphi \in \cont \left( \flattorus \right) $, we decompose $ \varphi $ into its positive and negative part and apply the previous argument to both in order to get the claim.
	
	The third and fourth inequality follow for a given non-negative $ \varphi \in \cont \left( \flattorus ; [ 0 , \infty ) \right) $ by the $ \lp^{ 2 } $ estimate
	\begin{align*}
		& \lim_{ \varepsilon \to 0 }
			\int
				\abs{ \sqrt{ \varphi } \sqrt{ \varepsilon } \abs{ \nabla u_{ \varepsilon } } - \sqrt{ \varphi } \frac{ 1 }{ \sqrt{ \varepsilon } } \sqrt{ 2 W ( u_{ \varepsilon } ) } }^{ 2 }
			\dd{ x }
		\\
		={} &
		\lim_{ \varepsilon \to 0 }
			2 \energy_{ \varepsilon } ( u_{ \varepsilon } ; \varphi )
			-
			2 \int	
				\varphi
				\sqrt{ 2 W ( u_{ \varepsilon } ) }
				\abs{ \nabla u_{ \varepsilon } }
			\dd{ x }
		=
		0
	\end{align*}
	which implies
	\begin{equation*}
		\lim_{ \varepsilon \to 0 }
			\int
				\varphi
				\varepsilon
				\abs{ \nabla u_{ \varepsilon } }^{ 2 }
			\dd{ x }
		=
		\lim_{ \varepsilon \to 0}
			\int
				\varphi
				\frac{ 1 }{ \varepsilon }
				2 W ( u_{ \varepsilon } )
			\dd{ x }
		=
		\lim_{ \varepsilon \to 0 }
		\energy_{ \varepsilon } ( u_{ \varepsilon } ; \varphi ),
	\end{equation*}
	finishing our proof.
\end{proof}

\section{Convergence to twophase mean curvature flow}

We start by definining a BV-formulation for motion by mean curvature.

\begin{definition}
	\label{motion_by_mcv}
	Fix some finite time horizon $ T < \infty $ and initial data $ \chi^{ 0 }\in \bv \left( \flattorus ; \{ 0 , 1 \} \right) $, we say that 
	\begin{equation*}
		\chi \in 
		\cont \left(
			[ 0 , T ] ; \lp^{ 2 } \left( \flattorus ; \{ 0 , 1 \}  \right)
		\right)
	\end{equation*}
	with $ \esssup_{ 0 \leq t \leq T } \energy ( \chi ) $ \emph{moves by mean curvature} if there is a normal velocity
	$ V \in \lp^{ 2 } \left( \abs{ \nabla \chi } \dd{ t } \right) $ such that 
	\begin{enumerate}
		\item 
		For all 
		$ \xi \in \cont_{ \mathrm{ C } }^{ \infty } \left( ( 0 , T ) \times \flattorus ; \mathbb{ R }^{ d } \right) $,
		we have
		\begin{equation}
			\label{integral_formulation_of_mcf}
			\int_{ 0 }^{ T }
				\int
					V \inner*{ \xi }{ \nu }
					- 
					\inner*{ \diff \xi }{ \mathrm{ Id } - \nu \otimes \nu }
				\abs{ \nabla \chi }
			\dd{ t }
			=
			0.
		\end{equation}
	\item 
	The function $ V $ is the normal velocity of $ \chi $ in the sense that 
	\begin{equation*}
		\partial \chi
		=
		V
		\abs{ \nabla \chi }
		\dd{ t }
	\end{equation*}
	holds distributionally in $ ( 0 , T ) \times \flattorus $.
	\item 
	The initial data $ \chi^{ 0 } $ is achieved in $ \cont \left( [ 0 , T ] ; \lp^{  2 } ( \flattorus ) \right) $, which simply means that $ \chi ( 0 ) = \chi^{ 0 } $ as functions in $ \lp^{ 2 } ( \flattorus ) $.
	\end{enumerate}
\end{definition}

Our main goal in this section is now to show that the function $ \chi $ we have found in \Cref{initial_convergence_result} moves my mean curvature. Looking at equation (\ref{ac_weak_equation}) which reads
\begin{equation*}
	\int
	\frac{ 1 }{ \varepsilon^{ 2 } } W' ( u_{ \varepsilon} ( t ) ) \varphi
	+
	\nabla u_{ \varepsilon } ( t ) \nabla \varphi
	+
	\partial_{ t } u_{ \varepsilon} ( t ) \varphi 
	\dd{x}
	=
	0,
	\tag{\ref{ac_weak_equation}}
\end{equation*}
we expect that for a suitable choice of testfunctions $ \varphi_{ \varepsilon } $, the following two terms converge:
\begin{align*}
	& \lim_{ \varepsilon \to 0 }
		\int_{ 0 }^{ T }
			\int 
			\partial_{ t } u_{ \varepsilon } \varphi_{ \varepsilon }
	=
	\sigma
	\int_{ 0 }^{ T }
		\int
			V \inner*{ \xi }{ \nu }
		\abs{ \nabla \chi }
	\dd{ t },
	\\
	& \lim_{ \varepsilon \to 0 }
		\int_{ 0 }^{ T }
		\int
			\frac{1 }{ \varepsilon^{ 2 } }
			W'( u_{ \varepsilon } )
			\varphi_{ \varepsilon}
			+ 
			\inner*{ \nabla u_{ \varepsilon } }{ \nabla \varphi_{ \varepsilon } }
		\dd{ x }
		\dd{ t }
	=
	\sigma
	- \int_{ 0 }^{ T }
		\int
			\inner*{ \diff \xi }{ \mathrm{Id} - \nu \otimes \nu }
		\abs{ \nabla \chi }
	\dd{ t }.
\end{align*}
But how do we find these testfunctions? For this, we first note that the curvature term
$ \int \inner*{ \diff \xi }{ \mathrm{Id} - \nu \otimes \nu } \abs{ \nabla \chi } \dd{ t } $ is by \cite[Thm.~17.5]{maggi_sets_of_finite_perimeter} the first inner variation with respect to $ \xi $ of the perimeter functional, which is just our energy $ \energy $ up to the surface tension constant $ \sigma > 0 $. Thus it is plausible to compute the first inner variation 
$ \left.\dv{ s } \right|_{ s = 0 } E ( \rho_{ s } ) $ and then we can hopefully choose the testfunction $ \varphi_{ \varepsilon } $ in such a way that it equals
$ 
\int
\frac{ 1 }{ \varepsilon^{ 2 } } W' ( u_{ \varepsilon} ( t ) ) \varphi
+
\nabla u_{ \varepsilon } ( t ) \nabla \varphi
\dd{x}
$.

Thus let $ ( \rho_{ s } )_{ s } $ be functions which solve the ODE
\[
	\begin{cases}
	\partial_{ t } \rho_{ s } 
	+
	\inner*{ \xi }{ \nabla \rho_{ s } }
	& = 0
	\\
	\rho_{ 0 } & = u_{ \varepsilon }.
	\end{cases}
\]
Then we formally compute 
\begin{align*}
	\left.\dv{ s }\right|_{ s = 0 }
		\int
			\frac{ \varepsilon }{ 2 }
			\abs{ \rho_{ s } }^{ 2 }
			+
			\frac{ 1 }{ \varepsilon }
			W ( \rho_{ s } )
		\dd{ x }
	& =
	\int
		\varepsilon 
		\inner*{ \nabla u_{ \varepsilon } }
		{ \nabla \left( - \inner*{ \xi }{ \nabla u_{ \varepsilon } } \right) }
		+
		\frac{ 1 }{ \varepsilon }
		W'( u_{ \varepsilon } ) ( -\inner*{ \xi }{ \nabla u_{ \varepsilon } } )
	\dd{ x }
	\\
	& =
	\int
		\left(
			\varepsilon \Delta u - \frac{ 1 }{ \varepsilon } W'( u_{ \varepsilon } )
		\right)
		\inner*{ \xi }{ \nabla u_{ \varepsilon } } 
	\dd{ x }.
\end{align*}
We therefore test equation (\ref{ac_weak_equation}) against $ \varphi_{ \varepsilon } \coloneqq \inner{ \xi }{ \nabla u_{ \varepsilon } } $.

\subsection{Convergence of the curvature term}

The goal of this section is to prove the convergence
\begin{equation*}
	\lim_{ \varepsilon \to 0 }
		\int
			\left(
				\varepsilon \Delta u_{ \varepsilon }
				- 
				\frac{ 1 }{ \varepsilon }
				W'( u_{ \varepsilon } )
			\right)
			\inner*{ \xi }{ \nabla u_{ \varepsilon } }
		\dd{ x }
	=
	\sigma
	\int
		\inner*{ \diff \xi }{ \mathrm{Id} - \nu \otimes \nu }
	\abs{ \nabla \chi }
\end{equation*} 
for almost every time $ t $.