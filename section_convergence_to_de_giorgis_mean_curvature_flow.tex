\section{Conditional convergence to De Giorgi's mean curvature flow}

In this section, we shall state our solution concept and prove convergence to 
the aforementioned.

\begin{definition}[De Giorgi type $ \bv $-solution to multiphase mean 
curvature 
flow]
	\label{de_giorgi_solution_to_mmcf}
	Fix some finite time horizon $ T < \infty $, a $ (P \times P) $-matrix of 
	surface tensions $ \sigma $ and initial data $ \chi^{ 0 } \colon \flattorus 
	\to \{ 0 , 1 \}^{ P } $ with $ \energy_{ 0 } \coloneqq \energy ( \chi^{ 0 } 
	) < \infty $ and 
	$ \sum_{ i = 1 }^{ P } \chi_{ i }^{ 0 } = 1 $ almost everywhere. We say that
	\begin{equation*}
		\chi \in \cont \left(
		[ 0 , T ]
		;
		\lp^{ 2 } \left( \flattorus ; \{ 0 , 1 \}^{ P } \right)
		\right)
		\cap 
		\bv \left(
			( 0 , T ) \times \flattorus ; \{ 0 , 1 \}^{ P } 
		\right)
	\end{equation*}
	with $ \esssup_{ 0 \leq t \leq T } \energy ( \chi ) < \infty  $ and $ 
	\sum_{ i = 1 
	}^{ P } \chi_{ i } = \sum_{ i = 1 }^{ P } \mathds{ 1 }_{ \Omega_{ i } } = 
	1  $ almost everywhere is a \emph{De Giorgi type }$\bv$\emph{-solution to 
	multiphase mean 
	curvature flow with initial data} $ \chi^{ 0 } $ \emph{and surface 
	tensions} $ \sigma $ if the following holds. 
	\begin{enumerate}
		\item 
		For all $ 1 \leq i \leq P $, there exists a normal 
		velocity $ V_{ i } \in \lp^{ 2 } ( \abs{ \nabla \chi_{ i } } \dd{ t } 
		) $ 
		such that
		\begin{equation*}
			\partial_{ t } \chi_{ i }
			=
			V_{ i } \abs{ \nabla \chi_{ i } } \dd{ t }
		\end{equation*}
		holds in the distributional sense on $ ( 0 , T ) \times \flattorus $.
		
		\item 
		There exist a mean curvature vector $ H \in 
		\lp^{ 2 } ( \energy ( u ; \cdot) \dd{ t } ; \mathbb{ R }^{ d } ) $ 
		which satisfies
		\begin{align}
			\notag
			& 
			\sum_{ 1 \leq i < j \leq P }
			\sigma_{ i j }
			\int_{ 0 }^{ T }
				\int_{ \Sigma_{ i j } }
				\inner*{ H }{ \xi }
				\dd{ \hm^{ d - 1 } }
			\dd{ t }
			\\
			\label{mean_curvature_vector_bv_de_giorgi}
			={} &
			-
			\sum_{ 1 \leq i < j \leq P }
			\sigma_{ i j }
			\int_{ 0 }^{ T }
				\int_{ \Sigma_{ i j } }
					\inner*{
						\diff \xi }
					{ \mathrm{Id} - \nu_{ i } \otimes \nu_{ i } }
				\dd{ \hm^{ d - 1 } }
			\dd{ t }
		\end{align}
		for all test vector fields 
		$ \xi \in \cont_{ \mathrm{c} }^{ \infty } \left(
			( 0 , T ) \times \flattorus ; \mathbb{ R }^{ d }
		\right) $,
		where $ \nu_{ i } \coloneqq \nabla \chi_{ i } / \abs{ \nabla \chi_{ i } 
		} $ are the inner unit normals and $ \Sigma_{ i j } \coloneqq 
		\partial_{ \ast } \Omega_{ i } \cap \partial_{ \ast } \Omega_{ j } $
		is the $ (i, j )$-th interface.
		
		\item 
		The partition $ \chi $ satisfies a De Giorgi type optimal energy 
		dissipation inequality in the sense that for almost every time $ 0 < T' 
		< T $, we have
		\begin{equation}
			\label{optimal_energy_dissipation_solution}
			\energy ( \chi ( T' ) )
			+
			\frac{ 1 }{ 2 }
			\sum_{ 1 \leq i < j \leq P }
				\sigma_{ i j }
				\int_{ 0 }^{ T' }
					\int_{ \Sigma_{ i j } }
						V_{ i }^{ 2 }
						+
						\abs{ H }^{ 2 }
					\dd{ \hm^{ d - 1 } }
				\dd{ t }
			\leq
			\energy_{ 0 }.
		\end{equation}
		
		\item
		The initial data is attained in $ \cont \left( [ 0 , T ] ; 
		\lp^{ 2 } ( \flattorus ) \right) $.
	\end{enumerate}
\end{definition}

We can now immediately prove a convergence result similar to
\Cref{convergence_to_multiphase_mcf}.
\begin{theorem}
	\label{convergence_to_de_giorgis_multiphase_mcf}
	Let $ W \colon \mathbb{ R }^{ N } \to [ 0, 
	\infty ) $ be a smooth multiwell potential satisfying the assumptions 
	(\ref{polynomial_growth})-(\ref{perturbation bound}). Let $ T < \infty 
	$ be an arbitrary finite time horizon. Let 
	$ u_{ \varepsilon }^{ 0 } \colon \flattorus \to \mathbb{ R }^{ N } $ be a 
	sequence of initial data
	approximating a partition 
	$ \chi^{ 0 } \in \bv \left( \flattorus ; \{ 0 , 1 \}^{ P } \right) $ 
	in the sense that 
	$ u_{ \varepsilon }^{ 0 } \to u^{ 0 } =  \sum_{ 1 \leq i \leq P } 
	\chi_{ i }^{ 0 } \alpha_{ i } $ 
	holds pointwise almost everywhere and 
	\begin{equation*} 
		\energy_{ 0 } 
		\coloneqq 
		\energy ( \chi^{ 0 } ) 
		= 
		\lim_{ \varepsilon \to 0 } 
		\energy_{ \varepsilon } ( u_{ \varepsilon }^{ 0 } ) 
		< 
		\infty.
	\end{equation*}
	Then we have for 
	some subsequence of solutions $ u_{\varepsilon } $ to the Allen--Cahn 
	equation
	(\ref{allen_cahn_eq})  with initial datum $ u_{ 
		\varepsilon }^{ 0 } $ that there exists a time-dependent partition $ 
		\chi $ 
	with 
	\begin{equation*}
	\chi \in \bv \left( ( 0 , T ) \times \flattorus ; \{ 0 , 1 \}^{ P } 
	\right) 
	\end{equation*} and
	$ \chi 
	\in \cont \left( [ 0 , T ] ; \lp^{ 2 } \left( \flattorus ;  \{ 0 , 1 
	\}^{ P } \right) \right) $ such that $ u_{ \varepsilon } $ converges to 
	$ u \coloneqq \sum_{ 1 \leq i \leq P } \chi_{ i } \alpha_{ i } $ almost 
	everyhwere. Moreover $ u $ assumes the initial data $ u^{ 0 } $ in $ 
	\cont \left( [ 0, T ] ; \lp^{ 2 }( \flattorus ) \right) $. If we 
	additionally assume that the 
	time-integrated energies converge (\ref{energy_convergence}), then $ 
	\chi $ is a De Giorgi type $ \bv $-solution to multiphase mean curvature 
	flow in the sense 
	of 
	\Cref{de_giorgi_solution_to_mmcf}.
\end{theorem} 

This result is similar to \Cref{convergence_to_multiphase_mcf} and we only need 
to prove that $ \chi $ is a De Giorgi type $ \bv $-solution to mean curvature 
flow.

\begin{proof}
	The idea of the proof is that we already have an optimal energy dissipation 
	inequality for the Allen--Cahn equation given by 
	(\ref{energy_dissipation_sharp}).
	If we additionally use the Allen--Cahn equation once, we arrive at 
	the De Giorgi type optimal energy dissipation inequality given 
	by
	\begin{equation*}
	\label{energy_dissipation_sharp_with_curvature}
	\energy_{ \varepsilon } ( u_{ \varepsilon } ( T' ) )
	+
	\frac{ 1 }{ 2 }
	\int_{ 0 }^{ T' }
		\int
			\varepsilon \abs{ \partial_{ t } u_{ \varepsilon } }^{ 2 }
			+
			\frac{ 1 }{ \varepsilon }
			\abs{
				\varepsilon \Delta u_{ \varepsilon } 
				-
				\frac{ 1 }{ \varepsilon } \nabla W ( u_{ \varepsilon } )
			}^{ 2 }
		\dd{ x }
	\dd{ t }
	\leq
	\energy_{ \varepsilon } ( u_{ \varepsilon}^{ 0 } ).
	\end{equation*}
	Our hope is that as $ \varepsilon $ tends to zero, we can pass to the 
	optimal energy dissipation inequality 
	(\ref{optimal_energy_dissipation_solution}) for $ \chi $ through lower 
	semicontinuity. 
	Since we assume the convergence of the initial energies and energy 
	convergence for almost every time, the only terms we have to care about are
	the velocity and curvature term. 
	The lower semicontinuity of the velocity term reads
	\begin{equation}
		\label{lsc_of_velocity}
		\liminf_{ \varepsilon \to 0 }
			\frac{ 1 }{ 2 }
			\int_{ 0 }^{ T' }
				\int
					\varepsilon 
					\abs{ \partial_{ t } u_{ \varepsilon } }^{ 2 }
				\dd{ x }
			\dd{ t }
		\geq
		\frac{ 1 }{ 2 }
		\sum_{ i < j  }
			\sigma_{ i j }
			\int_{ 0 }^{ T' }
				\int_{ \Sigma_{ i j } }
					V_{ i }^{ 2 }
				\dd{ \hm^{ d - 1 } }
			\dd{ t }
	\end{equation}
	and the lower semicontinuity of the curvature term is given by
	\begin{equation}
		\label{lsc_of_curvature}
		\liminf_{ \varepsilon \to 0 }
			\frac{ 1 }{ 2 }
			\int_{ 0 }^{ T' }
				\int
					\frac{ 1 }{ \varepsilon }
					\abs{
						\varepsilon
						\Delta u_{ \varepsilon }
						-
						\frac{ 1 }{ \varepsilon }
						\nabla W ( u_{ \varepsilon } ) 
					}^{ 2 }
				\dd{ x }
			\dd{ t }
		\geq
		\frac{ 1 }{ 2 }
		\sum_{ i < j  }
			\sigma_{ i j }
			\int_{ 0 }^{ T' }
				\int_{ \Sigma_{ i j } }
					\abs{ H }^{ 2 }
				\dd{ \hm^{ d - 1 } }
			\dd{ t }.
	\end{equation}
	Moreover we have to show the existence of the mean curvature vector $ H $.
	We could cheat in this step and simply use that by 
	\Cref{convergence_to_multiphase_mcf}, we already know that the tangential 
	divergence applied to $ \xi $ is given by the velocity. In other words, 
	that we already have $ V_{ i } \nu_{ i } = -H $ on $ \Sigma_{ i j } $. But 
	we want 
	to present a more direct approach.
	
	Consider the linear functional 
	\begin{equation*}
		L ( \xi )
		\coloneqq
		- \sum_{ 1 \leq i < j \leq P }
			\sigma_{ i j }
			\int_{ 0 }^{ T }
				\int_{ \Sigma_{ i j } }
					\inner*{ \diff \xi }{ \mathrm{Id} - \nu_{ i } \otimes \nu_{ 
					i } }
				\dd{ \hm^{ d - 1 } }
			\dd{ t }
	\end{equation*}
	defined on test vector fields $ \xi $. Then $ L $ is bounded with respect 
	to the $ \lp^{ 2 } $-norm on $ ( 0 , T ) \times \flattorus $ equipped with 
	$ \energy ( u ; \cdot ) \dd{ t } $ since by the convergence of the 
	curvature 
	term observed in \Cref{convergence_of_curvature_multiphase}, we have
	\begin{align*}
		\abs{ L ( \xi ) }
		& =
		\liminf_{ \varepsilon \to 0 }
			\abs{ 
				-
				\int_{ 0 }^{ T }
					\int
						\inner*{
							\varepsilon \Delta u_{ \varepsilon } 
							-
							\frac{ 1 }{ \varepsilon }
							\nabla W ( u_{ \varepsilon } )
						}
						{ \diff u_{ \varepsilon } \xi }
					\dd{ x }
				\dd{ t }
			}
		\\
		& \leq
		\left(
			\int_{ 0 }^{ T }
				\int
					\frac{ 1 }{ \varepsilon }
					\abs{ 
						\varepsilon \Delta u_{ \varepsilon }
						-
						\frac{ 1 }{ \varepsilon }
						\nabla W ( u_{ \varepsilon } )
					}^{ 2 }
				\dd{ x }
			\dd{ t }
		\right)^{ 1/2 }
		\left(
			\int_{ 0 }^{ T }
				\int
					\varepsilon
					\abs{ \diff u_{ \varepsilon } \xi }^{ 2 }
				\dd{ x }
			\dd{ t }
		\right)^{ 1/2 }
		\\
		& \leq
				\left(
		\int_{ 0 }^{ T }
		\int
		 \varepsilon
		\abs{ 
		\partial_{ t } u_{ \varepsilon }
		}^{ 2 }
		\dd{ x }
		\dd{ t }
		\right)^{ 1/2 }
		\left(
		\int_{ 0 }^{ T }
		\int
		\varepsilon
		\abs{ \nabla u_{ \varepsilon } }^{ 2 }
		\abs{ \xi }^{ 2 }
		\dd{ x }
		\dd{ t }
		\right)^{ 1/2 }.
	\end{align*}
	The first factor stays uniformly bounded due to the energy dissipation 
	inequality (\ref{energy_dissipation_sharp}), and by the equipartition of 
	energies (\Cref{equipartition_of_energies_multiphase}), the second factor 
	converges to the $ \lp^{ 2 } $-norm of $ \xi $ with respect to the energy 
	measure, proving our claim. Therefore we can extend the functional to 
	the square integrable functions with respect to the energy measure. By 
	Riesz representation theorem we obtain the existence of the desired mean 
	curvature vector $ H $.
	
	Let us now consider the lower semicontinuity of the curvature 
	term. Let again $ \xi $ be some test vector field. Then for all $ 
	\varepsilon > 0 $ and some fixed time, we have by Young's inequality 
	\begin{align*}
		& 
		\liminf_{ \varepsilon \to 0 }
		\frac{ 1 }{ 2 }
		\int
			\frac{ 1 }{ \varepsilon }
			\abs{ 
				\varepsilon
				\Delta u_{ \varepsilon }
				-
				\frac{ 1 }{ \varepsilon }
				\nabla W ( u_{ \varepsilon } )
			}^{ 2 }
		\dd{ x }
		\\
		\geq{} &
		\liminf_{ \varepsilon \to 0 }
		\int
			\inner*{ 
				\varepsilon
				\Delta u_{ \varepsilon }
				-
				\frac{ 1 }{ \varepsilon }
				\nabla W ( u_{ \varepsilon } )
			}{
				\diff u_{ \varepsilon } \xi
			}
		\dd{ x }
		-
		\frac{ 1 }{ 2 }
		\int 
			\varepsilon
			\abs{ \diff u_{ \varepsilon } \xi }^{ 2 }
		\dd{ x }
		\\
		\geq{} &
		-\energy \left( \chi ; \inner*{ H }{ \xi } \right)
		-
		\frac{ 1 }{ 2 }
		\energy \left( \chi ; \abs{ \xi }^{ 2 } \right).
	\end{align*}
	Here the last inequality is due to the Cauchy--Schwarz inequality and uses 
	the convergence of the curvature term 
	\Cref{convergence_of_curvature_multiphase}.
	Since this inequality holds for any test vector field, we may take a 
	sequence of test vector fields satisfying
	\begin{equation*}
		\lim_{ n \to \infty }
			\norm{ \xi_{ n } + H }_{ \lp^{ 2 } \left( \flattorus , \energy 
			( 
			\chi ; \cdot ) ; \mathbb{ R }^{ d } \right) } 
		=
		0.
	\end{equation*}
	This then yields the desired inequality (\ref{lsc_of_curvature}) by 
	applying Fatou's Lemma.
	
	In principle the proof is now already done, since 
	\Cref{convergence_to_multiphase_mcf} already gives us that for almost every 
	time $ t $, we have $ V_{ i } \nu_{ i } = - H $ on $ \Sigma_{ i , j 
	} $ $ \hm^{ d-  1 } $-almost everywhere. But since this makes heavy use of 
	the previous arguments, we instead want to present another proof which 
	directly proves the lower semicontinuity of the velocity term and may leave 
	more room for future generalizations.
	
	Let us first consider the two-phase case $ N = 1 $ and $ P = 2 $. By a 
	similar duality argument as for the 
	lower semicontinuity of the curvature term, we compute 
	that for 
	every test function $ \varphi $, we have
	\begin{align*}
		& \liminf_{ \varepsilon \to 0 }
			\frac{ 1 }{ 2 }
			\int_{ 0 }^{ T }
				\int
					\varepsilon \abs{ \partial_{ t } u_{ \varepsilon } }^{ 2 }
				\dd{ x }
			\dd{ t }
		\\
		\geq{} &
		\liminf_{ \varepsilon \to 0 }
			\int_{ 0 }^{ T }
				\int
					\partial_{ t } u_{ \varepsilon } 
					\phi ' ( u_{ \varepsilon } )
					\varphi
				\dd{ x }
			\dd{ t }
		-
		\frac{ 1 }{ 2 }
		\int_{ 0 }^{ T }
			\int
				\frac{ 1 }{ \varepsilon }
				\left( \phi ' ( u_{ \varepsilon } ) \varphi \right)^{ 2 }
			\dd{ x }
		\dd{ t }
		\\
		={} &
		\liminf_{ \varepsilon \to 0 }
			\int_{ 0 }^{ T }
				\int
					\partial_{ t } \psi_{ \varepsilon }
					\varphi
				\dd{ x }
			\dd{ t }
			-
			\frac{ 1 }{ 2 }
			\int_{ 0 }^{ T }
				\int
					\frac{ 1 }{ \varepsilon }
					2 W ( u_{ \varepsilon } )
					\varphi^{ 2 }
				\dd{ x }
			\dd{ t }
		\\
		={} &
		\sigma
		\int_{ 0 }^{ T }
			\int_{ \Sigma }
				\varphi V
			\dd{ \hm^{ d - 1 } }
		\dd{ t }
		-
		\frac{ 1 }{ 2 }
		\sigma
		\int_{ 0 }^{ T }
			\int_{ \Sigma }
				\varphi^{ 2 }
			\dd{ \hm^{ d - 1 } }
		\dd{ t }.
	\end{align*}
	Here the last equality uses the weak convergence of $ \partial_{  t } 
	\psi_{ \varepsilon } $ to $ \partial_{  t } \psi = \sigma V \abs{ \nabla 
	\chi } \dd{ t } $ for the first summand 
	and the equipartition of energies (\Cref{equipartition_of_energies}) for 
	the second summand.
	Since the inequality holds for any test function $ \varphi $, we may 
	plug in a sequence of test functions $ \varphi_{ n } $ satisfying
	\begin{equation*}
		\lim_{ n \to \infty }
			\norm{ \varphi_{ n } - V }_{ \lp^{ 2 } \left( ( 0 , T ) \times 
			\flattorus , \hm^{ d - 1 }  \llcorner_{ \Sigma } \dd{ t } \right) }
		= 0
	\end{equation*}
	and thereby obtain the desired inequality (\ref{lsc_of_velocity}).
	
	For the multiphase case, we do not find an immediate generalization of this 
	proof, but rather have to work with the usual localization argument in 
	order to 
	obtain a reduction to the two-phase case.
	
	As in the proof of \Cref{convergence_to_multiphase_mcf}, let $ \delta > 0 $ 
	and $ 0 = T_{ 0 } < T_{ 1 } < \dotsc < T_{ K } = T' $ be a partition of $ [ 
	0 , T' ] $. Let $ ( g_{ k } )_{ k = 1 , \dotsc , K } $ be a partition of 
	unity with respect to the intervals $ \left( ( T_{ k - 1 } - \delta , T_{ k 
	} + \delta )\right)_{ 1 \leq k \leq K} $, let $ r > 0 $
	and $ \eta_{ B } $ as in 
	\Cref{localization_lemma_with_normals}. Moreover we fix some $ R > 0 $. 
	Then we estimate by Young's inequality
	\begin{align*}
		 A \coloneqq{}& \liminf_{ \varepsilon \to 0 }
			\frac{ 1 }{ 2 }
			\int_{ 0 }^{ T }
				\int
					\varepsilon
					\abs{ \partial_{ t } u_{ \varepsilon } }^{ 2 }
				\dd{ x }
			\dd{ t }
		\\
		={} &
		\liminf_{ \varepsilon \to 0 }
			\sum_{ k = 1 }^{ K }
				\sum_{ B \in \mathcal{ B }_{ r } } 
					\frac{ 1 }{ 2 }
					\int_{ 0 }^{ T }
						\int
							g_{ k } \eta_{ B }
							\varepsilon
							\abs{ \partial_{ t } u_{ \varepsilon } }^{ 2 }
						\dd{ x }
					\dd{ t }
		\\
		\geq{} &
		\liminf_{ \varepsilon \to 0 }
			\sum_{ k = 1 }^{ K }
				\sum_{ B \in \mathcal{ B }_{ r } }
					\max_{ 1 \leq i \leq P }
						\sup_{ 
							\substack{ 
								\varphi \in \cont_{ \mathrm{c} }^{ \infty } 
								\left( ( 0 , T ) \times \flattorus \right)
								\\
								\abs{ \varphi } \leq R  
							}
						}
							\int_{ 0 }^{ T }
								\int
									g_{ k } \eta_{ B }
									\inner*{ \nabla \phi_{ i } ( u_{ 
									\varepsilon 
									} ) }{ \partial_{ t } u_{ \varepsilon } }
									\varphi
								\dd{ x }
							\dd{ t } 
		\\
							& \qquad \qquad \qquad \qquad \qquad \qquad \qquad 
							\quad 
							 -
							\frac{ 1 }{ 2 }
							\int_{ 0 }^{ T }
								\int
									g_{ k } \eta_{ B }
									\frac{ 1 }{ \varepsilon }
									\abs{ \nabla \phi_{ i } ( u_{ \varepsilon } 
									) }^{ 2 }
									\varphi^{ 2 }
								\dd{ x }
							\dd{ t }.
	\end{align*}
	\begin{comment}
	\geq{} &
	\liminf_{ \varepsilon \to 0 }
	\sum_{ k = 1 }^{ K }
	\sum_{ B \in \mathcal{ B }_{ r } }
	\max_{ 1 \leq i \leq P }
	\sup_{ 
		\substack{ 
			\varphi \in \cont_{ \mathrm{c} }^{ \infty } 
			\left( ( 0 , T ) \times \flattorus \right)
			\\
			\abs{ \varphi } \leq R  
		}
	}
	\int_{ 0 }^{ T }
	\int
	g_{ k } \eta_{ B }
	\partial_{ t } \psi_{ \varepsilon , i }
	\varphi
	\dd{ x }
	\dd{ t } 
	\\
	& \qquad \qquad \qquad \qquad \qquad \qquad \quad 
	\:\, -
	\frac{ 1 }{ 2 }
	\int_{ 0 }^{ T }
	\int
	g_{ k } \eta_{ B }
	\frac{ 1 }{ \varepsilon }
	2 W ( u_{ \varepsilon } )
	\varphi^{ 2 }
	\dd{ x }
	\dd{ t }
\end{comment}
	We identify that via the chain rule, we have $ \inner*{ \nabla \phi_{ i } ( 
	u_{ \varepsilon } ) }{ \partial_{ t } u_{ \varepsilon } } = \partial_{ t } 
	\psi_{ \varepsilon , i } $ and moreover remember $ \abs{ \nabla \phi_{ i } 
	} \leq \sqrt{ 2 W } $.
	Pulling the limit inferior inside the double sum and the suprema, we thus 
	obtain 
	that this term can be estimated from below by
	\begin{equation*}\sum_{ k = 1 }^{ K }
		\sum_{ B \in \mathcal{ B }_{ r } }
		\max_{ 1 \leq i \leq P }
		\sup_{ 
			\substack{ 
				\varphi \in \cont_{ \mathrm{c} }^{ \infty } 
				\left( ( 0 , T ) \times \flattorus \right)
				\\
				\abs{ \varphi } \leq R  
			}
		}
		\int_{ 0 }^{ T }
		\int
		g_{ k } \eta_{ B }
		\varphi
		\partial_{ t } \psi_{ i }
		-
		\frac{ 1 }{ 2 }
		\int_{ 0 }^{ T }
		\energy \left( \chi ; g_{ k } \eta_{ B } \varphi^{ 2 } \right)
		\dd{ t }.
		\end{equation*}
		By adding zero, we get
		\begin{align*}
		& \int_{ 0 }^{ T }
		\int
		g_{ k } \eta_{ B }
		\partial_{ t } \psi_{ i }
		\varphi
		\dd{ x }
		\dd{ t } 
		-
		\frac{ 1 }{ 2 }
		\int_{ 0 }^{ T }
		\energy \left( \chi ; g_{ k } \eta_{ B } \varphi^{ 2 } \right)
		\dd{ t }
		\\
		={}&
		\sum_{ j = 1 }^{ P }
		\left(
		\sigma_{ i j }
		\int_{ 0 }^{ T }
		\int
		g_{ k } \eta_{ B }
		\varphi
		V_{ j }
		\abs{ \nabla \chi_{ j } }
		\dd{ t } 
		-
		\frac{ 1 }{ 2 }
		\sigma_{ i j }
		\int_{ 0 }^{ T }
			\int
				g_{ k } \eta_{ B }
				\varphi^{ 2 }
			\abs{ \nabla \chi_{ j } }
		\dd{ t }
		\right)\\
		& 
		-\frac{ 1 }{ 2 }
		\left(
		\int_{ 0 }^{ T }
		\energy \left( \chi ; g_{ k } \eta_{ B } \varphi^{ 2 } \right)
		-
		\int
			g_{ k } \eta_{ B }
			\varphi^{ 2 }
			\abs{ \nabla \psi_{ i } }
		\dd{ t }
		\right).
	\end{align*}
	Since no derivative has fallen on $ g_{ k } $, we may send $ \delta $ to 
	zero and 
	obtain by the dominated convergence theorem that we can replace $ g_{ k } $ 
	by $ \mathds{ 1 }_{ ( T_{ k - 1 } , T_{ k } ) } $. We thus end up with a 
	good summand consisting of
	\begin{equation}
		\label{the_good_summand}
		\sum_{ k = 1 }^{ K }
		\sum_{ B \in \mathcal{ B }_{ r } }
		\max_{ 1 \leq i \leq P }
		\sup_{ 
			\substack{ 
				\varphi \in \cont_{ \mathrm{c} }^{ \infty } 
				\left( ( 0 , T ) \times \flattorus \right)
				\\
				\abs{ \varphi } \leq R  
			}
		}
		\sum_{ j = 1 }^{ P }
		\sigma_{ i j }
		\int_{ T_{ k - 1 } }^{ T_{ k } } 
		\int\eta_{ B }
		\varphi 
		\left( V_{ j } - \frac{ 1 }{ 2 } \varphi \right)
		\abs{ \nabla \chi_{ j } }
		\dd{ t } 
	\end{equation}
	and an error summand given by
	\begin{align}
		\notag
		& \sum_{ k = 1 }^{ K }
			\sum_{ B \in \mathcal{ B }_{ r } }
				\max_{ 1 \leq i \leq P }
					\sup_{ \substack{ 
							\varphi \in \cont_{ \mathrm{c} }^{ \infty } 
							\left( ( 0 , T ) \times \flattorus \right)
							\\
							\abs{ \varphi } \leq R  
					} }
				- \frac{ 1 }{ 2 }
				\int_{ T_{ k - 1 } }^{ T_{ k } }
				\left(
				\energy \left( \chi ; \eta_{ B } \varphi^{ 2 } \right)
				-
				\int
				\eta_{ B }
				\varphi^{ 2 }
				\abs{ \nabla \psi_{ i } }
				\right)
				\dd{ t }
		\\
		\label{minor_error_term}
		\geq{} &
		\sum_{ k = 1 }^{ K }
		\sum_{ B \in \mathcal{ B }_{ r } }
		\max_{ 1 \leq i \leq P }
		- \frac{ R^{ 2 } }{ 2 }
		\int_{ T_{ k - 1 } }^{ T_{ k } }
		\left(
		\energy \left( \chi ; \eta_{ B } \right)
		-
		\int
		\eta_{ B }
		\abs{ \nabla \psi_{ i } }
		\right)
		\dd{ t }.
	\end{align}
	We choose a majority phase $ ( i , j ) $ and estimate the good 
	summand (\ref{the_good_summand}) from below by
	\begin{align}
		\notag
		& \max_{ 1 \leq i \leq P }
		\sup_{ \substack{ 
				\varphi \in \cont_{ \mathrm{c} }^{ \infty } 
				\left( ( 0 , T ) \times \flattorus \right)
				\\
				\abs{ \varphi } \leq R  
		} }
		\sum_{ j = 1 }^{ P }
			\sigma_{ i j }
			\int_{ T_{ k  - 1 } }^{ T_{ k } }
				\int
					\eta_{ B } \varphi
					\left( V_{ j } - \frac{ 1 }{ 2 } \varphi  \right)
				\abs{ \nabla \chi_{ j } }
			\dd{ t }
		\\
		\notag
		\geq{} &
		\max_{ i < j }
		\sup_{ \substack{ 
				\varphi \in \cont_{ \mathrm{c} }^{ \infty } 
				\left( ( 0 , T ) \times \flattorus \right)
				\\
				\abs{ \varphi } \leq R  
		} }
		\sigma_{ i j }
		\int_{ T_{ k  - 1 } }^{ T_{ k } }
		\int
		\eta_{ B } \varphi
		\left( V_{ j } - \frac{ 1 }{ 2 } \varphi  \right)
		\abs{ \nabla \chi_{ j } }
		\dd{ t }
		\\
		\label{major_error}
		& \qquad \qquad \qquad \qquad -
		C \sum_{ l \notin \{ i , j \} }
		\int_{ T_{ k - 1 } }^{ T_{ k } }
			\int
				\eta_{ B } 
				\left(
					R \abs{ V_{ l } }
					+
					R^{ 2 }
				\right)
			\abs{ \nabla \chi_{ l } }
		\dd{ t }.
	\end{align}
	Concerning the error term, we have by \Cref{localization_estimate_weaker} 
	that
	\begin{equation*}
				\frac{ R^{ 2 } }{ 2 }
		\int_{ T_{ k - 1 } }^{ T_{ k } }
		\energy \left( \chi ; \eta_{ B } \right)
		-
		\int
		\eta_{ B }
		\abs{ \nabla \psi_{ i } }
		\dd{ t }
		\lesssim
		R^{ 2 }
		\sum_{ l \notin \{ i , j \} }
		\int_{ T_{ k - 1 } }^{ T_{ k } }
		\int
		\eta_{ B }
		\abs{ \nabla \chi_{ l } }
		\dd{ t },
	\end{equation*}
	which enables us to absorb the error (\ref{minor_error_term}) into the 
	error term  (\ref{major_error}).
	By applying Young's inequality, we get that for every parameter $ 
	\alpha \in ( 0 , 1 ) $, we have
	\begin{equation*}
		\int_{ T_{ k - 1 } }^{ T_{ k } }
			\int
				\eta_{ B } R V_{ l }
			\abs{ \nabla \chi_{ l } }
		\dd{ t }
		\lesssim
		\alpha
			\int_{ T_{ k - 1 } }^{ T_{ k } }
				\int
					\eta_{ B }
					V_{ l }^{ 2 }
				\abs{ \nabla \chi_{ l } }
			\dd{ t }
		+
		\frac{ R^{ 2 } }{ \alpha }
			\int_{ T_{ k - 1 } }^{ T_{k } }
				\int
					\eta_{ B }
				\abs{ \nabla \chi_{ l } }
			\dd{ t }.
	\end{equation*}
	Collecting our estimates, we end up with an error term which can be 
	estimated from below up to a constant by
	\begin{align*}
		- \alpha \sum_{ l = 1 }^{ P }
			\int_{ 0 }^{ T }
				\int
					V_{ l }^{ 2 }
				\abs{ \nabla \chi_{ l } }
			\dd{ t }
		-
		\frac{ R^{ 2 } }{ \alpha }
		\sum_{ k = 1 }^{ K }
			\sum_{ B \in \mathcal{ B }_{ r } }
				\max_{ i < j }
					\sum_{ l \notin \{ i , j \} }
						\int_{ T_{ k - 1 } }^{ T_{ k } }
							\int
								\eta_{ B }
							\abs{ \nabla \chi_{ l } }
						\dd{ t }.
	\end{align*}
	Moreover, we can choose for fixed $k , B $ and tuple $ (i, j ) $ a sequence 
	of test functions $ \varphi_{ n } $ with $ \abs{ \varphi_{ n } } \leq R $ 
	which converge to $ V_{ j } \mathds{ 1 }_{ \abs{ V_{ j } \leq R } } $
	in the sense that
	\begin{equation*}
		\lim_{ n \to \infty }
		\norm{ \varphi_{ n } - V_{ j } \mathds{ 1 }_{ \abs{ V_{ j } } \leq R } 
		}_{ \lp^{ 2 } \left(
			( T_{ k - 1 }, T_{ k } ) \times \flattorus ,
			\abs{ \nabla \chi_{ j } } \dd{ t }
			\right)
		}
		=
		0.
	\end{equation*}
	Combining these three arguments, we arrive at the estimate that for every 
	parameter $ \alpha \in ( 0 , 1 ) $, it holds that
	\begin{align*}
		A \geq &
		- C \alpha 
		\sum_{ 1 \leq l \leq P }
			\int_{ 0 }^{ T }
				\int
					V_{ l }^{ 2 }
				\abs{ \nabla \chi_{ l } }
			\dd{ t }
		\\
		& +
		\sum_{ k = 1 }^{ K }
			\sum_{ B \in \mathcal{ B }_{ r } }
				\max_{ i < j }
					\int_{ T_{ k - 1 } }^{ T_{ k } } 
						\frac{ \sigma_{ i j } }{ 2 }
						\int
							\eta_{ B }
							\abs{ V_{ j } \mathds{ 1 }_{ \abs{ V_{ j } } \leq R 
							} }^{ 2 }
						\abs{ \nabla \chi_{ j } }
					- 
					C \frac{ R^{ 2 } }{ \alpha }
					\sum_{ l \notin \{ i , j \} }
							\int
								\eta_{ B }
							\abs{ \nabla \chi_{ l } }
						\dd{ t }.
	\end{align*}
	With similar arguments as in the proof of 
	\Cref{convergence_to_multiphase_mcf}, we can argue that by choosing 
	partitions whose width tends to zero, we can pull the maximum inside the 
	time integral to obtain 
	\begin{align*}
		A\geq{}  & 
		- \alpha C
		\sum_{ 1 \leq l \leq P }
		\int_{ 0 }^{ T }
		\int
		V_{ l }^{ 2 }
		\abs{ \nabla \chi_{ l } }
		\dd{ t }
		\\
		& +
		\int_{ 0 }^{ T }
		\sum_{ B \in \mathcal{ B }_{ r } }
		\max_{ i < j }
		\frac{ \sigma_{ i j } }{ 2 }
		\int
		\eta_{ B }
		\abs{ V_{ j } \mathds{ 1 }_{ \abs{ V_{ j } } \leq R 
		} }^{ 2 }
		\abs{ \nabla \chi_{ j } }
		- 
		C \frac{ R^{ 2 } }{ \alpha }
		\sum_{ l \notin \{ i , j \} }
		\int
		\eta_{ B }
		\abs{ \nabla \chi_{ l } }
		\dd{ t }
		\\
		\geq{} &
		\frac{ 1 }{ 2 }
		\sum_{ 1 \leq i < j \leq P }
			\sigma_{ i j }
			\int_{ 0 }^{ T }
				\int_{ \Sigma_{ i j } }
					V_{ i }^{ 2 } \mathds{ 1 }_{ \abs{ V_{ i } } \leq R }
				\dd{ \hm^{ d - 1 } }
			\dd{ t }
		\\
		& -
		C \left(
			\alpha \sum_{ 1 \leq l \leq P }
				\int_{ 0 }^{ T }
					\int
						V_{ l }^{ 2 }
					\abs{ \nabla \chi_{ l } }
				\dd{ t }
			+
			\frac{ R^{ 2 } }{ \alpha }
			\int_{ 0 }^{ T }
				\sum_{ B \in \mathcal{ B }_{ r } }
					\min_{ i \neq j }
						\sum_{ k \notin \{ i , j \} }
							\int
								\eta_{ B }
							\abs{ \nabla \chi_{ k } }
			\dd{ t }
		\right).
	\end{align*}
	With the same arguments as in the proof of 
	\Cref{convergence_to_multiphase_mcf}, by first sending $ r \to 0 $ and then 
	$ \alpha \to 0$ , we thus obtain that for all $R > 0 $
	\begin{equation*}
		A = \liminf_{ \varepsilon \to 0 } \frac{ 1 }{ 2 }
		\int_{ 0 }^{ T }
			\int
				\varepsilon \abs{ \partial_{ t } u_{ \varepsilon } }^{ 2 }
			\dd{ x }
		\dd{ t }
		\geq
		\frac{ 1 }{ 2 }
		\sum_{ 1 \leq i < j \leq P }
			\sigma_{ i j }
			\int_{ 0 }^{ T }
				\int_{ \Sigma_{ i j } }
					V_{ i }^{ 2 }
					\mathds{ 1 }_{ \abs{ V_{ i } } \leq R }
				\dd{ \hm^{ d - 1 } }
			\dd{ t }.
	\end{equation*}
	Therefore the lower semicontinuity of the velocity term now follows form 
	the 
	monotone convergence theorem.
\end{proof}

