\section{Conditional convergence to De Giorgi's mean curvature flow}

In this section, we shall state our solution concept and prove convergence to 
the aforementioned.

\subsection{Definition of $ \bv $-solutions to De Giorgi's mean curvature flow}

\begin{definition}[$\bv$-solution to  De Giorgi's multiphase mean curvature 
flow]
	\label{de_giorgi_solution_to_mmcf}
	Fix some finite time horizon $ T < \infty $, a $ P \times P $ matrix of 
	surface tensions $ \sigma $ and initial data $ \chi^{ 0 } \colon \flattorus 
	\to \{ 0 , 1 \}^{ P } $ with $ \energy_{ 0 } \coloneqq ( \chi^{ 0 } ) $ and 
	$ \sum_{ i = 1 }^{ P } \chi_{ i }^{ 0 } = 1 $. We say that
	\begin{equation*}
		\chi \in \cont \left(
		[ 0 , T ]
		;
		\lp^{ 2 } \left( \flattorus ; \{ 0 , 1 \}^{ P } \right)
		\right)
	\end{equation*}
	with $ \esssup_{ 0 \leq t \leq T } \energy ( \chi ) $ and $ \sum_{ i = 1 
	}^{ P } \chi_{ i } = \sum_{ i = 1 }^{ P } \mathds{ 1 }_{ \Omega_{ i } } = 
	1  $ \emph{is a } $\bv$-\emph{solution to De Giorgi's mean 
	curvature flow with initial data} $ \chi^{ 0 } $ \emph{and surface 
	tensions} $ \sigma $ if the following holds. 
	\begin{enumerate}
		\item 
		For all $ 1 \leq i \leq P $, there exist normal 
		velocities $ V_{ i } \in \lp^{ 2 } ( \abs{ \nabla \chi_{ i } } \dd{ t } 
		) $ 
		of the interfaces 
		in the sense that
		\begin{equation*}
			\partial_{ t } \chi_{ i }
			=
			V_{ i } \abs{ \nabla \chi_{ i } } \dd{ t }
		\end{equation*}
		holds in the distributional sense on $ ( 0 , T ) \times \flattorus $.
		
		\item 
		There exist a mean curvature $ H \in 
		\lp^{ 2 } ( \sum_{ 1\leq i < j \leq P } \dd{ \hm^{ d -1 }|_{ \Sigma_{ i 
		, j } } } \dd{ t } ) $ which satisfies
		\begin{equation*}
			\sum_{ 1 \leq i < j \leq P }
			\int_{ 0 }^{ T }
				\int_{ \Sigma_{ i , j } }
					H
					\inner*{ \xi }{ \nu_{ i } }
				\dd{ \hm^{ d - 1 } }
			\dd{ t }
			=
			-
			\sum_{ 1 \leq i < j \leq P }
			\int_{ 0 }^{ T }
				\int_{ \Sigma_{ i , j } }
					\inner*{
						\diff \xi }
					{ \mathrm{Id} - \nu_{ i } \otimes \nu_{ i } }
				\dd{ \hm^{ d - 1 } }
			\dd{ t }
		\end{equation*}
		for all test vector fields 
		$ \xi \in \cont_{ \mathrm{c} }^{ \infty } \left(
			( 0 , T ) \times \flattorus ; \mathbb{ R }^{ d }
		\right) $,
		where $ \nu_{ i } \coloneqq \nabla \chi_{ i } / \abs{ \nabla \chi_{ i } 
		} $ are the inner unit normals and $ \Sigma_{ i , j } \coloneqq 
		\partial_{ \ast } \Omega_{ i } \cap \partial_{ \ast } \Omega_{ j } $
		is the $ (i, j )$-th interface.
		
		\item 
		The partition $ \chi $ satisfies a De Giorgi type optimal energy 
		dissipation inequality in the sense that for almost every time $ 0 < T' 
		< T $, we have
		\begin{equation*}
			\energy ( \chi ( T' ) )
			+
			\frac{ 1 }{ 2 }
			\sum_{ 1 \leq i < j \leq P }
				\sigma_{ i , j }
				\int_{ 0 }^{ T' }
					\int_{ \Sigma_{ i , j } }
						V_{ i }^{ 2 }
						+
						H^{ 2 }
					\dd{ \hm^{ d - 1 } }
				\dd{ t }
			\leq
			\energy ( \chi^{ 0 } ).
		\end{equation*}
		
		\item
		The initial data is achieved in the space $ \cont \left( [ 0 , T ] ; 
		\lp^{ 2 } ( \flattorus ) \right) $.
	\end{enumerate}
\end{definition}

\begin{theorem}
	\label{convergence_to_de_giorgis_multiphase_mcf}
	Let a smooth multiwell potential $ W \colon \mathbb{ R }^{ N } \to [ 0, 
	\infty ) $ satisfy the assumptions 
	(\ref{polynomial_growth})-(\ref{perturbation bound}). Let $ T < \infty 
	$ be an arbitrary finite time horizon. Given a sequence of initial data 
	$ u_{ \varepsilon }^{ 0 } \colon \flattorus \to \mathbb{ R }^{ N } $ 
	approximating a partition 
	$ \chi^{ 0 } \in \bv \left( \flattorus ; \{ 0 , 1 \}^{ P } \right) $ 
	in the sense that 
	$ u_{ \varepsilon }^{ 0 } \to u^{ 0 } =  \sum_{ 1 \leq i \leq P } 
	\chi_{ i }^{ 0 } \alpha_{ i } $ 
	holds pointwise almost everywhere and 
	\begin{equation*} 
		\energy_{ 0 } 
		\coloneqq 
		\energy ( \chi^{ 0 } ) 
		= 
		\lim_{ \varepsilon \to 0 } 
		\energy_{ \varepsilon } ( u_{ \varepsilon }^{ 0 } ) 
		< 
		\infty,
	\end{equation*}
	we have that for 
	some subsequence of solutions to the Allen--Cahn equation
	(\ref{allen_cahn_eq}) $ u_{\varepsilon } $ with initial datum $ u_{ 
		\varepsilon }^{ 0 } $, there exists a time-dependent partition $ \chi $ 
	with 
	$ \chi \in \bv \left( ( 0 , T ) \times \flattorus ; \{ 0 , 1 \}^{ P } 
	\right) $ and
	$ \chi 
	\in \cont \left( [ 0 , T ] ; \lp^{ 2 } \left( \flattorus ;  \{ 0 , 1 
	\}^{ P } \right) \right) $ such that $ u_{ \varepsilon } $ converges to 
	$ u \coloneqq \sum_{ 1 \leq i \leq P } \chi_{ i } \alpha_{ i } $ almost 
	everyhwere. Moreover $ u $ assumes the initial data $ u^{ 0 } $ in $ 
	\cont \left( [ 0, T ] ; \lp^{ 2 }( \flattorus ) \right) $. If we 
	additionally assume that the 
	time-integrated energies converge (\ref{energy_convergence}), then $ 
	\chi $ is a $ \bv $-solution to De Giorgis mean curvature flow in the sense 
	of 
	\Cref{de_giorgi_solution_to_mmcf}.
\end{theorem} 

This result is similar to \Cref{convergence_to_multiphase_mcf} and we only need 
to prove that $ \chi $ is a $ \bv $-solution to De Giorgis mean curvature flow.

\begin{proof}
	Let us for simplicity first consider the twophase case. The idea of the 
	proof is that we already have an optimal energy dissipation inequality 
\end{proof}


