\section{Gradient flows}

In the simplest case, a gradient flow of a given energy $ \energy \colon \mathbb{ R }^{ N } \to \mathbb{ R } $ (with respect to the euclidean inner product) is a solution to the ordinary differential equation
\begin{equation}
	\label{gf_basic_equation}
	\dv{ t } x ( t ) = - \nabla \energy ( x ( t ) ) ,
\end{equation}
where we usually prescribe some initial value $ x ( 0 ) = x_{ 0 } \in \mathbb{ R }^{ N } $. The central structure here is that on the right hand side of the equation, we do not have any function, but the gradient of some function. What a solution $ x $ does is that it moves in the direction of the steepest descent of the energy $ \energy $. Moreover this allows for the following computation given a solution $ x $ of (\ref{gf_basic_equation}):
\begin{align*}
	\dv{ t } \energy ( x ( t ) ) 
	& =
	\inner*{ \nabla \energy ( x ( t ) ) }{ x ' ( t ) }
	=
	- \abs{ \nabla \energy ( x ( t ) ) }^{ 2 }
	=
	- \abs{ x' ( t ) }^{ 2 }
	\\
	& =
	- \frac{ 1 }{ 2 }
	\left(
		\abs{ x'( t ) }^{ 2 }
		+
		\abs{ \nabla \energy ( x ( t ) ) }^{ 2 }
	\right).
\end{align*}
We especially obtain that the the function $ \energy ( x ( t ) ) $ is non-increasing, which coincides with our intuition that $ x $ moves along the steepest descent of the energy. But more precisely, we obtain from the fundamental theorem of calculus the \emph{energy dissipation identity}
\begin{equation}
	\label{basic_energy_dissipation_identity}
	\energy ( x ( T ) )
	+
	\frac{ 1 }{ 2 }
	\int_{ 0 }^{ T }
		\abs{ x' ( t ) }^{ 2 }
		+
		\abs{ \nabla \energy ( x ( t ) ) }^{ 2 }
	\dd{ t }
	= 
	\energy ( x ( 0 ) ).
\end{equation}
One could now raise the question if this is identity already characterizes the equation, which means that if some $ x $ satisfies equation (\ref{basic_energy_dissipation_identity}), it should already be a solution to the ordinary differential equation (\ref{gf_basic_equation}). But as it turns out, we can go even one step further, namely we only ask for the \emph{optimal energy dissipation inequality} given by
\begin{equation}
	\label{basic_optimal_energy_dissipation_inequality}
	\energy ( x ( T ) )
	+
	\frac{ 1 }{ 2 }
	\int_{ 0 }^{ T }
		\abs{ x' ( t ) }^{ 2 }
		+
		\abs{ \nabla \energy ( x ( t ) ) }^{ 2 }
	\dd{ t }
	\leq
	\energy ( x ( 0 ) ).
\end{equation}
We call this inequality optimal since as demonstrated before, we usually expect an equality to hold.
Now let us assume that $ x $ satisfies (\ref{basic_optimal_energy_dissipation_inequality}) and has sufficient regularity.
Then we can estimate again by the fundamental theorem that
\begin{align*}
	& \frac{ 1 }{ 2 }
	\int_{ 0 }^{ T }
		\abs{
			\nabla \energy ( x ( t ) )
			+
			x' ( t ) 
		}^{ 2 }
	\dd{ t }
	\\
	={} &
	\int_{ 0 }^{ T }
		\inner*{ \nabla \energy ( x ( t ) ) }{ x' ( t ) }
	\dd{ t }
	+
	\frac{ 1 }{ 2 }
	\int_{ 0 }^{ T }
		\abs{ x' ( t ) }^{ 2 }
		+
		\abs{ \nabla \energy ( x ( t ) ) }^{ 2 }
	\dd{ t }
	\\
	={} &
	\energy ( x ( T ) ) - \energy ( x ( 0 ) ) 
	+
	\frac{ 1 }{ 2 }
	\int_{ 0 }^{ T }
		\inner*{ \nabla \energy ( x ( t ) ) }{ x' ( t ) }
	\dd{ t }
	\leq 0.
\end{align*}
Since we started with an integral over a non-negative function, this implies that for almost every time $ t $, we have that $ x' ( t ) = - \nabla \energy ( x ( t ) ) $, which proves that $ x $ (again under sufficient regularity assumptions) is a gradient flow of the energy $ \energy $.

The real strength of formulating the differential equation (\ref{gf_basic_equation}) via inequality (\ref{basic_optimal_energy_dissipation_inequality}) becomes clear if we want to consider gradient flows in a more complicated setting by replacing the space $ \mathbb{ R }^{ N } $. In order to formulate equation (\ref{gf_basic_equation}), we need to have a notion of differentiation and a gradient in the target of $ x $. Therefore we may use Hilbert spaces or smooth Riemannian manifolds as a suitable substitute for $ \mathbb{ R }^{ N } $. 

A first example for a generalization is therefore the heat equation $ \partial_{ t } u - \Delta u = 0 $ for some open and bounded set $ \Omega $. As our space, we shall take $ \mathrm{ H }_{ 0 }^{ 1 } ( \Omega ) $ equipped with the inner product
\begin{equation*}
	\inner*{ u }{ v }
	\coloneqq
	\int
		\inner*{ \nabla u }{ \nabla v }
	\dd{ x }.
\end{equation*}
Our energy is given by the Dirichlet functional
\begin{equation*}
	\energy ( u ) \coloneqq
	\frac{ 1 }{ 2 }
	\int
		\abs{ \nabla u }^{ 2 }
	\dd{ x }
\end{equation*}
Then we can compute that at least on $ \mathrm{ H }_{ 0 }^{ 1 } \cap \mathrm{ H }^{ 2 } ( \Omega ) $, the Fréchet derivative of $ \energy $ is given by $ \diff ( E ) ( u ) [ v ] = - \int \Delta u v \dd{ x } $.