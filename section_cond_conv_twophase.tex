\section{Convergence in the two-phase case}

The proof for the existence of an evolving set in this section is loosely based 
on its multiphase version presented 
in \cite{convergence_of_allen_cahn_equation_to_multiphase_mean_curvature_flow}. 
However the proof simplifies in the two-phase case and thus gives an easier 
introduction into the topic.

The only assumption we make for now is that the energies of the initial 
functions $ \energy_{ \varepsilon } ( u_{ \varepsilon }^{ 0 } ) $ stay 
uniformly bounded as $ \varepsilon $ tends to zero. 
Then due to the energy dissipation inequality (\ref{energy_dissipation_sharp}), 
we already obtain that $ \energy_{ \varepsilon } \left( u_{\varepsilon } ( t ) 
\right) $ stays uniformly bounded for all
$ 0 \leq t \leq T $.

Another important observation for the convergence is the classic 
Modica--Mortola 
trick (\cite{modica_mortola_un_esempio_di_gamma_convergenza}) : Let $ \alpha < 
\beta $ be the two distinct 
zeros of the doublewell potential $ W $.
Then we define a primitive of $ \sqrt{ 2 W ( u ) } $ via
\begin{equation*}
	\phi ( u ) 
	\coloneqq
	\int_{ \alpha }^{ u }
	\sqrt{ 2 W ( s ) }
	\dd{ s },
\end{equation*}
see also \Cref{graph_of_phi}.
\begin{figure}[ht]
	\centering
	\begin{tikzpicture}
		\label{Plot of the doublewell potential $ W(u) = (u^2 - 1 )^{ 2 } $}
		\begin{axis}[
			axis lines = left,
			xlabel = \(u\),
			ylabel = {\(\phi(u)\)},
			]
			%Below the red parabola is defined
			\addplot [
			domain=-2:2, 
			samples=100, 
			color=black,
			]
			{(x^5)/5-(2*x^3)/3 + x  + 1/5 -2/3 + 1 };
		\end{axis}
	\end{tikzpicture}
	\caption{The graph of the function $ \phi $ for $ W ( u ) = ( u^{ 2 } - 1 
	)^{ 2 } $ }
	\label{graph_of_phi}
\end{figure}
For $ \psi_{ \varepsilon } \coloneqq \phi \circ u_{ \varepsilon } $, we can 
show that $ \psi_{ \varepsilon } \in \wkp^{ 1, 1 } ( ( 0 , T ) \times 
\flattorus ) $ with weak derivatives $ \nabla \psi_{ \varepsilon } = \sqrt{ 2 W 
( u_{ \varepsilon } ) } \nabla u_{ \varepsilon } $ and $ \partial_{ t } \psi_{ 
\varepsilon } = \sqrt{ 2 W ( u_{ \varepsilon } ) } \partial_{ t } u_{ 
\varepsilon } $. 
Thus via Young's inequality, we can compute
\begin{align}
	\energy_{ \varepsilon } ( u_{ \varepsilon } )
	& =
	\int
	\frac{ 1 }{ \varepsilon }
	W ( u_{ \varepsilon } ) 
	+
	\frac{ \varepsilon }{ 2 }
	\abs{ \nabla u_{ \varepsilon } }^{ 2 }
	\dd{ x }
	\notag
	\\
	& \geq
	\int
	\sqrt{ 2 W ( u_{ \varepsilon } ) }
	\abs{ \nabla u_{ \varepsilon } }
	\dd{ x }
	\notag
	\\
	& =
	\int
	\abs{ \nabla \psi_{ \varepsilon} }
	\dd{ x }
	\label{modica_mortula_trick}.
\end{align}
Therefore we might hope for good compactness properties of $ \phi 
\circ u_{ \varepsilon } $.
Moreover since the energies stay bounded and the summand $  W ( u ) / 
\varepsilon $ penalizes any mass outside of the wells, we expect that a limit 
function 
is concentrated in $ \alpha $ and $ \beta $.
Indeed we show this in the following Proposition.

\begin{proposition}
	\label{initial_convergence_result}
	Let $ u_{ \varepsilon }^{ 0 } $ be well prepared initial data in the sense 
	that
	\begin{equation}
		\lim_{ \ \varepsilon \to 0 }
		\energy_{ \varepsilon } ( u_{ \varepsilon }^{ 0 } ) 
		=
		\energy ( u^{ 0 } )
		\eqqcolon 
		\energy_{ 0 }
		<
		\infty.
	\end{equation}
	Then there exists for any sequence $ \varepsilon \to 0 $ some 
	non-relabelled subsequence such that the solutions of the Allen--Cahn 
	equation (\ref{allen_cahn_eq}) with initial conditions $ u_{ \varepsilon 
	}^{ 0 } $ converge in $ \lp^{ 1 } ( ( 0, T ) \times \flattorus ) $ to some 
	$ u = \alpha ( 1 - \chi ) + \beta \chi $ with $ \chi \in \bv \left( ( 0 , T 
	) \times \flattorus ; \{ 0 , 1 \} \right) $.
	Furthermore we have
	\begin{equation*}
		\energy ( u ( t ) ) 
		\leq
		\liminf_{ \varepsilon \to 0 }
			\energy_{ \varepsilon } ( u_{ \varepsilon } ( t ) )
		\leq 
		\energy_{ 0 }
	\end{equation*}
	for almost every $ 0 \leq t \leq T $. Moreover the compositions $ \psi_{ 
	\varepsilon } = \phi \circ u_{ \varepsilon } $ are uniformly bounded in $ 
	\bv \left( ( 0, T ) \times 
	\flattorus \right) $ and converge to $ \phi \circ u $ in $ \lp^{ 1 } ( ( 0 
	, T ) \times \flattorus ) $.
\end{proposition}

\begin{proof}
	From the energy dissipation inequality (\ref{energy_dissipation_sharp}) in \Cref{existence_of_ac_solution} we infer that for all $ \varepsilon > 0 $, it holds that
	\begin{equation}
		\label{unif_bound_on_energies}
		\sup_{ 0 \leq t \leq T }
		\energy_{ \varepsilon } ( u_{ \varepsilon } ( t ) ) 
		\leq
		\energy_{ \varepsilon } ( u_{ \varepsilon}^{ 0 } )
		\leq C.
	\end{equation}
	By the calculation (\ref{modica_mortula_trick}) we thus obtain that $ 
	\nabla 
	\psi_{ \varepsilon } $ is uniformly bounded in $ \lp^{ 1 } \left( ( 0 , T ) 
	\times \flattorus \right) $.
	Moreover we may estimate by the upper and lower growth bound 
	(\ref{polynomial_growth}) on $ W $ 
	\begin{align}
		\int_{ 0 }^{ T }
		\int
		\abs{ \psi_{ \varepsilon } }
		\dd{ x }
		\dd{ t }
		& =
		\int_{ 0 }^{ T }
		\int
		\abs{
			\int_{ \alpha }^{ u_{ \varepsilon } }
			\sqrt{ 2 W ( s ) }
		}
		\dd{ s }
		\dd{ x }
		\dd{ t }
		\notag
		\\
		& \leq
		\int_{ 0 }^{ T }
		\int
		\abs{ u_{ \varepsilon } - \alpha }
		\sup_{ s \in [ \alpha, u_{ \varepsilon } ] }
		\sqrt{ 2 W ( s ) }
		\dd{ x }
		\dd{ t }
		\notag
		\\
		& \lesssim
		1 + 
		\int_{ 0 }^{ T }
		\int
		\abs{ u_{ \varepsilon } }^{ 1 + p/ 2 }
		\dd{ x }
		\dd{ t }
		\notag
		\\
		& \lesssim
		1 + 
		\int_{ 0 }^{ T }
		\int
		W ( u_{ \varepsilon } )
		\dd{ x }
		\dd{ t },
		\label{l1_estimate_for_psi_epsilon}
	\end{align}
	which is uniformly bounded via the energy bound. Hence we have that $ 
	\psi_{ \varepsilon } $ is a bounded sequence in $ \bv( ( 0 , T ) \times 
	\flattorus ) $ and therefore there exists some non-relabelled subsequence 
	and some $ \psi \in \bv ( ( 0 , T ) \times \flattorus ) $ such that $ 
	\psi_{ \varepsilon } $ converges to $ \psi $ in $ \lp^{ 1 } ( ( 0 , T ) 
	\times \flattorus ) $.
	
	We notice that since $ W $ is non-negative and only has a discrete set of zeros, the function $ \phi $ is strictly increasing and continuous on $ \mathbb{ R } $, and is thus invertible. Moreover we may pass to a further non-relabelled subsequence of $ \psi_{ \varepsilon } $ which converges almost everywhere to $ \psi $. 
	Thus defining $ u \coloneqq \phi^{ - 1 } ( \psi ) $, we obtain that
	\begin{equation*}
		u_{ \varepsilon } = \phi^{ - 1 } ( \psi_{ \varepsilon } ) \to \phi^{ - 1 } ( \psi ) = u
	\end{equation*}
	converges pointwise almost everywhere.
	Moreover we notice that by Fatou's Lemma and the boundedness of the 
	energies, we have
	\begin{equation*}
		\int
		W ( u ) 
		\dd{ x }
		\leq
		\liminf_{ \varepsilon \to 0 }
		\int
		W ( u_{ \varepsilon } )
		\dd{ x }
		\leq
		\liminf_{ \varepsilon \to 0 }
		\varepsilon \energy_{ \varepsilon } ( u_{ \varepsilon } )
		=
		0.
	\end{equation*}
	Again using the non-negativity of $ W $, this yields that $ W ( u ) = 0 $ almost everywhere. 
	Thus $ u \in \{ \alpha , \beta \} $ almost everywhere and we may write $ u = \alpha ( 1 - \chi ) + \beta \chi $ for some $ \chi \colon ( 0 , T ) \times \flattorus \to \{ 0 , 1 \} $.
	By looking at the definition of $ u $, we moreover obtain that 
	\begin{equation}
		\label{represenation_of_psi}
		\psi = \phi ( u ) = 
		\phi ( \alpha ) ( 1 - \chi )
		+ 
		\phi ( \beta )	\chi
		= 
		\int_{ \alpha }^{ \beta } \sqrt{ 2 W ( s ) } \dd{ s } \chi 
		\eqqcolon
		\sigma \chi.
	\end{equation}
	Since $ \psi $ is a function of bounded variation, this implies that $ \chi 
	$ is of bounded variation as well.
	
	Finally from the energy bound and the estimate $ \abs{ u_{ \varepsilon } 
	}^{ p } \lesssim 1 + W ( u_{ \varepsilon } ) $, we infer that $ u_{ 
	\varepsilon } $ is $ 
	\lp^{ p }$-bounded. Since $ u_{ \varepsilon } $ converges pointwise 
	almost everywhere to $ u $, we obtain the desired $ \lp^{ 1 } $-convergence.
	This finishes the proof.
\end{proof}

Notice that in the proof, we defined the constant $ \sigma \coloneqq \int_{ 
\alpha }^{ \beta } \sqrt{ 2 W ( s ) } \dd{ s } $, which will later be the 
surface tension for the mean curvature flow.
Nextup, we want to make sure that $ u $ respectively $ \chi $ assume their 
initial data.
Furthermore we will obtain a useful bound on the time derivative.

\begin{lemma}
	\label{l2_bound_on_psi_varepsilon}
	In the situation of \Cref{initial_convergence_result}, we have $ \psi_{ 
	\varepsilon } \in \wkp^{ 1 , 2 } ( [ 0 , T ] ; \lp^{ 1 } ( \mathbb{ T } ) ) 
	$ with the corresponding estimate
	\begin{equation}
		\label{bound_on_time_dv_of_psi_epsilon}
		\left(
		\int_{ 0 }^{ T }
		\left(
		\int
		\abs{ \partial_{ t } \psi_{ \varepsilon } }
		\dd{ x }
		\right)^{ 2 }
		\dd{ t }
		\right)^{ 1/2 }
		\lesssim
		\energy_{ \varepsilon } ( u_{ \varepsilon }^{ 0 } ).
	\end{equation}
	Furthermore the sequence $ u_{ \varepsilon } $ is precompact in $ \cont \left( [ 0 , T ] ; \lp^{ 2 } ( \flattorus ) \right) $. 
\end{lemma}

\begin{proof}
	The desired regularity will follow quite directly from Hölder's inequality. 
	For the desired precompactness, we will first show precompactness of $ 
	\psi_{ \varepsilon } $ by Arzelà--Ascoli, which implies that $ u _{ 
	\varepsilon } $ converges in measure. 
	To finish the proof we use the equiintegrability of $ u_{ 
	\varepsilon } $ uniformly in time and deduce our claim.
	\begin{description}[wide=0pt]
		\item[Step 1:] $ \psi_{ \varepsilon } \in \wkp^{ 1 , 2 } \left( [ 0 , T 
		] ; \lp^{ 1 } ( \flattorus ) \right) $ and satisfies the inequality 
		(\ref{bound_on_time_dv_of_psi_epsilon}).
		
		By the same calculation as in estimate 
		(\ref{l1_estimate_for_psi_epsilon}), we 
		can infer that
		\begin{equation*}
			\int_{ 0 }^{ T }
			\left(
			\int
			\abs{ \psi_{ \varepsilon } }
			\dd{ x }
			\right)^{ 2 }
			\dd{ t }
			\lesssim
			\int_{ 0 }^{ T }
			\left(
			1 + \int W ( u_{ \varepsilon } ) \dd{ x }
			\right)^{ 2 }
			\dd{ t } 
			\leq C
		\end{equation*}
		from the uniform boundedness of the energies 
		(\ref{unif_bound_on_energies}). For the desired bound 
		(\ref{bound_on_time_dv_of_psi_epsilon}), we estimate via Hölder's 
		inequality that
		\begin{align*}
			\int_{ 0 }^{ T }
			\left(
			\int
			\abs{ 
				\partial_{ t } \psi_{ \varepsilon }
			}
			\dd{ x }
			\right)^{ 2 }
			\dd{ t }
			& =
			\int_{ 0 }^{ T }
			\left(
			\int
			\sqrt{ 2 W ( u_{ \varepsilon } ) }
			\abs{ \partial_{ t } u_{ \varepsilon } }
			\dd{ x }
			\right)^{ 2 }
			\dd{ t }
			\\
			& \leq
			\int_{ 0 }^{ T }
			\int
			\frac{ 1 }{ \varepsilon }
			2 W ( u_{ \varepsilon } )
			\dd{ x }
			\int
			\varepsilon
			\abs{ \partial_{ t } u_{ \varepsilon } }^{ 2 }
			\dd{ x }
			\dd{ t }
			\\
			& \leq
			2 \energy_{ \varepsilon } ( u_{ \varepsilon }^{ 0 } )
			\int_{ 0 }^{ T }
			\int
			\varepsilon
			\abs{ \partial_{ t } u_{ \varepsilon } }^{ 2 }
			\dd{ x }
			\dd{ t }
			\\
			& \leq
			2 \left(\energy_{ \varepsilon } ( u_{ \varepsilon }^{ 0 } 
			)\right)^{ 2 }.
		\end{align*}
		The last two inequalities follow from the energy dissipation 
		estimate
		(\ref{energy_dissipation_sharp}).
		
		\item[Step 2:] The sequence $ \psi_{ \varepsilon } $ is precompact in $ \cont \left( [ 0 , T ] ; \lp^{ 1 } ( \flattorus ) \right) $.
		
		As noted in embedding (\ref{w12_embeds_into_c1half}), we have
		\begin{equation*}
			\wkp^{ 1 , 2 } \left( [ 0 , T ] ; \lp^{ 2 } ( \flattorus ; \mathbb{ R }^{ N } ) \right)
			\hookrightarrow
			\cont^{ 1 / 2 } \left( [ 0 , T ] ; \lp^{ 2 } ( \flattorus ; \mathbb{ R }^{ N } ) \right)
		\end{equation*}
		from which the equicontinuity of the sequence follows by Step 1.
		Moreover for a fixed time $ t $, the arguments from 
		\Cref{initial_convergence_result} yield that $ \psi_{ \varepsilon } ( t 
		) $ is a bounded sequence in $ \wkp^{ 1 , 1 } ( \flattorus ) $ and is 
		thus precompact in $ \lp^{ 1 } ( \flattorus ) $.
		Therefore the Arzelà--Ascoli Theorem yields the desired precompactness.
		
		\item[Step 3:] The sequence $ u_{ \varepsilon } $ converges to $ u $ in measure uniformly in time.
		
		We already know that $ \psi_{ \varepsilon } $ converges to $ \phi \circ 
		u $ almost everywhere. By passing to another non-relabelled 
		subsequence, we infer from the precompactness in Step 2 that
		\begin{equation*}
			\lim_{ \varepsilon \to 0 }
			\esssup_{ 0 \leq t \leq T }
			\int	
			\abs{ \psi_{ \varepsilon } ( t , x ) - \phi \circ u ( t , x ) }
			\dd{ x }
			= 0.
		\end{equation*}
		By plugging in the definition of $ \phi $, we obtain
		\begin{align*}
			\lim_{ \varepsilon \to 0 }
			\esssup_{ 0 \leq t \leq T }
			\int
			(1 - \chi)
			\abs{ \psi_{ \varepsilon } ( t , x ) }
			\dd{ x }
			& =
			\lim_{ \varepsilon \to 0 }
			\esssup_{ 0 \leq t \leq T }
			\int
			(1 - \chi)
			\abs{ \psi_{ \varepsilon } ( t , x ) - \phi \circ u ( t , x ) }
			\dd{ x }
			=
			0	
			\shortintertext{and}
			\lim_{ \varepsilon \to 0 }
			\esssup_{ 0 \leq t \leq T }
			\int
			\chi
			\abs{ \psi_{ \varepsilon } ( t , x ) - \sigma }
			\dd{ x }
			& = 
			\lim_{ \varepsilon \to 0 }	
			\esssup_{ 0 \leq t \leq T }
			\int
			\chi
			\abs{ \psi_{ \varepsilon } ( t , x ) - \phi \circ u ( t , x ) }
			\dd{ x }
			= 0.
		\end{align*}
		Moreover since $ \phi $ is strictly increasing we have that for a given 
		$ \delta < \beta - \alpha $, there exists some $ \rho > 0 $ such that
		$ \abs{ s - \alpha } > \delta / 2 $ and $ \abs{ s - \beta } > \delta / 
		2 $ already implies $ \abs{ \phi ( s ) } > \rho $ and $ \abs{ \phi ( s 
		) - \sigma } > \rho $.
		Therefore we may estimate by the triangle inequality
		\begin{align*}
			& \esssup_{ 0 \leq t \leq T }
			\lm^{ d } \left( \left\{ x \in \flattorus \, \colon \, \abs{ u_{ \varepsilon } ( t, x ) - u ( t, x ) } > \delta \right\} \right)
			\\
			\leq {} &
			\esssup_{ 0 \leq t \leq T }
			\lm^{ d } \left(
			\left\{  ( 1- \chi ( t, x ) ) \abs{ u_{ \varepsilon } ( t , x ) - \alpha } > \delta / 2 \right\}
			\right)
			+
			\lm^{ d } \left( \left\{ \chi ( t , x ) \abs{ u_{ \varepsilon } ( t , x ) - \beta } > \delta / 2 \right\} \right)
			\\
			\leq {} &
			\esssup_{ 0 \leq t \leq T }
			\lm^{ d } \left(
			\left\{ ( 1 - \chi ( t, x ) ) \abs{ \psi_{ \varepsilon } ( t , x ) } > \rho \right\} 
			\right)
			+
			\lm^{ d } \left(
			\left\{ \chi ( t , x )  \abs{ \psi_{ \varepsilon } ( t , x ) - \sigma } > \rho \right\} 
			\right)
			\\
			\leq {} &
			\esssup_{ 0 \leq t \leq T }
			\frac{ 1 }{ \rho }
			\int
			( 1 - \chi ( t , x ) ) \abs{ \psi_{ \varepsilon } ( t , x ) } 
			\dd{ x }
			+
			\frac{ 1 }{ \rho }
			\int
			\chi ( t , x ) \abs{ \psi_{ \varepsilon } ( t , x ) - \sigma }
			\dd{ x }.
		\end{align*}
		The last estimate is Markov's inequality.
		This term vanishes as $ \varepsilon $ goes to zero, proving our claim.
		
		\item[Step 4:] $ u_{ \varepsilon }^{ 2 } $ is equiintegrable uniformly 
		in time.
		
		We have 
		\begin{equation}
			\label{gives_us_uniformintg}
			0 \leq u_{ \varepsilon }^{ 2 } \lesssim 1 + W ( u_{ \varepsilon } )
		\end{equation}
		by the growth bounds (\ref{polynomial_growth}) on $ W $. The function $ 
		W ( u_{ \varepsilon } ) $ converges to 0 in $ \lp^{ 1 } ( \flattorus ) 
		$ uniformly in time by the energy bound (\ref{unif_bound_on_energies}).
		Hence $ W ( u_{ \varepsilon } ) $ is equiintegrable uniformly in time 
		and by inequality (\ref{gives_us_uniformintg}) $ u_{ \varepsilon }^{ 2 
		} $ is as well.
		
		\item[Step 5:] $ u_{ \varepsilon } $ converges in $ \cont \left( [ 0 , T ] ; \lp^{ 2 } ( \flattorus ) \right) $.
		
		We repeat the proof that convergence in measure and 
		equiintegrability imply $ \lp^{ 1 } $-convergence, while additionally 
		keeping the 
		uniformity in time.
		
		Take some $ \delta > 0 $. Then we decompose the integral into
		\begin{equation*}
			\int \abs{ u_{ \varepsilon } - u }^{ 2 } \dd{ x }
			=
			\int_{ \{ \abs{ u_{ \varepsilon } - u } \geq \delta \} }
			\abs{ u_{\varepsilon } - u }^{ 2 }
			\dd{ x }
			+
			\int_{ \{ \abs{ u_{ \varepsilon } - u } < \delta \} }
			\abs{ u_{ \varepsilon } - u }^{ 2 }
			\dd{ x }.
		\end{equation*}
		For the first summand, we notice that
		\begin{equation*}
			\esssup_{ 0 \leq t \leq T }
			\int_{ \{ \abs{ u_{ \varepsilon } - u } \geq \delta \} }
			\abs{ u_{\varepsilon } - u }^{ 2 }
			\dd{ x }
			\lesssim
			\esssup_{ 0 \leq t \leq T }
			\int_{ \{ \abs{ u_{ \varepsilon } - u } \geq \delta \} }
			1 + u_{\varepsilon }^{ 2 }
			\dd{ x }
			\to 0
		\end{equation*}
		as $ \varepsilon $ goes to zero since $ \lm^{ d }\left( \abs{ 
		u_{\varepsilon } - u } \geq \delta \right)  $ converges to zero 
		uniformly in time by Step 3 and $ u_{\varepsilon }^{ 2 } + 1 $ is 
		equiintegrable uniformly in time by Step 4.
		For the second summand, we simply estimate
		\begin{equation*}
			\esssup_{ 0 \leq t \leq T }
			\int_{ \{ \abs{ u_{ \varepsilon } - u } < \delta \} }
			\abs{ u_{ \varepsilon } - u }^{ 2 }
			\dd{ x }
			\leq
			\delta^{ 2 } \Lambda^{ d }
		\end{equation*}
		Taking the limit superior as $ \varepsilon $ tends to zero of this 
		inequality yields that the right hand side can be made arbitrarily 
		small, which gives us
		\begin{equation*}
			\esssup_{ 0 \leq t \leq T }
			\int
			\abs{ u_{ \varepsilon } - u }^{ 2 }
			\dd{ x }
			\to 
			0
		\end{equation*}
		as $ \varepsilon $ tends to zero.
	\end{description}
	This concludes the proof.
\end{proof}

\begin{remark}
	Assume that the initial conditions $ u_{ \varepsilon }^{ 0 } $ converge 
	pointwise almost everywhere to a function $ u^{ 0 } $. Then it follows from 
	the previous Lemma that $ u $ assumes this initial data in $ \lp^{ 2 } ( 
	\flattorus ) $.
\end{remark}

