\chapter{Notation}


	\begin{itemize}[leftmargin=*]
		\item	
		Let $ u \colon \mathbb{ R }^{ d } \to \mathbb{ R }^{ N } $ be 
		differentiable at a point $ x \in \mathbb{ R }^{d } $. We write $ \diff 
		u ( x ) $ for the (total) derivative at $ x $, which means that
		$ \diff u ( x ) \in \mathbb{ R }^{ N \times d } $ and
		$ \left( \diff u ( x ) \right)_{ i j } = \partial_{ x_{ j } } u^{ i } ( 
		x ) $. We always use the notation $ \nabla u ( x ) $ for the transposed 
		of the total derivative. 
		
		\item
		If $ u \colon ( 0 , T ) \times \mathbb{ R }^{ d } \to \mathbb{ R 
		}^{ N } $, then we denote by $ \diff u (t, x ) $  respectively $ \nabla 
		u ( 
		t,  x ) $ only the derivatives in space, and always write $ \partial_{ 
		t } u $ for the derivative in time.
	
		\item
		We use the symbol $ \lesssim $ if an inequality holds up to a positive 
		constant on the right hand side. This constant may only depend on the 
		dimensions and the chosen potential $ W $.
		
		\item 
		The bracket $ \inner*{\cdot}{\cdot} $ is used as the euclidean inner 
		product. Depending on the situation, it acts on vectors or matrices.
		If we apply $ \abs{ \cdot } $ to a vector or matrix, then we always 
		uses the norm induced by the euclidean inner product.
		
		\item 
		We denote by $ \cont_{ \mathrm{c} } $ the compactly supported 
		continuous functions. Note that for the flat torus $ \flattorus $, we 
		have $ \cont_{ \mathrm{c} } ( \flattorus ) = \cont ( \flattorus ) $.
		
		\item
		For a locally integrable function $ u \colon ( 0 , T ) \times \Omega $ 
		with $ \Omega \subseteq \mathbb{ R }^{ d } $ being some open set or the 
		flat torus, we denote the total variation in space for a time $ t \in ( 
		0 , T ) $ by
		\begin{equation*}
			\abs{ \nabla u ( t , \cdot ) }
			\coloneqq
			\sup 
			\left\{
				\int_{ \Omega }
					u ( t , x ) \divg ( \xi ) ( x ) 
				\dd{ x }
				\, \colon \,
				\xi \in \cont_{ \mathrm{c} }^{ 1 } ( \Omega ; \mathbb{ R }^{ d 
				} ) , \abs{ \xi } \leq 1
			\right\}
		\end{equation*}	
		and the total variation in space taken both over time and space by
		\begin{equation*}
			\abs{ \nabla u }_{ d + 1 }
			\coloneqq
			\left\{
			\int_{ ( 0 , T ) \times \Omega }
			u ( t , x ) \divg_{ x } ( \xi ) (t , x ) 
			\dd{ x }
			\, \colon \,
			\xi \in \cont_{ \mathrm{c} }^{ 1 } (( 0 , T ) \times \Omega ; 
			\mathbb{ R }^{ d 
			} ) , \abs{ \xi } \leq 1
			\right\}.
		\end{equation*}
	\end{itemize}
