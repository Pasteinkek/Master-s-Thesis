\section{Structure of the equation}

This chapter follows \cite{convergence_of_allen_cahn_equation_to_multiphase_mean_curvature_flow}, but since the authors decided to only sketch some of the proofs, we want to go into more detail.

Let $ \Lambda > 0 $ and define the flat torus 
$ \flattorus = [0, \Lambda )^{ d } \subset \mathbb{ R }^{ d } $, 
where we work with periodic boundary conditions and write $ \int \dd{x} $ instead of $ \int_{ \flattorus } \dd{x} $.
Then for 
$ u \colon [ 0 , \infty ) \times \flattorus \to \mathbb{ R }^{ N } $ 
and some potential 
$ W \colon \mathbb{ R }^{ N } \to [0, \infty ) $,
the \emph{Allen--Cahn equation} with parameter $ \varepsilon > 0 $ is given by
\begin{equation}
	\label{allen_cahn_eq}
	\partial_{ t } u 
	=
	\Delta u - \frac{1 }{ \varepsilon^{ 2 } } \nabla W ( u ).
\end{equation}

To understand this equation better, we consider the \emph{Cahn--Hilliard 
energy}, which assigns to $ u $ for a fixed time the real number
\begin{equation}
	\label{cahn_hilliard_energy}
	\energy_{ \varepsilon } 
	(u)
	\coloneqq
	\int
	\frac{ 1 }{ \varepsilon }
	W ( u )
	+
	\frac{ \varepsilon }{ 2 }
	\abs{ \nabla u }^{ 2 }
	\dd{x}.
\end{equation}

If everything is nice and smooth, we can compute that under the assumption that $ u $ satisfies 
equation (\ref{allen_cahn_eq}), we have that
\begin{align*}
	\dv{t} \energy_{ \varepsilon } ( u )
	&=
	\int
	\frac{ 1 }{ \varepsilon }
	\inner*{\nabla W ( u )}{ \partial_{ t } u}
	+
	\varepsilon
	\inner*{\nabla u}{ \nabla \partial_{ t } u}
	\dd{x}
	\\
	&=
	\int
	\inner*{ \frac{ 1 }{\varepsilon } \nabla W ( u ) - \varepsilon \Delta u  }{ \partial_{ t } u }
	\dd{x}
	\\
	&=
	\int - \varepsilon \abs{ \partial_{ t } u }^{ 2 } \dd{ x }
	\tag{\ref{allen_cahn_eq}}.
\end{align*}

This calculation suggests that equation (\ref{allen_cahn_eq}) is the $ \lp^{ 2 
} $ gradient flow (rescaled by $ \sqrt{\varepsilon} $) of the Cahn--Hilliard 
energy. Thus we can try to construct a solution to the PDE 
(\ref{allen_cahn_eq}) via De Giorgis minimizing movements scheme 
(\cite{de_giorgi_new_problems_on_minimizing_movements}), which we will do in 
\Cref{existence_of_ac_solution}.

But first we need to clarify what our potential $ W $ should look like. Classic 
examples in the scalar case are given by $ W ( u ) = \left( u^{ 2 } - 1 
\right)^{ 2 } $ or $ W( u ) = u^{ 2 } ( u - 1 )^{ 2 } $, and we call functions 
like these \emph{doublewell potentials}, see also 
\Cref{graph_of_doublewell_potential}.
TIM MEINTE AUCH FÜR HÖHERE DIMENSION LEVEL SET GRAFIK ERSTELLEN:

\begin{figure}[ht]
	\centering
	\begin{tikzpicture}
		\label{Plot of the doublewell potential $ W(u) = (u^2 - 1 )^{ 2 } $}
		\begin{axis}[
			axis lines = left,
			xlabel = \(u\),
			ylabel = {\(W(u)= (u^2 - 1)^{ 2 }\)},
			]
			%Below the red parabola is defined
			\addplot [
			domain=-2:2, 
			samples=100, 
			color=black,
			]
			{x^4-2*x^2 + 1};
		\end{axis}
	\end{tikzpicture}
	\caption{The graph of a doublewell potential}
	\label{graph_of_doublewell_potential}
\end{figure}

In higher dimensions, we want to accept the following potentials: $ W \colon 
\mathbb{ R }^{ N } \to [0, \infty ) $ has to be a a smooth multiwell potential, 
meaning that it has finitely many zeros at $ u = \alpha_{ 1 }, \dotsc , 
\alpha_{ P } \in \mathbb{ R }^{ N } $. Furthermore we aks for polynomial growth 
in the sense that there exists some $ p \geq 2 $ such that
\begin{equation}
	\label{polynomial_growth}
	\abs{ u }^{ p } \lesssim W(u) \lesssim \abs{ u }^{ p }
\end{equation}
and
\begin{equation}
	\label{polynomial_growth_derivative}
	\abs{ \nabla W ( u ) } \lesssim \abs{ u }^{ p -1 }
\end{equation}
for all $ u $ sufficiently large. Lastly we want $ W $ to be convex up to a small perturbation in the sense that there exist smooth functions 
$ W_{ \mathrm{conv} }$, $ W_{ \mathrm{pert} } \colon \mathbb{ R }^{ N } \to [ 0 , \infty ) $ such that
\begin{equation}
	\label{decomposition_of_w}
	W = W_{ \mathrm{conv}} + W_{ \mathrm{pert}}\, ,
\end{equation}
$ W_{ \mathrm{conv} } $ is convex and
\begin{equation}
	\label{perturbation bound}
	\sup_{ x \in \mathbb{ R }^{ N } }
	\abs{ \nabla^{ 2 } W_{ \mathrm{pert} } } < \infty.
\end{equation}
These assumptions are in particular satisfied by our two examples for doublewell potentials and therefore seem to be plausible.

As it is custom for parabolic PDEs, we view solutions of the Allen-Cahn equation (\ref{allen_cahn_eq}) as maps from $ [0,T] $ into some suitable function space and thus use the following definition.

\begin{definition}
	\label{solution_to_ac}
	We say that a function 
	$ u_{ \varepsilon} \in 
	\mathrm{ C } \left( [0 , T ] ; \lp^{ 2 } \left( \flattorus; \mathbb{ R }^{ N } \right) \right) $
	which is also in
	$\lp^{ \infty } \left( [0, T ]; \wkp^{ 1, 2 } ( \flattorus ; \mathbb{ R }^{ N } ) \right)
	$
	is a weak solution of the Allen--Cahn equation (\ref{allen_cahn_eq}) with parameter $ \varepsilon > 0 $ and initial condition $ u_{ \varepsilon}^{ 0 } \in \lp^{ 2 } ( \flattorus ; \mathbb{ R }^{ N } ) $ if
	\begin{enumerate}
		\item the energy stays bounded, which means that
		\begin{equation}
			\esssup_{ 0 \leq t \leq T }
			\energy_{ \varepsilon } ( u_{ \varepsilon } ( t ) ) 
			< \infty,
		\end{equation}
		\item 
		its weak time derivative satisfies
		\begin{equation}
			\partial_{ t } u_{ \varepsilon }
			\in
			\lp^{ 2 } \left( [ 0 , T ] \times \flattorus ; \mathbb{ R }^{ N } \right),
		\end{equation}
		\item 
		for almost every $ t \in [ 0 , T ] $ and every 
		$ \xi \in \lp^{ p } ( [ 0 , T ] \times \flattorus; \mathbb{ R }^{ N } ) 
		\cap
		\wkp^{ 1, 2 } ( [ 0, T ] \times \flattorus ; \mathbb{ R }^{ N } ) $,
		we have
		\begin{equation}
			\label{ac_weak_equation}
			\int
			\inner*{ \frac{ 1 }{ \varepsilon^{ 2 } } \nabla W ( u_{ \varepsilon} ( t ) ) }{ \xi }
			+
			\inner*{ \nabla u_{ \varepsilon } ( t ) }{ \nabla \xi } 
			+
			\inner*{\partial_{ t } u_{ \varepsilon} ( t ) }{ \xi }
			\dd{x}
			=
			0,
		\end{equation}
		\item 
		the initial conditions are achieved in the sense that $ u_{ \varepsilon } ( 0 ) = u_{ \varepsilon}^{ 0 } $.
	\end{enumerate}
\end{definition}

\begin{remark}
	Given our assumptions, we automatically have $ \nabla W ( u_{ \varepsilon } ) ( t ) \in \lp^{ p' } ( \flattorus ) $ since
	\begin{equation}
		\abs{ \nabla W ( u_{ \varepsilon} ) }^{ p/(p-1) }
		\lesssim
		1+ \abs{u_{ \varepsilon }}^{ p }
		\lesssim
		1 + W ( u ),
	\end{equation}
	which is integrable for almost every time $ t $ since we assume that the energy stays bounded,
	thus the integral in equation (\ref{ac_weak_equation}) is well defined.
	
	Moreover we already obtain $ 1/2 $ Hölder-continuity in time from the embedding
	\begin{equation}
		\label{w12_embeds_into_c1half}
		\wkp^{ 1 , 2} \left( [ 0 , T ] ; \lp^{ 2 } ( \flattorus;\mathbb{ R }^{ 
		N } ) \right)
		\hookrightarrow
		\cont^{ 1/2 } \left( [ 0 , T ] ; \lp^{ 2 } ( \flattorus ; \mathbb{ R 
		}^{ N } ) \right),
	\end{equation}
	which follows from a generalized version of the fundamental theorem of calculus and Hölder's inequality.
\end{remark}