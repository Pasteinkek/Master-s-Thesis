\section{Mean curvature flow}
	\label{section_mcf}

Mean curvature flow describes the geometric evolution of a set $ ( \Omega ( t ) 
)_{ t \geq 0 } $ respectively the evolution of its boundary $ \Sigma ( t ) 
\coloneqq \partial \Omega ( t ) $. It is formulated through the equation
\begin{equation}
	\label{mcf_twophase_basic_equation}
	\frac{ 1 }{ \mu } V = - \sigma H 
	\quad
	\text{on }
	\Sigma.
\end{equation}
Here $ \mu > 0 $ is a positive constant which is called the \emph{mobility} and 
$ \sigma > 0 $ is a positive constant as well which we define as the 
\emph{surface tension}. By $ V $ we denote the normal velocity of the set and 
by $ H $ its mean curvature, which is defined as the sum of the principle 
curvatures at a given point.
There are a lot of different notions of solutions to this equation as already 
mentioned in the introduction.
For us the central structure of this equation is its gradient flow structure. Formally we want to consider the space
\begin{equation*}
	\mathcal{ M }
	\coloneqq
	\left\{
		\text{hypersurfaces in} \flattorus
	\right\},
\end{equation*}
where the tangent space at a given $ \Sigma \in \mathcal{ M } $ consists of the normal velocities on $ \Sigma $.
The metric tensor at a surface $ \Sigma $ of two normal velocities is then 
given by the rescaled $ \lp^{ 2 } $-inner product on $ \Sigma $
\begin{equation}
	\label{definition_of_metric_twophase}
	\inner*{ V }{ W }_{ \Sigma }
	\coloneqq
	\frac{ 1 }{ \mu }
	\int_{ \Sigma }
		V W
	\dd{ \hm^{ d - 1 } }
\end{equation}
and our energy will simply be the rescaled perimeter functional
\begin{equation*}
	\energy ( \Sigma )
	\coloneqq
	\sigma \hm^{ d -1 } ( \Sigma ).
\end{equation*}
By \cite[Thm.~17.5]{maggi_sets_of_finite_perimeter} the first inner variation 
of the perimeter functional is given by the mean curvature vector. 
Remember moreover that by the definition of the metric 
(\ref{definition_of_metric_twophase}) the gradient of the energy at a given 
hypersurface $ \Sigma $ has to satisfy
\begin{equation*}
	\frac{ 1 }{ \mu }
	\int_{ \Sigma }
		\inner*{ \nabla_{ \Sigma } \energy }{ V }
	\dd{ \hm^{ d - 1 } }
	=
	\dd_{ \Sigma }
	\energy ( V )
\end{equation*} 
for all normal vector fields $ V $ on $ \Sigma $.
Combining these arguments the gradient of the energy should simply be the mean 
curvature vector multiplied by $ \sigma \mu $.
Thus the mean curvature flow equation (\ref{mcf_twophase_basic_equation}) 
corresponds exactly to the gradient flow equation (\ref{gf_basic_equation}).
Note however that this metric tensor induces a degenerate metric in the sense 
that the distance between any two hypersurfaces is zero, which has been shown 
by Michor and Mumford 
in \cite{michor_mumford_riemannian_geometries_on_spaces_of_plane_curves}. 
To give some intuition for this result, we can roughen up a hypersurface, then 
move it, and 
afterwards flatten 
it back, while only producing an arbitrarily small distance due to a scaling 
invariance.

In \Cref{section_gradient_flows} we highlighted the importance of
De Giorgi's optimal energy dissipation inequality (\ref{basic_optimal_energy_dissipation_inequality}). In our setting  this translates to the inequality
\begin{align*}
	& \energy ( \Sigma ( T ) )
	+
	\frac{ 1 }{ 2 }
	\int_{ 0 }^{ T }
		\inner*{ V ( t ) }{ V ( t ) }_{ \Sigma ( t ) }
		+
		\inner*{ \nabla_{ \Sigma ( t ) } E }{ \nabla_{ \Sigma ( t ) } E }_{ 
		\Sigma ( t ) }
	\dd{ t }
	\\
	={}
	&\energy ( \Sigma ( T ) )
	+
	\frac{ 1 }{ 2 }
	\int_{ 0 }^{ T }
		\int_{ \Sigma (t ) }
			\frac{ 1 }{ \mu }
			V( t )^{ 2 } 
			+
			\sigma^{ 2 } \mu 
			H( t ) ^{ 2 }
		\dd{ \hm^{ d - 1 } }
	\dd{ t }
	\\
	\leq{} &
	\energy ( \Sigma ( 0 ) ).
\end{align*}
This is the main motivation for the definition of De Giorgi type solutions to 
mean curvature flow in the two-phase case, see 
\Cref{de_giorgi_solution_to_mmcf} and 
\Cref{de_giorgi_varifold_solution_for_mcf}.

Far more important for us shall however be multiphase mean curvature flow, 
which is of high importance in the applied sciences and mathematically quite 
complex. 
Essentially instead of just considering the evolution of one single set, we 
look at the evolution of a partition of the flat torus and require that at each 
interface of the sets, the mean curvature flow equation 
(\ref{mcf_twophase_basic_equation}) is satisfied. Moreover we require a 
stability condition at points where three interfaces meet.

More precisely we say that a partition $ ( \Omega_{ i } ( t ) )_{ i = 1 , 
\dotsc , P } $ with 
$ \Sigma_{ i j } \coloneqq \partial \Omega_{ i } \cap \partial \Omega_{ j } $ 
satisfies multiphase mean curvature flow with mobilities $ \mu_{ i j } $ and 
surface 
tensions $ \sigma_{ i j } $ if 
\begin{align}
		\label{v_is_equal_to_h}
		\frac{ 1 }{ \mu_{ i j } }
		V_{ i j }
		& =
		-
		\sigma_{ i  j }
		H_{ i j }
		\quad
		&\text{on } \Sigma_{i j }\text{ for all }i\neq j \text{ and}
		\\
		\label{herrings_angle_condition}
			\sigma_{ i j }
			\nu_{ i j }
			+
			\sigma_{ j k }
			\nu_{ j k }
			+
			\sigma_{ k i }
			\nu_{ k i }
		& =
		0
		& \text{at triple junctions}.
\end{align}
The second equation is a stability condition when more than two sets meet and is
called \emph{Herring's angle condition}. Here $ \nu_{ i j } $ is the outer unit 
normal of $ \Omega_{ i } $
on $ \Sigma_{ i j } $ pointing towards $ \Omega_{ j } $. 
One could argue that we would also want stability conditions at for example 
quadruple 
junctions. In two dimensions though, such quadruple junctions are expected to 
immediately dissipate. In higher dimensions, quadruple junctions might become 
stable, but only on lower dimensional sets, and shall thus not be relevant for 
us.

As our space, we shall thus now consider tuples of hypersurfaces in $ 
\flattorus $. 
The energy and metric tensor are given by
\begin{align}
	\label{definition_of_multiphase_energy}
	\energy ( \Sigma )
	&\coloneqq
	\sum_{ i < j }
		\sigma_{ i j }
		\hm^{ d- 1 } ( \Sigma_{ i j } )
	\shortintertext{and}
	\notag
	\inner*{ V }{ W }_{ \Sigma }
	& \coloneqq
	\sum_{ i < j }
		\frac{ 1 }{ \mu_{ i j } }
		\int_{ \Sigma_{ i j } }
			V_{ i j }
			W_{ i j }
		\dd{ \hm^{ d - 1 } }.
\end{align}
Of course this again produces a degenerate metric.
Using a variant of the divergence theorem on surfaces 
(\cite[Thm.~11.8]{maggi_sets_of_finite_perimeter}) and again the computation 
for the first variation of the perimeter 
(\cite[Thm.~17.5]{maggi_sets_of_finite_perimeter}), we see that multiphase mean 
curvature has the desired gradient flow structure.
In this case De Giorgi's 
inequality (\ref{basic_optimal_energy_dissipation_inequality}) translates to
\begin{equation*}
	\energy ( \Sigma ( T ) )
	+
	\frac{ 1 }{ 2 }
	\sum_{ i < j }
		\int_{ 0 }^{ T }
			\int_{ \Sigma_{ i j } }
				\frac{ 1 }{ \mu_{ i j } }
				V_{ i j }^{ 2 }
				+
				\sigma_{ i j }^{ 2 } \mu_{ i j }
				H_{ i j }^{ 2 }
			\dd{ \hm^{ d - 1 } }
		\dd{ t }
	\leq
	\energy ( \Sigma ( 0 ) ).
\end{equation*}
Furthermore if we make the simplifying assumption that the mobilities are 
already determined through the surface tensions by the relation $ \mu_{ i j } = 
1 / \sigma_{ i j } $, then this inequality becomes
\begin{equation*}
	\energy ( \Sigma ( T ) )
	+
	\frac{ 1 }{ 2 }
	\sum_{ i < j }
	\sigma_{ i j }
	\int_{ 0 }^{ T }
	\int_{ \Sigma_{ i j } }
	V_{ i j }^{ 2 }
	+
	H_{ i j }^{ 2 }
	\dd{ \hm^{ d - 1 } }
	\dd{ t }
	\leq
	\energy ( \Sigma ( 0 ) ),
\end{equation*}
which motivates \Cref{de_giorgi_solution_to_mmcf} and 
\Cref{de_giorgi_varifold_solutions_for_mmcf}.