\section{Conditional convergence in the two-phase case}

\subsection{Convergence to a $\bv$-solution to two-phase mean curvature flow}

Convergence of energies often boosts our 
modulus of convergence and gives our limit therefore additional regularity. 
Thus let us assume 
\begin{equation}
	\label{energy_convergence}
	\int_{ 0 }^{ T }
	\energy_{ \varepsilon } ( u_{ \varepsilon } ) 
	\dd{ t }
	\to 
	\int_{ 0 }^{ T }
	\energy ( u )
	\dd{ t }
	\text{ as }
	\varepsilon \to 0,
\end{equation}
where in the two-phase case, the surface tension energy is defined for $ u = 
\alpha ( 1 - \chi ) + 
\beta \chi $ by 
\begin{equation}
	\energy ( u ) 
	\coloneqq
	\energy ( \chi )
	\coloneqq
	\sigma \int \abs{ \nabla \chi }
\end{equation}
already motivated in \Cref{section_mcf}. We will go into more depth about why 
this assumption is important in 
\Cref{subsection_de_giorgi_type_varifold_solutions_for_mcf}.
Moreover we notice that the energy of $ u $ is exactly the total variation of $ 
\psi = \sigma \chi $.
Since for almost every time $ t $, we have that $ \psi_{ \varepsilon } ( t ) $ 
converges to $ \psi ( t ) $ in $ \lp^{ 1 }  ( \flattorus ) $ by 
\Cref{initial_convergence_result}, it follows from the lower semicontinuity of 
the variation measure and Young's inequality that
\begin{equation*}
	\energy ( u ) 
	\leq
	\liminf_{ \varepsilon \to 0 }
	\int
	\abs{ \nabla \psi_{ \varepsilon } }
	\dd{ x }
	\leq
	\liminf_{ \varepsilon \to 0 }
	\energy_{ \varepsilon } ( u_{ \varepsilon } ).
\end{equation*}
Moreover by the energy dissipation inequality (\ref{energy_dissipation_sharp}), 
the energies stay uniformly bounded in time. Thus using the dominated 
convergence theorem, we see that our time integrated energy convergence 
assumption is equivalent to saying that for almost every time $ t \in ( 0, T ) 
$, we have convergence of the energies $ \energy_{ \varepsilon } ( u_{ 
\varepsilon } ( t ) ) \to \energy ( u ( t ) ) $.

Since the energies themselves can be interpreted as measures on the flat torus, 
we define for a continuous function $ \varphi \in \cont ( \flattorus ) $ the 
corresponding energy measures by
\begin{align*}
	& \energy_{ \varepsilon } ( u_{ \varepsilon} ; \varphi )
	\coloneqq
	\int
	\varphi
	\left(
	\frac{ 1 }{ \varepsilon }
	W ( u_{ \varepsilon } )
	+
	\frac{ \varepsilon }{ 2 }
	\abs{ \nabla u_{ \varepsilon } }^{ 2 }
	\right)
	\dd{x } 
	\text{ and }
	\\
	& \energy ( u ; \varphi )
	\coloneqq
	\sigma
	\int
	\varphi
	\abs{ \nabla \chi }
	=
	\int
		\varphi
	\abs{ \nabla \psi }.
\end{align*}
Then from the energy convergence and lower semicontinuity just discussed, it 
follows that 
\begin{equation}
	\label{convergence_of_energy_measures}
	\lim_{ \varepsilon \to 0 }
	\energy_{ \varepsilon } ( u_{ \varepsilon } ; \varphi )
	=
	\energy ( u ; \varphi ).
\end{equation}

We start by defining a BV-formulation for motion by mean curvature in the 
spirit of Luckhaus und Sturzenhecker 
\cite{luckhaus_sturzenhecker_implicit_time_discretization_for_mcf}.

\begin{definition}[$\bv$-solution to two-phase mean curvature flow]
	\label{motion_by_mcv}
	Fix some finite time horizon $ T < \infty $ and initial data $ \chi^{ 0 
	}\in \bv \left( \flattorus ; \{ 0 , 1 \} \right) $. We say that 
	\begin{equation*}
		\chi \in 
		\cont \left(
		[ 0 , T ] ; \lp^{ 2 } \left( \flattorus ; \{ 0 , 1 \}  \right)
		\right)
	\end{equation*}
	with $ \esssup_{ 0 \leq t \leq T } \energy ( \chi ) < \infty $ \emph{moves 
	by mean 
	curvature with initial data } $ \chi^{ 0 } $ if there is a normal velocity
	$ V \in \lp^{ 2 } \left( \abs{ \nabla \chi } \dd{ t } \right) $ such that 
	\begin{enumerate}
		\item 
		For all 
		$ \xi \in \cont_{ \mathrm{ c } }^{ \infty } \left( ( 0 , T ) \times 
		\flattorus ; \mathbb{ R }^{ d } \right) $,
		we have
		\begin{equation}
			\label{integral_formulation_of_mcf}
			\int_{ 0 }^{ T }
			\int
			V \inner*{ \xi }{ \nu }
			- 
			\inner*{ \diff \xi }{ \mathrm{ Id } - \nu \otimes \nu }
			\abs{ \nabla \chi }
			\dd{ t }
			=
			0,
		\end{equation}
		where $ \nu \coloneqq \nabla \chi / \abs{ \nabla \chi } $ is the inner 
		unit normal.
		\item 
		The function $ V $ is the normal velocity of $ \chi $ in the sense that 
		\begin{equation*}
			\partial_{ t } \chi
			=
			V
			\abs{ \nabla \chi }
			\dd{ t }
		\end{equation*}
		holds distributionally in $ ( 0 , T ) \times \flattorus $.
		\item 
		The initial data $ \chi^{ 0 } $ is achieved in $ \cont \left( [ 0 , T ] 
		; \lp^{  2 } ( \flattorus ) \right) $, which means that $ \chi ( 
		0 ) = \chi^{ 0 } $ as functions in $ \lp^{ 2 } ( \flattorus ) $.
	\end{enumerate}
\end{definition}

Let us give a brief motivation for this definition. First we note that the 
boundedness of energies already implies that $ \chi $ is of bounded variation 
in space for almost every time. Moreover equation 
(\ref{integral_formulation_of_mcf}) hinges on the distributional formulation 
for the mean curvature vector as stated in 
\cite{maggi_sets_of_finite_perimeter} and yields that
\begin{equation*}
	\int
		\inner*{ \diff \xi }{ \mathrm{Id} - \nu \otimes \nu }
	\abs{ \nabla \chi }
	=
	-
	\int
		H \inner*{ \xi }{ \nu }
	\abs{ \nabla \chi }.
\end{equation*}
Thus the equation is equivalent to $ H = - V $ on $ \Sigma $.

Our main goal in this section is now to show that the function $ \chi $ we have 
found in \Cref{initial_convergence_result} moves my mean curvature, assuming 
the time integrated energies converge (\ref{energy_convergence}). Thus our 
goal is to prove the following Theorem.
\begin{theorem}
	\label{convergence_to_twophase_mcf}
	Let a smooth double-well potential $ W \colon \mathbb{ R } \to [ 0, \infty 
	) 
	$ satisfy the assumptions (\ref{polynomial_growth})-(\ref{perturbation 
		bound}). Let $ T < \infty $ be an arbitrary finite time horizon. Given 
		a 
	sequence of initial data $ u_{ \varepsilon }^{ 0 } \colon \flattorus \to 
	\mathbb{ R } $ with $ \chi^{ 0 } \in \bv \left( \flattorus ; \{ 0 , 1 \} 
	\right) 
	$ such that $ u_{ \varepsilon }^{ 0 } $ converges to $ u^{ 0 } = ( 1- 
	\chi^{ 0} ) \alpha 
	+ \chi^{ 0 } \beta $ pointwise almost everyhwere and
	\begin{equation*}
		\energy_{ 0 }
		\coloneqq
		\energy ( \chi^{ 0 } )
		=
		\lim_{ \varepsilon \to 0 }
		\energy_{ \varepsilon } ( u_{ \varepsilon}^{ 0 } )
		< 
		\infty,
	\end{equation*}
	we have that that for some subsequence of solutions $ u_{ \varepsilon } $ 
	to 
	(\ref{allen_cahn_eq}), there exists a pointwise 
	almost everywhere limit $ u = ( 1 - \chi ) \alpha + \chi \beta $ with $ 
	\chi \in \bv \left( ( 0, T ) \times \flattorus ; \{ 0 , 1 \} \right) $ and
	$ \chi \in \cont \left( [ 0 , T ] ; \lp^{ 2 } (\flattorus ; \{ 0 , 1 \} ) 
	\right) $ which assumes the initial data in $ \cont \left( [ 0, T ] ; 
	\lp^{ 2 }( \flattorus ) \right) $. If we additionally assume that the 
	time-integrated energies converge (\ref{energy_convergence}), then $ \chi 
	$ moves by mean curvature in the sense of \Cref{motion_by_mcv}.
\end{theorem} 
Looking at the distributional form of the Allen--Cahn equation 
(\ref{ac_weak_equation}), which reads
\begin{equation*}
	\int
	\frac{ 1 }{ \varepsilon^{ 2 } } W' ( u_{ \varepsilon} ( t ) ) \varphi
	+
	\inner*{ \nabla u_{ \varepsilon } ( t ) }{\nabla \varphi }
	+
	\partial_{ t } u_{ \varepsilon} ( t ) \varphi 
	\dd{x}
	=
	0,
\end{equation*}
we expect that for a suitable choice of testfunctions $ \varphi_{ \varepsilon } 
$, the following two terms converge:
\begin{align*}
	& \lim_{ \varepsilon \to 0 }
	\int_{ 0 }^{ T }
	\int 
	\partial_{ t } u_{ \varepsilon } \varphi_{ \varepsilon }
	=
	\sigma
	\int_{ 0 }^{ T }
	\int
	V \inner*{ \xi }{ \nu }
	\abs{ \nabla \chi }
	\dd{ t },
	\\
	& \lim_{ \varepsilon \to 0 }
	\int_{ 0 }^{ T }
	\int
	\frac{1 }{ \varepsilon^{ 2 } }
	W'( u_{ \varepsilon } )
	\varphi_{ \varepsilon}
	+ 
	\inner*{ \nabla u_{ \varepsilon } }{ \nabla \varphi_{ \varepsilon } }
	\dd{ x }
	\dd{ t }
	=
	- 
	\sigma \int_{ 0 }^{ T }
	\int
	\inner*{ \diff \xi }{ \mathrm{Id} - \nu \otimes \nu }
	\abs{ \nabla \chi }
	\dd{ t }.
\end{align*}
But how do we find these testfunctions? For this, we first note that the 
curvature term
$ \int \inner*{ \diff \xi }{ \mathrm{Id} - \nu \otimes \nu } \abs{ \nabla \chi 
} \dd{ t } $ is by \cite[Thm.~17.5]{maggi_sets_of_finite_perimeter} the first 
inner variation with respect to $ \xi $ of the perimeter functional, which is 
just our energy $ \energy $ up to the surface tension constant $ \sigma > 0 $. 
Since the energy $ \energy_{ \varepsilon } $ converges to $ \energy $, we may 
also hope that their first variations converge to each other. Thus it is 
plausible to compute the first inner variation 
$ \left.\dv{ s } \right|_{ s = 0 } \energy_{ \varepsilon } ( \rho_{ s } ) $ and 
then we can hopefully choose the testfunction $ \varphi_{ \varepsilon } $ in 
such a way that it equals
$ 
\int
1 / \varepsilon^{ 2 }  W' ( u_{ \varepsilon} ) \varphi
+
\inner*{ \nabla u_{ \varepsilon } }{ \nabla \varphi }
\dd{x}
$.

Thus let $ ( \rho_{ s } )_{ s } $ be functions which solve the ordinary 
differential equation
\[
\begin{cases}
	\partial_{ s } \rho_{ s } 
	+
	\inner*{ \xi }{ \nabla \rho_{ s } }
	& = 0,
	\\
	\rho_{ 0 } & = u_{ \varepsilon }.
\end{cases}
\]
Then we formally compute that the first inner variation of the energy $ 
\energy_{ \varepsilon } $ with respect to $ \xi $ is given by
\begin{align*}
	\left.\dv{ s }\right|_{ s = 0 }
	\int
	\frac{ \varepsilon }{ 2 }
	\abs{ \rho_{ s } }^{ 2 }
	+
	\frac{ 1 }{ \varepsilon }
	W ( \rho_{ s } )
	\dd{ x }
	& =
	\int
	\varepsilon 
	\inner*{ \nabla u_{ \varepsilon } }
	{ \nabla \left( - \inner*{ \xi }{ \nabla u_{ \varepsilon } } \right) }
	+
	\frac{ 1 }{ \varepsilon }
	W'( u_{ \varepsilon } ) ( -\inner*{ \xi }{ \nabla u_{ \varepsilon } } )
	\dd{ x }
	\\
	& =
	\int
	\left(
	\varepsilon \Delta u - \frac{ 1 }{ \varepsilon } W'( u_{ \varepsilon } )
	\right)
	\inner*{ \xi }{ \nabla u_{ \varepsilon } } 
	\dd{ x }.
\end{align*}
We therefore test equation (\ref{ac_weak_equation}) with $ \varphi_{ 
\varepsilon } \coloneqq \inner{ \xi }{ \nabla u_{ \varepsilon } } $.

A first additional regularity result under the energy convergence assumption is 
the following Proposition, which ensures that we have a square-integrable 
normal velocity.

\begin{proposition}
	\label{existence_of_velocity_twophase}
	In the setting of \Cref{initial_convergence_result} and given the energy 
	convergence assumption (\ref{energy_convergence}), the measure $ \partial_{ 
		t } \chi $ is absolutely continuous with respect to the measure $ \abs{ 
		\nabla \chi } \dd{ t } $ and the corresponding density $ V $ is square 
	integrable with the estimate
	\begin{equation*}
		\int_{ 0 }^{ T }
		\int
		V^{ 2 }
		\abs{ \nabla \chi }
		\dd{ t }
		\lesssim
		\energy_{ 0 }.
	\end{equation*}
\end{proposition}

\begin{remark}
	By 
	\cite[Thm.~3.103]{ambrosio_fusco_pallara_functions_of_bv_and_free_discontinuity_problems},
	we can disintegrate the measure $ \abs{ \nabla \chi }_{ d + 1 } $, which 
	is the variation in space in time and space, into the measure $ \abs{ 
		\nabla \chi }_{ d } \dd{ t } $, which we will use by a slight abuse of 
	notation and drop the subindex depending on the situation at hand. Here $ 
	\abs{ \nabla \chi }_{ d } $ simply denotes the variation in space for a 
	fixed time.
\end{remark}

\begin{proof}
	Take a smooth test function $ \varphi \in \cont_{ \mathrm{ c } }^{ \infty } 
	( ( 0 , T ) \times \flattorus )$. 
	Then via the $ \lp^{ 1 } $-convergence of $ \psi_{ \varepsilon } $ to $ 
	\psi 
	$, we have
	\begin{align*}
		\partial_{ t } \psi ( \varphi )
		& =
		\liminf_{ \varepsilon \to 0 }
		\partial_{ t } \psi_{ \varepsilon } ( \varphi )
		\\
		& =
		\liminf_{ \varepsilon \to 0 }
		\int_{ 0 }^{ T }
		\int
		\sqrt{ 2 W ( u_{ \varepsilon } ) }
		\partial_{ t } u_{ \varepsilon }
		\varphi
		\dd{ x }
		\dd{ t }
		\\
		& \leq
		\liminf_{ \varepsilon \to 0 }
		\left( 
		\int_{ 0 }^{ T }
		\int
		\frac{ 1 }{ \varepsilon } 2 W ( u_{ \varepsilon } )
		\varphi^{ 2 }
		\dd{ x }
		\dd{ t }
		\right)^{ 1/ 2 }
		\left(
		\int_{ 0 }^{ T }
		\int
		\varepsilon
		\abs{ \partial_{ t } u_{ \varepsilon } }^{ 2 }
		\dd{ x }
		\dd{ t }
		\right)^{ 1/2 }
		\\
		& \leq
		\liminf_{ \varepsilon \to 0 }
		\left(
		2 
		\int_{ 0 }^{ T }
		\energy_{ \varepsilon} \left( u_{ \varepsilon } ; \varphi^{ 2 } \right)
		\dd{ t }
		\right)^{ 1/ 2 }
		\left(
		\energy_{ \varepsilon } ( u_{ \varepsilon } )
		\right)^{ 1/2 }
		\\
		& =
		\sqrt{ 2 \sigma }
		\norm{ \varphi }_{ \lp^{ 2 } ( ( 0 , T ) \times \flattorus , \abs{ 
				\nabla \chi } \dd{ t 
			} ) }
		\sqrt{ \energy_{ 0 } }.
	\end{align*}
	The second inequality is due to the energy dissipation inequality 
	(\ref{energy_dissipation_sharp}) and the last equality due to the 
	convergence of the localized energies 
	(\ref{convergence_of_energy_measures}).
	This proves both the absolute continuity and via a duality argument the 
	desired bound since $ \partial_{ t } \psi = \sigma \partial_{ t } \chi $ 
	and $ \sigma > 0 $.
\end{proof}

We finish this section with a proof for the \emph{equipartition of the 
	energyies}, which 
tells us that both the summand involving the potential $ W ( u_{ \varepsilon } 
) $ and the norm of the gradient contribute to the energy in equal parts. Or in 
other words
\begin{equation*}
	\frac{ 1 }{ \varepsilon } W ( u_{ \varepsilon } ) 
	- 
	\frac{ \varepsilon }{ 2 }
	\abs{ \nabla u_{ \varepsilon } }^{ 2 }
	\rightharpoonup^{ \ast }
	0 
\end{equation*}
holds in the distributional sense. Notice that under the 
energy convergence assumption (\ref{energy_convergence}), the proof is quite 
simple. Ilmanen actually showed in 
\cite{ilmanen_convergence_of_ac_to_brakkes_mcf} that in the two-phase case this 
is true
for a big class of well-prepared initial conditions, but the proof relies on a 
comparison principle, which has no obvious substitute in the multiphase case 
and the proof is much more involved.

\begin{lemma}
	\label{equipartition_of_energies}
	In the situation of \Cref{initial_convergence_result} and under the energy 
	convergence assumption (\ref{energy_convergence}), we have 
	for any continuous function $ \varphi \in \cont ( \flattorus ) $ 
	that
	\begin{align*}
		\energy ( u ; \varphi )
		=
		\lim_{ \varepsilon \to 0 }
		\energy_{ \varepsilon } ( u_{ \varepsilon } ; \varphi )
		& = 
		\lim_{ \varepsilon \to 0 }
		\int
		\varphi
		\sqrt{ 2 W ( u_{ \varepsilon } ) }
		\abs{ \nabla u_{ \varepsilon } }
		\dd{ x }
		\\
		& =
		\lim_{ \varepsilon \to 0 }
		\int
		\varphi
		\varepsilon
		\abs{ \nabla u_{ \varepsilon } }^{ 2 }
		\dd{ x}
		\\
		& =
		\lim_{ \varepsilon \to 0 }
		\int
		\varphi
		\frac{ 1 }{ \varepsilon }
		2 W ( u_{ \varepsilon } )
		\dd{ x }
	\end{align*}
	for almost every time $ 0 \leq t \leq T $.
\end{lemma}

\begin{proof}
	We have already established the first equality before. For the second 
	equality, we first assume that $ \varphi \in \cont \left( \flattorus  
	\right) $ is non-negative.
	By the lower semicontinuity of the variation measure, we immediately obtain
	\begin{equation*}
		\liminf_{ \varepsilon \to 0 }
		\int
		\varphi
		\sqrt{ 2 W ( u_{ \varepsilon } ) }
		\abs{ \nabla u_{ \varepsilon } }
		\dd{ x }
		=
		\liminf_{ \varepsilon \to 0 }
		\int
		\varphi
		\abs{ \nabla \psi_{ \varepsilon } }
		\dd{ x }
		\geq
		\energy ( u ; \varphi ).
	\end{equation*}
	But by Young's inequality, we also have 
	\begin{equation*}
		\limsup_{ \varepsilon \to 0 }
		\int
		\varphi
		\sqrt{ 2 W ( u_{ \varepsilon } ) }
		\abs{ \nabla u_{ \varepsilon } }
		\dd{ x }
		\leq
		\limsup_{ \varepsilon \to 0 }
		\energy_{ \varepsilon } ( u_{ \varepsilon } ; \varphi )
		= \energy ( u ; \varphi ).
	\end{equation*}
	For general $ \varphi \in \cont \left( \flattorus \right) $, we decompose $ 
	\varphi $ into its positive and negative part and apply the previous 
	argument to both in order to get the claim.
	
	The third and fourth equality follow for a given non-negative $ \varphi 
	\in \cont \left( \flattorus ; [ 0 , \infty ) \right) $ by the 
	$ \lp^{ 2 } $-estimate
	\begin{align*}
		& \lim_{ \varepsilon \to 0 }
		\int
		\abs{ \sqrt{ \varphi } \sqrt{ \varepsilon } \abs{ \nabla u_{ 
					\varepsilon } } - \sqrt{ \varphi } \frac{ 1 }{ \sqrt{ 
					\varepsilon } } 
			\sqrt{ 2 W ( u_{ \varepsilon } ) } }^{ 2 }
		\dd{ x }
		\\
		={} &
		\lim_{ \varepsilon \to 0 }
		2 \energy_{ \varepsilon } ( u_{ \varepsilon } ; \varphi )
		-
		2 \int	
		\varphi
		\sqrt{ 2 W ( u_{ \varepsilon } ) }
		\abs{ \nabla u_{ \varepsilon } }
		\dd{ x }
		=
		0
	\end{align*}
	which implies that 
	\begin{equation*}
		\lim_{ \varepsilon \to 0 }
		\int
		\varphi
		\varepsilon
		\abs{ \nabla u_{ \varepsilon } }^{ 2 }
		\dd{ x }
		=
		\lim_{ \varepsilon \to 0}
		\int
		\varphi
		\frac{ 1 }{ \varepsilon }
		2 W ( u_{ \varepsilon } )
		\dd{ x }
		=
		\lim_{ \varepsilon \to 0 }
		\energy_{ \varepsilon } ( u_{ \varepsilon } ; \varphi ),
	\end{equation*}
	since the integral of $ \varphi 1/ \varepsilon 2 W ( u_{ \varepsilon } ) + 
	\varphi \varepsilon \abs{ \nabla u_{\varepsilon } }^{ 2 } $ converges .
\end{proof}

\subsection{Convergence of the curvature term}

The goal of this section is to prove the convergence
\begin{equation*}
	\lim_{ \varepsilon \to 0 }
	\int
	\left(
	\varepsilon \Delta u_{ \varepsilon }
	- 
	\frac{ 1 }{ \varepsilon }
	W'( u_{ \varepsilon } )
	\right)
	\inner*{ \xi }{ \nabla u_{ \varepsilon } }
	\dd{ x }
	=
	\sigma
	\int
	\inner*{ \diff \xi }{ \mathrm{Id} - \nu \otimes \nu }
	\abs{ \nabla \chi }
\end{equation*} 
for almost every time $ t $. We directly follow the proof from Luckhaus and 
Modica in \cite{luckhaus_modica_gibbs_thompson_relation} with some extra steps.

Through an integration by parts, we obtain
\begin{align}
	\label{conv_of_curv_first_ibp_scalar}
	\int
	\left(
	\varepsilon \Delta u_{ \varepsilon }
	-
	\frac{ 1 }{ \varepsilon } W'(u_{ \varepsilon } ) 
	\right)
	\inner*{ \xi }{ \nabla u_{ \varepsilon } }
	= 
	\int
	& - \varepsilon 
	\sum_{ i, j = 1 }^{ d }
	\partial_{ x_{ i } } u_{ \varepsilon }
	\left(
	\partial_{ x_{ i } } \xi^{ j }
	\partial_{ x_{ j } } u_{ \varepsilon }
	+
	\xi^{ j }
	\partial_{ x_{ i } x_{ j } }^{ 2 } u_{ \varepsilon }
	\right)
	\notag
	\\
	& +
	\frac{ 1 }{ \varepsilon }
	W ( u_{ \varepsilon } )
	\divg \xi 
	\dd{ x }.
\end{align}

Moreover by another integration by parts, we have
\begin{align*}
	\int
	\sum_{ i, j = 1 }^{ d }
	\partial_{ x_{ i } } u_{ \varepsilon }
	\xi^{ j }
	\partial_{ x_{ i } x_{ j } }^{ 2 } u_{ \varepsilon }
	\dd{ x }
	& =
	\int
	- \sum_{ i, j = 1 }^{ d }
	\partial_{ x_{ i } } u_{ \varepsilon }
	\left(
	\partial_{ x_{ i } x_{ j } }^{ 2 } u_{ \varepsilon }
	\xi^{ j }
	+
	\partial_{ x_{ j } } \xi^{ j }
	\partial_{ x_{ i } } u_{ \varepsilon }
	\right)
	\dd{ x }
	\\
	& = 
	\int
	- \abs{ \nabla u_{ \varepsilon } }^{ 2 }
	\divg{ \xi }
	-
	\sum_{ i, j = 1 }^{ d }
	\partial_{ x_{ i } } u_{ \varepsilon }
	\partial_{ x_{ i } x_{ j } }^{ 2 } u_{ \varepsilon }
	\xi^{ j }
	\dd{ x }
	\shortintertext{which is equivalent to}
	\int
	\sum_{ i, j = 1 }^{ d }
	\partial_{ x_{ i } } u_{ \varepsilon }
	\xi^{ j }
	\partial_{ x_{ i } x_{ j } }^{ 2 }
	u_{ \varepsilon }
	\dd{ x }
	& =
	- \frac{ 1 }{ 2 }
	\int
	\abs{ \nabla u_{\varepsilon } }^{ 2 }
	\divg \xi 
	\dd{ x }.
\end{align*}

Plugging this equation into the first equation 
(\ref{conv_of_curv_first_ibp_scalar}), we obtain
\begin{align}
	& \int
	\left(
	\varepsilon \Delta u_{ \varepsilon }
	-
	\frac{ 1 }{ \varepsilon }
	W'( u_{ \varepsilon } ) 
	\right)
	\inner*{ \xi }{ \nabla u_{ \varepsilon } }
	\dd{ x }
	\notag
	\\
	\label{curvature_term_equality_which_bounds}
	={} &
	\int
	-\varepsilon \sum_{ i, j = 1}^{ d }
	\partial_{ x_{ i } } u_{ \varepsilon }
	\partial_{ x_{ j } } u_{ \varepsilon }
	\partial_{ x_{ i } } \xi^{ j } 
	+
	\frac{ \varepsilon }{ 2 }
	\abs{ \nabla u_{ \varepsilon } }^{ 2 }
	\divg \xi 
	+
	\frac{ 1 }{ \varepsilon }
	W ( u_{ \varepsilon } ) 
	\divg \xi 
	\dd{ x }
	\\
	= {} &
	\varepsilon
	\int
	\abs{ \nabla u_{ \varepsilon } }^{ 2 }
	\divg \xi 
	-
	\sum_{ i, j = 1 }^{ d }
	\partial_{ x_{ i } } u_{ \varepsilon }
	\partial_{ x_{ j } } u_{ \varepsilon }
	\partial_{ x_{ i } } \xi^{ j }
	\dd{ x }
	\notag
	\\
	& + 
	\int
	\frac{ 1 }{ \varepsilon }
	W ( u_{ \varepsilon } )
	\divg \xi 
	-
	\frac{ \varepsilon }{ 2 }
	\abs{ \nabla u_{ \varepsilon } }^{ 2 }
	\divg \xi
	\dd{ x }
	\notag.
\end{align}
The last integral vanishes by the equipartition of the energies 
(\Cref{equipartition_of_energies}).
Since $ \partial_{ x_{ i } } u_{ \varepsilon } / \abs{ \nabla u_{ \varepsilon } 
} = \partial_{ x_{ i } } \psi_{ \varepsilon } / \abs{ \nabla \psi_{ 
\varepsilon  } } $ by the chain rule and non-negativity of $ W $, the former 
integral can be written as
\begin{equation}
	\label{conv_of_curv_scalar_written_as_g}
	\int_{ \flattorus_{ \varepsilon } }
	g ( x , \nabla \psi_{ \varepsilon } )
	\varepsilon \abs{ \nabla u_{ \varepsilon } }^{ 2 }
	\dd{ x }.
\end{equation}
Here the function $ g $ is defined by
\begin{equation*}
	g ( x, p )
	=
	\begin{cases}
		\sum_{ i, j = 1}^{ d }
		-
		\frac{ p_{ i } }{ \abs{ p } }
		\partial_{ x_{ i } } \xi^{ j }
		\frac{ p_{ j } }{ \abs{ p } }
		+
		\divg \xi 
		\quad
		&\text{if } p \neq 0,
		\\
		0
		&\text{else},
	\end{cases}
\end{equation*}
and the set $ \flattorus_{ \varepsilon } $ is defined as
\begin{equation*}
	\flattorus_{ \varepsilon }
	\coloneqq
	\left\{
	x \in \flattorus
	\, \colon \,
	\nabla \psi_{ \varepsilon } ( x ) \neq 0
	\right\}
	=
	\left\{
	x \in \flattorus
	\, \colon \,
	\nabla u_{ \varepsilon } ( x ) \neq 0 
	\right\}
	\cap
	\left\{
	x \in \flattorus
	\, \colon \,
	u_{ \varepsilon } ( x ) \notin \{ \alpha , \beta \}
	\right\}.
\end{equation*}
For the representation (\ref{conv_of_curv_scalar_written_as_g}), we also have 
to use that 
\begin{equation*}
	\lm^{ d } \left(
	\left\{
	x \in \flattorus
	\, \colon \,
	\nabla u_{ \varepsilon } ( x ) \neq 0 \text{ and } u_{ \varepsilon } ( x ) 
	\in \{ \alpha , \beta \}
	\right\}
	\right)
	=
	0,
\end{equation*}
which is a known result for Sobolev functions.

Again by the equipartition of energies (\Cref{equipartition_of_energies}) and 
the boundedness of $ g $, we can replace $ \varepsilon \abs{ \nabla u_{ 
\varepsilon } }^{ 2 } $ by $ \sqrt{ 2 W ( u_{ \varepsilon } ) } \abs{ \nabla 
u_{ \varepsilon } } $ in the integral (\ref{conv_of_curv_scalar_written_as_g}) 
via the estimate
\begin{align*}
	& \int_{ \flattorus_{ \varepsilon } }
	\abs{ 
		\varepsilon \abs{ \nabla u_{ \varepsilon } }^{ 2 }
		-
		\sqrt{ 2 W ( u_{ \varepsilon } ) } \abs{ \nabla u_{\varepsilon } }
	}
	\dd{ x }
	\\
	\leq {} &
	\left(
	\int_{ \flattorus_{ \varepsilon } }
	\abs{ 
		\sqrt{ \varepsilon } \abs{ \nabla u_{ \varepsilon } } 
		- 
		\frac{ 1 }{ \sqrt{ \varepsilon } } \sqrt{ 2 W ( u_{ \varepsilon } ) } 
	}^{ 2 }
	\dd{ x }
	\right)^{ 1/2 }
	\left(
	\int_{ \flattorus_{ \varepsilon } }
	\varepsilon \abs{ \nabla u_{ \varepsilon } }^{ 2 }
	\dd{ x }
	\right)^{ 1/ 2 }
	\\
	\leq {} &
	\left(
	\int
	\varepsilon \abs{ \nabla u_{ \varepsilon } }^{ 2 }
	-
	2 \sqrt{ 2 W ( u_{ \varepsilon } ) } \abs{ \nabla u_{ \varepsilon } }
	+
	\frac{ 1 }{ \varepsilon }
	2 W ( u_{ \varepsilon } )
	\dd{ x }
	\right)
	\sqrt{ 2 \energy_{ \varepsilon } ( u_{ \varepsilon } ) }
\end{align*}
which vanishes as $ \varepsilon $ tends to zero.
Thus
\begin{align*}
	\lim_{ \varepsilon \to 0 }
	\int_{ \flattorus_{ \varepsilon } }
	g ( x , \nabla \psi_{ \varepsilon  } )
	\varepsilon \abs{ \nabla u_{ \varepsilon } }^{ 2 }
	\dd{ x }
	& =
	\lim_{ \varepsilon \to 0 }
	\int_{ \flattorus_{ \varepsilon } }
	g ( x , \nabla \psi_{ \varepsilon  } )
	\sqrt{ 2 W ( u_{ \varepsilon } ) } 
	\abs{ u_{ \varepsilon } }
	\dd{ x }
	\\
	& =
	\lim_{ \varepsilon \to 0 }
	\int_{ \flattorus_{ \varepsilon } }
	g ( x , \nabla \psi_{ \varepsilon  } ) \abs{ \nabla \psi_{ \varepsilon  } }
	\dd{ x }
	\\
	& = 
	\lim_{ \varepsilon \to 0 }
	\int
	F ( x, \nabla \psi_{ \varepsilon } )
	\dd{ x },
\end{align*}
where $ F ( x, p ) $ is defined as $ g ( x , p ) \abs{ p } $ at points where $ 
p $ is
not equal to 0, and defined as 0 elsewhere. Since $ F (x , \lambda p ) = 
\lambda F ( x, p ) $ for positive $ \lambda $ and since $ F $ satisfies the 
periodic boundary condition in $ x $, we are in the position to apply a 
Theorem proven by Reshetnyak in \cite{Reshetnyak_weak_convergence} and again by 
Luckhaus and Modica in \cite{luckhaus_modica_gibbs_thompson_relation}. We 
will later see a quantitative version of this in the proof of
\Cref{convergence_of_curvature_multiphase}. 
Here it yields that since $ \abs{ \nabla \psi_{ 
\varepsilon  } } ( \flattorus ) \to \abs{ \nabla \psi } ( \flattorus ) $ by the 
equipartition of energies (\Cref{equipartition_of_energies}), we obtain
\begin{align*}
	\lim_{ \varepsilon \to 0 }
	\int
	F ( x, \nabla \psi_{ \varepsilon } )
	\dd{ x }
	& =
	\sigma
	\int
	F ( x , \nu )
	\abs{ \nabla \chi }
	\\
	& = 
	\sigma
	\int
	\left(
	\sum_{ i, j = 1}^{ d }
	-
	\frac{ \nu_{ i } }{ \abs{ \nu } }
	\partial_{ x_{ i } } \xi^{ j }
	\frac{ \nu_{ j } }{ \abs{ \nu } }
	+
	\divg \xi 
	\right)
	\abs{ \nu }
	\abs{ \nabla \chi }
	\\
	& =
	\sigma
	\int
	\inner*{ \diff \xi }{ \mathrm{Id} - \nu \otimes \nu }
	\abs{ \nabla \chi },
\end{align*}
which finishes the proof.
The time integrated version given by
\begin{equation*}
	\lim_{ \varepsilon \to 0 }
	\int_{ 0 }^{ T }
	\int
	\left(
	\varepsilon \Delta u_{ \varepsilon }
	- 
	\frac{ 1 }{ \varepsilon }
	W'( u_{ \varepsilon } )
	\right)
	\inner*{ \xi }{ \nabla u_{ \varepsilon } }
	\dd{ x }
	\dd{ t }
	=
	\sigma
	\int_{ 0 }^{ T }
	\int
	\inner*{ \diff \xi }{ \mathrm{Id} - \nu \otimes \nu }
	\abs{ \nabla \chi }
	\dd{ t }
\end{equation*} 
follows from the generalized dominated convergence theorem via the equality 
(\ref{curvature_term_equality_which_bounds}) which yields
\begin{equation*}
	\abs{
		\int
		\left(
		\varepsilon \Delta u_{ \varepsilon } 
		-
		\frac{ 1 }{ \varepsilon }
		W'( u_{ \varepsilon } )
		\right)
		\inner*{ \xi } { \nabla u_{ \varepsilon } }
		\dd{ x }
	}
	\lesssim
	\energy_{ \varepsilon } ( u_{ \varepsilon } ).
\end{equation*}


\subsection{Convergence of the velocity term}

We now want to prove the convergence of the velocity term given by
\begin{equation*}
	\lim_{ \varepsilon \to 0 }
	\int_{ 0 }^{ T }
	\int
	\partial_{ t } u_{ \varepsilon }
	\inner*{ \xi }{ \varepsilon \nabla u_{ \varepsilon } }
	\dd{ x }
	\dd{ t }
	=
	\sigma
	\int_{ 0 }^{ T }
	\int
	V \inner*{ \xi }{ \nu }
	\abs{ \nabla \chi }
	\dd{ t }.
\end{equation*}
The difficulty here is that products of weakly converging sequences will in 
general not weakly converge. To be more precise, we only have $ \partial_{ t } 
u_{ \varepsilon } \rightharpoonup V \abs{ \nabla \chi } \dd{ t } $ and 
$ \varepsilon \nabla u_{ \varepsilon } \approx \sigma \nu $ in a 
weak sense. 

Therefore we try to freeze the normal in a fixed direction, apply the weak 
convergence of $ \partial_{ t } u_{ \varepsilon } $ and then unfreeze the 
normal. Freezing the approximate normal $ \varepsilon \nabla u_{ \varepsilon } 
$ amounts to replacing $ \varepsilon \nabla u_{ 
\varepsilon } $ by $ 
\varepsilon \abs{ \nabla u_{ \varepsilon } } \nu^{ \ast } $ for a suitably 
chosen $ \nu^{ \ast } \in \mathbb{ S }^{ d-1 } $. Let $ \eta $ be a cutoff on 
the support of $ \xi $ and denote by $ \nu_{ \varepsilon } = \nabla u_{ 
\varepsilon } / \abs{ \nabla u_{ \varepsilon } } $ the approximate unit normal. 
Then the error we make can be estimated for all $ 
\alpha > 0 $ via Young's inequality by
\begin{align}
	\notag
	&\abs{
		\int_{ 0 }^{ T }
		\int
		\partial_{ t } u_{ \varepsilon }
		\inner*{ \xi }{ \nabla u_{ \varepsilon } }
		\dd{ x }
		\dd{ t }
		-
		\int_{ 0 }^{ T }
		\int
		\partial_{ t } u_{ \varepsilon }
		\inner*{ \xi }{ \varepsilon \abs{ \nabla u_{ \varepsilon } } \nu^{ \ast 
		} }
		\dd{ x }
		\dd{ t }
	}
	\\
	\notag
	\leq {} &
	\norm{ \xi }_{ \lp^{ \infty } }
	\int_{ 0 }^{ T }
	\int
	\eta
	\sqrt{ \varepsilon }
	\abs{ \partial_{  t } u_{ \varepsilon } }
	\sqrt{  \varepsilon } 
	\abs{
		\nabla u_{ \varepsilon }
		-
		\abs{ \nabla u_{ \varepsilon } }
		\nu^{ \ast }
	}
	\dd{ x } 
	\dd{ t }
	\\
	\notag
	\leq {} &
	\norm{ \xi }_{ \lp^{ \infty } }
	\left(
	\alpha 
	\int_{ 0 }^{ T }
	\int
	\eta
	\varepsilon \abs{ \partial_{ t } u_{ \varepsilon } }^{ 2 }
	\dd{ x }
	\dd{ t }
	+
	\frac{ 1 }{ \alpha }
	\int_{ 0 }^{ T }
	\int
	\eta
	\varepsilon
	\abs{ \nabla u_{ \varepsilon } }^{ 2 }
	\abs{
		\nu_{ \varepsilon } - \nu^{ \ast } 
	}^{ 2 } 
	\dd{ x }
	\dd{ t }
	\right)
	\\
	\label{acceptable_error_term_velocity_twophase}
	\leq{}
	&
	\norm{ \xi }_{ \lp^{ \infty } }
	\left( 
	\alpha 	
	\int_{ 0 }^{ T }
	\int
	\eta
	\varepsilon \abs{ \partial_{ t } u_{ \varepsilon } }^{ 2 }
	\dd{ x }
	\dd{ t }
	+ 
	\frac{ 1 }{ \alpha }
	\tilt_excess_{ \varepsilon } ( \nu^{ \ast } ; \eta ) 
	\right).
\end{align}
Here the approximate tilt-excess in direction $ \nu^{ \ast } $ is given by
\begin{equation*}
	\tilt_excess_{ \varepsilon } ( \nu^{ \ast } ; \eta )
	\coloneqq
	\int_{ 0 }^{ T }
	\int
	\eta
	\varepsilon 
	\abs{ \nabla u_{ \varepsilon } }^{ 2 }
	\abs{ \nu_{ \varepsilon } - \nu^{ \ast } }^{ 2 }
	\dd{ x }
	\dd{ t }.
\end{equation*}
So let us accept the error term \ref{acceptable_error_term_velocity_twophase} 
for now.
With our frozen normal, we now notice that via the equipartition of energies, 
we may replace $ \varepsilon \abs{ \nabla u_{ \varepsilon } } $ by $ \sqrt{ 2 W 
( u_{ \varepsilon } ) } $ via the estimate
\begin{align*}
	& 
	\int_{ 0 }^{ T }
	\int
	\abs{ \partial_{ t } u_{ \varepsilon } } \eta
	\abs{ \varepsilon \abs{ \nabla u_{ \varepsilon } } - \sqrt{ 2 W ( u_{ 
	\varepsilon } ) } }
	\dd{ x }
	\dd{ t }
	\\
	\leq{} &
	\left(
	\int_{ 0 }^{ T }
	\int
	\varepsilon 
	\abs{ \partial_{ t } u_{ \varepsilon } }^{ 2 }
	\dd{ x }
	\dd{ t }
	\right)^{ 1/2 }
	\left(
	\int_{ 0 }^{ T }
	\int
	\eta^{ 2 }
	\left(
	\varepsilon \abs{ \nabla u_{ \varepsilon } }^{ 2 }
	-
	2 \abs{ \nabla u_{ \varepsilon } } \sqrt{ 2 W ( u_{ \varepsilon } ) }
	+
	\frac{ 1 }{ \varepsilon }
	2 W ( u_{ \varepsilon } ) 
	\right)
	\dd{ x }
	\dd{ t }
	\right)^{ 1 / 2 }
\end{align*}
The first factor is uniformly bounded by the energy dissipation inequality 
(\ref{energy_dissipation_sharp}) and the second term vanishes as $ \varepsilon 
$ tends to zero by the equipartition of energies 
(\Cref{equipartition_of_energies}).
But now we recognize the identity
\begin{equation*}
	\int_{ 0 }^{ T }
	\int
	\partial_{ t } u_{ \varepsilon }
	\sqrt{ 2 W ( u_{ \varepsilon } ) }
	\inner*{ \xi }{ \nu^{ \ast } }
	\dd{ x }
	\dd{ t }
	=
	\int_{ 0 }^{ T }
	\int
	\partial_{ t } \psi_{ \varepsilon }
	\inner*{ \xi }{ \nu^{ \ast } }
	\dd{ x }
	\dd{ t },
\end{equation*}
which converges as $ \varepsilon $ approaches zero to 
\begin{equation}
	\label{rewriting_limit_of_frozen_velocity_term}
	\int_{ 0 }^{ T }
	\int
	\inner*{ \xi }{ \nu^{ \ast } }
	\partial_{ t } \psi
	=
	\sigma
	\int_{ 0 }^{ T }
	\int
	V \inner*{ \xi }{ \nu^{ \ast } }
	\abs{ \nabla \chi }
	\dd{ t }.
\end{equation}
Finally we want to unfreeze the normal, which means that we want to replace $ 
\nu^{ \ast } $ by $ \nu $ on the right hand side of equation 
(\ref{rewriting_limit_of_frozen_velocity_term}). This can be estimated again by 
Young's inequality via the error
\begin{equation*}
	\norm{ \xi }_{ \lp^{ \infty } }
	\left(
	\alpha
	\int_{ 0 }^{ T }
	\int
	\eta
	V^{ 2 }
	\abs{ \nabla \chi }
	\dd{ t }
	+
	\frac{ 1 }{ \alpha }
	\tilt_excess ( \nu^{ \ast }; \eta )
	\right),
\end{equation*}
where the tilt-excess is given by 
\begin{equation*}
	\tilt_excess ( \nu^{ \ast } ; \eta )
	\coloneqq
	\sigma
	\int_{ 0 }^{ T }
	\int
	\eta
	\abs{ \nu - \nu^{ \ast } }^{ 2 }
	\abs{ \nabla \chi }
	\dd{ t }.
\end{equation*}
This finishes the proof for the convergence of the velocity term up to arguing 
that the errors can be made arbitrarily small.

First we study the behaviour of the approximate tilt-excess $ \tilt_excess_{ 
\varepsilon } $ by connecting it to $ \tilt_excess $. We notice that by 
expansion, we have
\begin{align*}
	\tilt_excess_{ \varepsilon } ( \nu^{ \ast } ; \eta )
	& =
	2
	\int_{ 0 }^{ T }
	\int
	\eta
	\varepsilon 
	\abs{ \nabla u_{ \varepsilon } }^{ 2 }
	\dd{ x }
	\dd{ t }
	-
	2 
	\inner*{\int_{ 0 }^{ T }
		\int
		\eta
		\varepsilon
		\abs{ \nabla u_{ \varepsilon } }
		\nabla u_{ \varepsilon }
		\dd{ x }
		\dd{ t } }
	{ \nu^{ \ast } }
	\\
	\tilt_excess ( \nu^{ \ast } ; \eta )
	& =
	2 \energy ( u ; \eta )
	-
	2 \sigma
	\inner*{
		\int_{ 0 }^{ T }
		\int
		\eta
		\nu 
		\abs{ \nabla \chi }
		\dd{ t }
	}{ \nu^{ \ast } }.
\end{align*}
But by the equipartition of energies \Cref{equipartition_of_energies}, we have 
that
\begin{equation*}
	2 \int_{ 0 }^{ T }
	\int
	\eta \varepsilon 
	\abs{ \nabla u_{ \varepsilon } }^{ 2 }
	\dd{ x }
	\dd{ t }
	\to 
	2 \energy ( u ; \eta )
\end{equation*} 
and recognizing that $ \sigma \int_{0 }^{ T } \int \eta \nu \abs{ \nabla \chi } 
\dd{ t } = \int_{ 0 }^{ T } \int \eta \nabla \psi $, we also obtain
\begin{align*}
	& 
	\abs{ 
		\int_{ 0 }^{ T }
		\int
		\eta
		\varepsilon
		\abs{ \nabla u_{ \varepsilon } }
		\nabla u_{ \varepsilon }
		\dd{ x }
		\dd{ t }
		-
		\sigma
		\int_{ 0 }^{ T }
		\int
		\eta
		\nu 
		\abs{ \nabla \chi }
		\dd{ t }
	}
	\\
	\leq {} &
	\abs{
		\int_{ 0 }^{ T }
		\int	
		\eta
		\sqrt{ 2 W ( u_{ \varepsilon } ) }
		\nabla u_{ \varepsilon }
		\dd{ x }
		\dd{ t }
		-
		\int_{ 0 }^{ T }
		\int
		\eta
		\nabla \psi
	}
	+
	\int_{ 0 }^{ T }
	\int
	\varepsilon
	\abs{ \nabla u_{ \varepsilon } }
	\abs{ \nabla u_{ \varepsilon } - \frac{ 1 }{ \varepsilon } \sqrt{ 2 W ( u_{ 
	\varepsilon } ) } }
	\dd{ x }
	\dd{ t }
	\\
	\leq {} &
	\abs{
		\int_{ 0 }^{ T }
		\int
		\eta
		\nabla \psi_{ \varepsilon } 
		\dd{ x }
		\dd{ t }
		-
		\int_{ 0 }^{ T }
		\int
		\eta
		\nabla \psi 
	}
	\\
	& +
	\left(
	\int_{ 0 }^{ T }
	\int
	\varepsilon \abs{ \nabla u_{ \varepsilon } }^{ 2 }
	\dd{ x }
	\dd{ t }
	\right)^{ 1/2 }
	\left(
	\int_{ 0 }^{ T }
	\int
	\left(
	\sqrt{ \varepsilon }
	\abs{ \nabla u_{ \varepsilon } }
	-
	\frac{ 1 }{ \sqrt{ \varepsilon } }
	\sqrt{ 2 W ( u_{ \varepsilon } ) }
	\right)^{ 2 }
	\dd{ x }
	\dd{ t }
	\right)^{ 1/2 },
\end{align*}
which vanishes as $ \varepsilon $ tends to zero by the weak convergence of $ 
\nabla \psi_{ \varepsilon } \rightharpoonup^{ \ast } \nabla \psi $ and the 
equipartition of the energies.
Since we are taking $ \varepsilon $ to zero, we thus only have to make sure 
that 
the tilt-excess is sufficiently small since the approximate tilt excess 
converges to the tilt excess.

We are now in the position to argue why the error can be made arbitrarily 
small. 
Let $ \delta > 0 $. Then we first choose our $ \alpha > 0 $ so small that 
\begin{equation*}
	\limsup_{ \varepsilon \to 0 }
	\alpha \norm{ \xi }_{ \lp^{ \infty } }
	\left(
	\int_{ 0 }^{ T }
	\int
	\varepsilon 
	\abs{ \partial_{  t } u_{ \varepsilon } }^{ 2 }
	\dd{ x }
	\dd{ t }
	+
	\int_{ 0 }^{ T }
	\int
	V^{ 2 }
	\abs{ \nabla \chi }
	\dd{ t }
	\right)
	<
	\frac{ \delta }{ 2 },
\end{equation*}
which is possible by the energy dissipation inequality 
(\ref{energy_dissipation_sharp}) and the square-integrability of the normal 
velocity (\Cref{existence_of_velocity_twophase}).
Then we choose a partition of unity $ \left( \eta_{ i } \right)_{ i = 1 , 
\dotsc , n } $ and approximate unit normals $ \left( \nu^{ \ast }_{ i 
}\right)_{i = 1 , \dotsc, n } $ such that
\begin{equation*}
	\frac{ 2 }{ \alpha }
	\norm{ \xi }_{ \lp^{ \infty } }
	\sigma
	\sum_{ i = 1 }^{ n }
	\int_{ 0 }^{ T }
	\int
	\eta_{ i }
	\abs{ \nu - \nu_{ i }^{ \ast } }^{ 2 }
	\abs{ \nabla \chi }
	\dd{ t }
	<
	\frac{ \delta }{2 }.
\end{equation*}
The existence of these can be seen by taking a smooth approximation of $ \nu $ 
with respect to the measure $ \abs{ \nabla \chi } \dd{ t } $.
Collecting all of our errors, we obtain
\begin{align*}
	& \limsup_{ \varepsilon \to 0 }
	\abs{
		\int_{ 0 }^{ T }
		\int
		\partial_{  t } u_{ \varepsilon }
		\inner*{ \xi }{ \varepsilon \nabla u_{ \varepsilon } }
		\dd{ x }
		\dd{ t }
		-
		\sigma
		\int_{ 0 }^{ T }
		\int
		V \inner*{ \xi }{ \nu }
		\abs{ \nabla \chi }
		\dd{ t }
	}
	\\
	= {} &
	\limsup_{ \varepsilon \to 0 }
	\abs{
		\sum_{ i = 1 }^{ n }
		\int_{ 0 }^{ T }
		\int
		\eta_{ i }
		\partial_{  t } u_{ \varepsilon }
		\inner*{ \xi }{ \varepsilon \nabla u_{ \varepsilon } }
		\dd{ x }
		\dd{ t }
		-
		\sigma
		\int_{ 0 }^{ T }
		\int
		\eta_{ i }
		V \inner*{ \xi }{ \nu }
		\abs{ \nabla \chi }
		\dd{ t }
	}
	\\
	\leq {} & 
	\limsup_{ \varepsilon \to 0 }
	\norm{ \xi }_{ \lp^{ \infty } }
	\left(
	\sum_{ i = 1 }^{ n } 
	\alpha 
	\int_{ 0 }^{ T }
	\left(
	\int
	\eta_{ i }
	\varepsilon
	\abs{ \partial_{  t } u_{ \varepsilon } }^{ 2 }
	\dd{ x }
	+
	\int
	\eta_{ i }
	V^{ 2 }
	\abs{ \nabla \chi }
	\right)
	\dd{ t }
	+
	\frac{ 2 }{ \alpha }
	\tilt_excess ( \nu_{ i }^{ \ast } ; \eta_{ i } )
	\right)
	< \delta,
\end{align*}
which finishes the proof.